\batchmode
\documentclass[12pt]{article}
\makeatletter


\oddsidemargin -.5in
\textwidth 7in
\textheight 8.5in
\headsep 0in
\headheight 0in


\renewcommand{\baselinestretch}{1.2}


\usepackage[dvips]{color}
\pagecolor[gray]{.7}



\makeatletter

\makeatletter
\count@=\the\catcode`\_ \catcode`\_=8 
\newenvironment{tex2html_wrap}{}{} \catcode`\_=\count@
\makeatother
\let\mathon=$
\let\mathoff=$
\ifx\AtBeginDocument\undefined \newcommand{\AtBeginDocument}[1]{}\fi
\newbox\sizebox
\setlength{\hoffset}{0pt}\setlength{\voffset}{0pt}
\addtolength{\textheight}{\footskip}\setlength{\footskip}{0pt}
\addtolength{\textheight}{\topmargin}\setlength{\topmargin}{0pt}
\addtolength{\textheight}{\headheight}\setlength{\headheight}{0pt}
\addtolength{\textheight}{\headsep}\setlength{\headsep}{0pt}
\setlength{\textwidth}{349pt}
\newwrite\lthtmlwrite
\makeatletter
\let\realnormalsize=\normalsize
\global\topskip=2sp
\def\preveqno{}\let\real@float=\@float \let\realend@float=\end@float
\def\@float{\let\@savefreelist\@freelist\real@float}
\def\end@float{\realend@float\global\let\@freelist\@savefreelist}
\let\real@dbflt=\@dbflt \let\end@dblfloat=\end@float
\let\@largefloatcheck=\relax
\def\@dbflt{\let\@savefreelist\@freelist\real@dbflt}
\def\adjustnormalsize{\def\normalsize{\mathsurround=0pt \realnormalsize
 \parindent=0pt\abovedisplayskip=0pt\belowdisplayskip=0pt}\normalsize}%
\def\lthtmltypeout#1{{\let\protect\string\immediate\write\lthtmlwrite{#1}}}%
\newcommand\lthtmlhboxmathA{\adjustnormalsize\setbox\sizebox=\hbox\bgroup}%
\newcommand\lthtmlvboxmathA{\adjustnormalsize\setbox\sizebox=\vbox\bgroup%
 \let\ifinner=\iffalse }%
\newcommand\lthtmlboxmathZ{\@next\next\@currlist{}{\def\next{\voidb@x}}%
 \expandafter\box\next\egroup}%
\newcommand\lthtmlmathtype[1]{\def\lthtmlmathenv{#1}}%
\newcommand\lthtmllogmath{\lthtmltypeout{l2hSize %
:\lthtmlmathenv:\the\ht\sizebox::\the\dp\sizebox::\the\wd\sizebox.\preveqno}}%
\newcommand\lthtmlfigureA[1]{\let\@savefreelist\@freelist
       \lthtmlmathtype{#1}\lthtmlvboxmathA}%
\newcommand\lthtmlfigureZ{\lthtmlboxmathZ\lthtmllogmath\copy\sizebox
       \global\let\@freelist\@savefreelist}%
\newcommand\lthtmldisplayA[1]{\lthtmlmathtype{#1}\lthtmlvboxmathA}%
\newcommand\lthtmldisplayB[1]{\edef\preveqno{(\theequation)}%
  \lthtmldisplayA{#1}\let\@eqnnum\relax}%
\newcommand\lthtmldisplayZ{\lthtmlboxmathZ\lthtmllogmath\lthtmlsetmath}%
\newcommand\lthtmlinlinemathA[1]{\lthtmlmathtype{#1}\lthtmlhboxmathA  \vrule height1.5ex width0pt }%
\newcommand\lthtmlinlineA[1]{\lthtmlmathtype{#1}\lthtmlhboxmathA}%
\newcommand\lthtmlinlineZ{\egroup\expandafter\ifdim\dp\sizebox>0pt %
  \expandafter\centerinlinemath\fi\lthtmllogmath\lthtmlsetinline}
\newcommand\lthtmlinlinemathZ{\egroup\expandafter\ifdim\dp\sizebox>0pt %
  \expandafter\centerinlinemath\fi\lthtmllogmath\lthtmlsetmath}
\def\lthtmlsetinline{\hbox{\vrule width.1em\vtop{\vbox{%
  \kern.1em\copy\sizebox}\ifdim\dp\sizebox>0pt\kern.1em\else\kern.3pt\fi
  \ifdim\hsize>\wd\sizebox \hrule depth1pt\fi}}}
\def\lthtmlsetmath{\hbox{\vrule width.1em\vtop{\vbox{%
  \kern.1em\kern0.8 pt\hbox{\hglue.17em\copy\sizebox\hglue0.8 pt}}\kern.3pt%
  \ifdim\dp\sizebox>0pt\kern.1em\fi \kern0.8 pt%
  \ifdim\hsize>\wd\sizebox \hrule depth1pt\fi}}}
\def\centerinlinemath{%\dimen1=\ht\sizebox
  \dimen1=\ifdim\ht\sizebox<\dp\sizebox \dp\sizebox\else\ht\sizebox\fi
  \advance\dimen1by.5pt \vrule width0pt height\dimen1 depth\dimen1 
 \dp\sizebox=\dimen1\ht\sizebox=\dimen1\relax}

\def\lthtmlcheckvsize{\ifdim\ht\sizebox<\vsize\expandafter\vfill
  \else\expandafter\vss\fi}%
\makeatletter \tracingstats = 1 


\begin{document}
\pagestyle{empty}\thispagestyle{empty}%
\lthtmltypeout{latex2htmlLength hsize=\the\hsize}%
\lthtmltypeout{latex2htmlLength vsize=\the\vsize}%
\lthtmltypeout{latex2htmlLength hoffset=\the\hoffset}%
\lthtmltypeout{latex2htmlLength voffset=\the\voffset}%
\lthtmltypeout{latex2htmlLength topmargin=\the\topmargin}%
\lthtmltypeout{latex2htmlLength topskip=\the\topskip}%
\lthtmltypeout{latex2htmlLength headheight=\the\headheight}%
\lthtmltypeout{latex2htmlLength headsep=\the\headsep}%
\lthtmltypeout{latex2htmlLength parskip=\the\parskip}%
\lthtmltypeout{latex2htmlLength oddsidemargin=\the\oddsidemargin}%
\makeatletter
\if@twoside\lthtmltypeout{latex2htmlLength evensidemargin=\the\evensidemargin}%
\else\lthtmltypeout{latex2htmlLength evensidemargin=\the\oddsidemargin}\fi%
\makeatother

% !!! IMAGES START HERE !!!

{\newpage\clearpage
\lthtmlinlinemathA{tex2html_wrap_inline102}%
$A^Ty \leq 0$%
\lthtmlinlinemathZ
\hfill\lthtmlcheckvsize\clearpage}

{\newpage\clearpage
\lthtmlinlinemathA{tex2html_wrap_inline104}%
$A^Ty \neq 0$%
\lthtmlinlinemathZ
\hfill\lthtmlcheckvsize\clearpage}

{\newpage\clearpage
\lthtmldisplayA{displaymath18}%
\begin{displaymath}
      \begin{array}{lrcrcrcl}
        \min               & 2x_1 & - & 3x_2 & + & 4x_3 \\
        \mbox{subject to } & x_1 & - & x_2 & + & 3x_3 & = & 2 \\
                           & x & \in & X
      \end{array}
    \end{displaymath}%
\lthtmldisplayZ
\hfill\lthtmlcheckvsize\clearpage}

{\newpage\clearpage
\lthtmlinlinemathA{tex2html_wrap_inline106}%
$X=\{x:0\leq x_1 \leq 1, 0 \leq x_2, 0\leq x_3 \leq 1\}$%
\lthtmlinlinemathZ
\hfill\lthtmlcheckvsize\clearpage}

{\newpage\clearpage
\lthtmlinlinemathA{tex2html_wrap_inline112}%
$x=(0,0,\frac{2}{3})$%
\lthtmlinlinemathZ
\hfill\lthtmlcheckvsize\clearpage}

{\newpage\clearpage
\lthtmldisplayA{displaymath29}%
\begin{displaymath}
x_k = \left\{ \begin{array}{ll}
1 & \mbox{if subset } $k$\space \mbox{ is one of the $K$\space chosen clusters}\\
0 & \mbox{otherwise.}
\end{array}   \right.
\end{displaymath}%
\lthtmldisplayZ
\hfill\lthtmlcheckvsize\clearpage}

{\newpage\clearpage
\lthtmlinlinemathA{tex2html_wrap_inline140}%
$k=1,\ldots,2^n-1$%
\lthtmlinlinemathZ
\hfill\lthtmlcheckvsize\clearpage}

{\newpage\clearpage
\lthtmldisplayA{displaymath37}%
\begin{displaymath}
w_k := \sum_{i,j \in S^k, i<j} d_{ij}
\end{displaymath}%
\lthtmldisplayZ
\hfill\lthtmlcheckvsize\clearpage}

{\newpage\clearpage
\lthtmldisplayA{displaymath41}%
\begin{displaymath}
\begin{array}{lrclr}
\min & \sum_{k=1}^{2^n-1} w_k x_k \\
\mbox{subject to } & A x & = & e & \qquad \qquad (MP) \\
& e^Tx & = & K \\
&   x  & \geq & 0
\end{array}
\end{displaymath}%
\lthtmldisplayZ
\hfill\lthtmlcheckvsize\clearpage}

{\newpage\clearpage
\lthtmldisplayA{displaymath51}%
\begin{displaymath}
D := \left( \begin{array}{rrrrr} 0 & 8 & 21 & 17 & 18 \\
& 0 & 22 & 15 & 20 \\&& 0 & 2 & 4 \\
&&& 0 & 3 \\&&&& 0
\end{array}  \right) .
\end{displaymath}%
\lthtmldisplayZ
\hfill\lthtmlcheckvsize\clearpage}

{\newpage\clearpage
\lthtmlinlinemathA{tex2html_wrap_inline188}%
$\bar{x}$%
\lthtmlinlinemathZ
\hfill\lthtmlcheckvsize\clearpage}

{\newpage\clearpage
\lthtmldisplayA{displaymath62}%
\begin{displaymath}
\begin{array}{lrclllrcll}
\min & c^Tx &&&& \max & b^Ty \\
\mbox{subject to} & Ax & = & b & \quad (P) \qquad &
   \mbox{subject to} & A^Ty & \leq & c  & \quad (D)  \\
& x & \geq & 0
\end{array}
\end{displaymath}%
\lthtmldisplayZ
\hfill\lthtmlcheckvsize\clearpage}

{\newpage\clearpage
\lthtmlinlinemathA{tex2html_wrap_inline216}%
$\hat{b}=b-Ae$%
\lthtmlinlinemathZ
\hfill\lthtmlcheckvsize\clearpage}

{\newpage\clearpage
\lthtmlinlinemathA{tex2html_wrap_inline218}%
$\hat{c}=c-e$%
\lthtmlinlinemathZ
\hfill\lthtmlcheckvsize\clearpage}

{\newpage\clearpage
\lthtmlinlinemathA{tex2html_wrap_inline222}%
$\hat{d}=c^Te+1$%
\lthtmlinlinemathZ
\hfill\lthtmlcheckvsize\clearpage}

{\newpage\clearpage
\lthtmldisplayA{displaymath72}%
\begin{displaymath}
\begin{array}{lrcrcrcccl}
\min &&&&&&& (n+1)w \\
\mbox{subject to} &&& Ax & - & bt & + & \hat{b} w & = & 0  \\
& -A^Ty &&& + & ct & - & \hat{c} w & \geq & 0 \qquad (HLP)  \\
& b^Ty & - & c^T x &&& + & \hat{d} w & \geq & 0 \\
& -\hat{b}^Ty & + & \hat{c}^Tx & - & \hat{d} t &&& = & -(n+1)  \\
&&& x,t & \geq & 0, && y,w &&\mbox{free,}
\end{array}
\end{displaymath}%
\lthtmldisplayZ
\hfill\lthtmlcheckvsize\clearpage}

{\newpage\clearpage
\lthtmlinlinemathA{tex2html_wrap_inline240}%
$(\tilde{x},\tilde{y},\tilde{t},\tilde{w})$%
\lthtmlinlinemathZ
\hfill\lthtmlcheckvsize\clearpage}

{\newpage\clearpage
\lthtmlinlinemathA{tex2html_wrap_inline242}%
$\tilde{t}>0$%
\lthtmlinlinemathZ
\hfill\lthtmlcheckvsize\clearpage}

{\newpage\clearpage
\lthtmlinlinemathA{tex2html_wrap_inline244}%
$\tilde{w}=0$%
\lthtmlinlinemathZ
\hfill\lthtmlcheckvsize\clearpage}

{\newpage\clearpage
\lthtmlinlinemathA{tex2html_wrap_inline254}%
$x\geq 0$%
\lthtmlinlinemathZ
\hfill\lthtmlcheckvsize\clearpage}


\end{document}
