\documentclass[12pt]{article}

\pagestyle{empty}

\usepackage{fullpage}

%\oddsidemargin -.5in
%\textwidth 7.5in
%\textheight 10in
%\topmargin -30pt
%\headsep 0in
%\headheight 0in

\newcommand{\re}{I \!\! R}

\begin{document}

\begin{center}
  \begin{large}
     MATP6640/DSES6770 Linear Programming, Homework 6.
  \end{large}
\end{center}

\begin{flushright}
   Due:  Tuesday, May 2, 2006.
\end{flushright}


%\vspace{\baselineskip}

%Note: In the following, $P_M$ denotes projection onto the nullspace
%of the matrix~$M$. If $M$ has full row rank, then $P_M$ is given by
%\begin{displaymath}
%P_M = I - M^T (MM^T)^{-1} M.
%\end{displaymath}
%Note that if $M$ is a $m \times n$ matrix then $P_M$ is an $n \times n$
%matrix.

\begin{enumerate}
%  \item  Let $f(x)=n \log (c^Tx) - \sum_{i=1}^n \log(x_i)$.
%        Show that $f(x+\theta d)=f(x)$ if $d=kx$ for some scalar $k$,
%        provided $x>0$, $x+\theta d>0$, $c^Tx>0$, and $c^T(x+\theta d)>0$.
%  \item
%Starting from equation (1.20) in {\em Wright},
%show that there is a typo in equation (1.25a). It should read:
%\begin{displaymath}
%A D^2 A^T \Delta \lambda =
%-r_b + A ( {\mathbf -}S^{-1}X r_c +x - \sigma \mu S^{-1} e).
%\end{displaymath}
%(Note: your version of the text may not contain the typo!)
\item
Let $K$ be a cone. A function $f:\mbox{int}(K) \rightarrow \re$ is
{\em logarithmically homogeneous} if there exists a constant $\Theta$
such that $f(tx)=f(x)-\Theta\ln(t)$ for all $x\in\mbox{int}(K)$ and $t>0$.
(Here, $\mbox{int}(K)$ denotes the interior of~$K$.)
Show that the barrier function for $\re^n$, namely $f(x)=-\sum_{i=1}^n\ln(x_i)$,
and the barrier function for the semidefinite cone, namely $f(X)=-\ln\det(X)$,
are both logarithmically homogeneous.
%  \item
%    Consider the linear programming problem $(P)$
%    \begin{displaymath}
%      \begin{array}{lrcrcrcr}
%        \min        & 9x_1 & + & 6x_2 & - & 7x_3 \\
%        \mbox{s.t.} &  x_1 & + & 2x_2 & + & 3x_3 & = & 6  \\
%                    && x_1,& x_2,& x_3 & \geq & 0.
%      \end{array}
%    \end{displaymath}
%    Assume the current point is $e=(1,1,1)^T$.
%    Calculate the affine descent direction $-P_Ac$ and the centering direction
%    $P_Ae$.  We wish to move in the direction $d=-P_Ac+\beta P_Ae$ for some
%    $\beta \geq 0$.  Show that by choosing $\beta$ appropriately, we can find
%    a steplength $\alpha \geq 0$ such that $e+\alpha d$ is optimal for
%    the linear program.
%    (Note: For this problem use the primal rescaling matrix~$X$.
%    For the given point, we have $X=I$, the identity matrix.
%    The projection matrix $P_A=I-A^T(AA^T)^{-1}A$.)
%  \item
%    Assume the standard form linear programming problem
%    $\min\{c^Tx: Ax=b, x \geq 0\}$
%    has a bounded set of optimal solutions.
%    Let $\mu > 0$ and let $x$ be a strictly feasible solution.
%    Assume $A$ has full row rank.
%    Let $s^*$ solve the problem $\min \{|| Xs - \mu e ||^2 : A^Ty +s = c \}$.
%    Find~$s^*$. Show that if the optimal value is smaller than $\mu^2$
%    then $s^* > 0$.
%  \item
%    For the problem $(P)$ in question~1, with the given primal point $x=e$,
%    find the choice of $y^*$, $s^*$, and $\mu^*$ which minimizes
%    $\{|| Xs - \mu e ||^2 : A^Ty +s = c \}$.
%     Verify that the optimal value is smaller than~$(\mu^*)^2$.
%  \item
%    Show that the standard dual form $\max\{b^Ty: A^Ty+s=c, s \geq 0\}$
%    is equivalent to a problem of the form
%    $\min\{g^Ts : Ms = Mc, s \geq 0\}$
%    for some matrix $M$ and some vector~$g$.
%    You may assume the rows of $A$ are linearly independent.
%  \item
%    Wright, Chapter 5, page 105, question 7.
 \item
   Let $x \in \re^n$. Show that the constraint
   \begin{displaymath}
   x_1 x_2 \geq \sum_{i=3}^n x_i^2
   \end{displaymath}
   is equivalent to a second order cone constraint.
   Hence show that the constraint that a $2\times 2$ matrix be positive semidefinite
   is equivalent to two linear constraints and a second order cone constraint.
  \item
    Let $X$ be a symmetric real $n \times n$ matrix.
    Show that $X$ is positive semidefinite if and only if
    $\mbox{trace}(MX) \geq 0$ for all symmetric positive semidefinite
    $n \times n$ rank one matrices~$M=uu^T$ where $||u||=1$.
    Hence show that the requirement that $X$ be positive semidefinite is
    equivalent to a collection of linear constraints.
  \item
    Most semidefinite relaxations of combinatorial optimization problems
    result in a linear constraint on the trace of the primal matrix~$X$.
    For example, in the relaxation of MaxCut, the diagonal entries are
    all required to equal one, so the trace must equal the number of nodes.
    Show that if the primal problem contains a constraint of the form
    trace$(X)=a$ for some constant $a$ then the feasible region
    for the dual is unbounded.
 \item
    Assume the feasible region of the SDP $\min\{\mbox{trace}(CX): \mathcal{A}(X)=b, X \succeq 0\}$
    is bounded.
    Give an equivalent SDP which includes a linear equality constraint on the 
    trace of the primal matrix.
\end{enumerate}


\vfill

\begin{tabular}{@{\hspace{.5in}}l}
   John Mitchell  \\
   Amos Eaton 325  \\
   x6915.  \\
   mitchj@rpi.edu  \\
   Office hours: Tuesday: 2 -- 3.30pm.
\end{tabular}

\end{document}
