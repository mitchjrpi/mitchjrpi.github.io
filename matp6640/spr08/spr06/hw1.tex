\documentclass[11pt]{article}
%\documentstyle[12pt]{article}


%\oddsidemargin -.5in
%\textwidth 7.5in
%\textheight 10in
%\topmargin -30pt
%\headsep 0in
%\headheight 0in

\usepackage{fullpage}

\pagestyle{empty}

\newcommand{\real}{I\!\! R}
\newcommand{\re}{I\!\! R}

\begin{document}

\begin{center}
  \begin{large}
     MATP6640/DSES6770 Linear Programming, Homework 1.
  \end{large}
\end{center}

\begin{flushright}
   Due:  Friday, January 27, 2006.
\end{flushright}

\vspace{\baselineskip}


\begin{enumerate}
  \item  Let $a\in L$, where $L$ is an affine space in $\real^n$.
         Let $L\setminus\{a\}:=
                \{x\in\real^n:x=\bar{x}-a\mbox{ for some } \bar{x} \in L\}$.
         \begin{enumerate}
           \item Let $x$ and $y$ be two points in $L\setminus\{a\}$.
             Show that any linear combination of $x$ and $y$ is also
             in $L\setminus\{a\}$.
             What do you conclude about $L\setminus\{a\}$?
           \item Let $b$ be another point in $L$.
             Let $x$ be a point in $L\setminus\{a\}$.
             Show that $x$ is also in $L\setminus\{b\}$.
             How are $L\setminus\{a\}$ and $L\setminus\{b\}$ related?
         \end{enumerate}
%  \item  In class, we saw that if a problem of the form $(P)$ has a feasible
%         solution then it has a basic feasible solution.
%         Construct a similar proof to show that if $(P)$ has an optimal
%         solution, it has a basic feasible solution that is optimal.
  \item  Consider the linear program
  % Let the system $Ax=b$ be defined by:
         \begin{displaymath}
          \begin{array}{lcccccrcrcl}
          \min & x_1 & &- & x_3 & + & 5x_4 & - & x_5 \\
                 \mbox{s.t.} &x_1  &       & - & x_3 & + & 3x_4  & - &  x_5 & = & 1  \\
                  &    &  x_2  & + & x_3 & + & 4x_4  & + & 2x_5 & = & 4  \\
                     & &       &   &     & - &  x_4  & + & 3x_5 & = & 0. \\
                     & \multicolumn{8}{r}{x_1,\ldots,x_5} & \geq & 0
          \end{array}
         \end{displaymath}
%         Let $K:=\{x \in {\real}^5:Ax=b, x \geq 0\}$.
         The point $x=(1,4,0,0,0)^T$ is a basic feasible solution for this
         problem. Find all the bases corresponding to this bfs.
         Use complementary slackness to show that this point is optimal.
%  \item  Let $Q$ be a convex polyhedron in ${\real}^n$ and let $F$ be a nonempty
%         face of $Q$.
%         Let $x$ be a point on the face $F$.
%         Show that $x$ is an extreme point of $Q$ if and only if it is an
%         extreme point of $F$.
   \item
Consider the standard form linear programming problem
\begin{displaymath}
  \begin{array}{lrclr}
      \min  &  c^Tx  &   &    \\
      \mbox{s.t. }  &  Ax  &  =  &  b  & \qquad \qquad  (P) \\
                    &   x  & \geq & 0.
  \end{array}
\end{displaymath}
Here, $A \in {\real}^{m \times n}$, the dimensions of $x$, $c$, and $b$
are defined appropriately, and $m \leq n$.
Let $K$ be the feasible region of $(P)$.
         \begin{enumerate}
            \item Construct a linear programming problem of the form $(P)$
              with dim($K)>n-m$.
            \item Construct a feasible
              linear programming problem of the form $(P)$
              with dim($K)<n-m$, $b \neq 0$, and rank($A)=m$.
            \item In part~(b),
              the linear program you defined has
              a degenerate basic feasible solution.
              What are the bases associated with that bfs?
         \end{enumerate}
%  \item  Construct a dual pair of linear programming problems where both
%         problems are infeasible.
%  \item  What is the dual of the following linear programming problem?
%         \begin{displaymath}
%           \begin{array}{lrcrcrcrcl}
%             \min        &  3x_1 & - & 5x_2 & + & 8x_3 &   &    \\
%             \mbox{s.t.} &   x_1 &   &      & - &  x_3 & = &  6  \\
%                         &       &   & 2x_2 & - &  x_3 & \geq & -5 \\
%                         &   x_1 & + &  x_2 &   &      & \leq &  8 \\
%                         &&& x_2 & \leq 0, & x_3 & \geq & 0.
%           \end{array}
%         \end{displaymath}
%         Use complementary slackness to show that $x=(6,0,0)$ is
%         optimal for this problem.
%         Find a linear programming problem in standard form which is
%         equivalent to this problem.
%   \item  \label{qcs}
%     Consider the linear programming problem
%     \begin{displaymath}
%       \begin{array}{lrcrcrcrcrclr}
%         \min         &  x_1 & + &  x_2 & + &  x_3 & + & 4x_4 & + & 3x_5 \\
%         \mbox{s.t. } & -x_1 &   &      & + &  x_3 & + & 3x_4 & - &  x_5
%                                                              & = & b_1 \\ %3
%                      &  x_1 & + &  x_2 &   &      &   &      & - &  x_5
%                                                   & = & b_2 & \qquad (P)\\ %1
%                      &      &   & 2x_2 & + &  x_3 & + & 2x_4 & + &  x_5
%                                                              & = & b_1+2b_2  \\
%                                                                        %5
%                      &      &   &&& x_i & \geq & 0, && i & = & 1,...,5.
%       \end{array}
%     \end{displaymath}
%     %In what follows,
%     Assume that
%     the problem $(P)$ has an optimal solution $\bar{x}$, where
%     $\bar{x}_1$, $\bar{x}_2$ and $\bar{x}_3$ are basic.
%     \begin{enumerate}
%       \item  What is the dual to this linear program?
%       \item  Use complementary slackness to find an optimal solution
%         $\bar{y}$ to the dual problem.  %(Hint:
%         %\begin{displaymath}
%         %  \left[ \begin{array}{rrr} -1 & 0 & 1 \\ 1 & 1 & 0 \\ 0 & 2 & 1
%         %            \end{array}  \right]^{-1}  =
%         %  \left[ \begin{array}{rrr} 1 & 2 & -1 \\ -1 & -1 & 1 \\ 2 & 2 & -1
%         %            \end{array}  \right]. )
%         %\end{displaymath}
%     \end{enumerate}
%   \item Consider again the problem $(P)$ in Question~\ref{qcs}.
%     \begin{enumerate}
%       \item  Looking purely at the optimal dual solution you found above,
%         can you conclude that
%         no optimal primal solution will have $x_4>0$? What about~$x_5$?
%       \item  The optimal dual solution $\bar{y}$ is not unique: there are
%         other optimal solutions of the form $\bar{y}+\lambda[-1,-2,1]^T$.
%         What is the largest possible value of~$\lambda$?
%         Do these other optimal solutions imply anything about the values of
%         $\bar{x}_1$, $\bar{x}_2$ and~$\bar{x}_3$?
%         Do they imply anything about the values of $x_4$ and $x_5$ in any
%         optimal solution to~$(P)$?
%     \end{enumerate}
%  \item  Recall that the polar cone of~$S$ is
%         $S^+=\{x:x^Ty \leq 0 \mbox{ for all } y \in S\}$.
%         A constrained cone is a set of points
%         $T$ satisfying $T=\{x:a^Tx \leq 0 \mbox{ for all } a \in A\}$
%         for some set $A$.
%         \begin{enumerate}
%           \item Let $S$ be a subset of $\real^n$. Show that $S\subseteq S^{++}$.
%           \item Let $S$ be a subset of $\real^n$. Show that $S^+=S^{+++}$.
%           \item Let $S$ be a constrained cone. Show that $S=S^{++}$.
%         \end{enumerate}
%  \item  Show that a convex cone has at most one extreme point,
%         namely the origin.
%  \item  Consider the subspace alternative theorem:  \\
%         For $A\in{\real}^{m{\times}n}$ and $c\in{\real}^n$, exactly one of the
%         following holds:
%         \begin{enumerate}
%           \item  $  \exists y \in {\real}^m  \mbox{ s.t. }  A^Ty=c   $
%           \item  $  \exists x \in {\real}^n  \mbox{ s.t. }  Ax=0, c^Tx \neq 0. $
%         \end{enumerate}
%         Develop a proof for this along the following lines:
%         \begin{quote}
%            First observe that both (a) and (b) can not hold simultaneously.
%            Thus, it suffices to show that when (a) fails, (b) must hold.
%            Next assume (a) fails and then construct an $x$ which satisfies (b).
%         \end{quote}
%         (Hint: You may assume that,
%             given a matrix $M \in {\real}^{m \times n}$ and a vector
%             $p \in {\real}^n$, there exist unique $p'$ and $p''$ such that
%             $p=p'+p''$, $Mp'=0$, $p''=M^Tq$ for some $q$, and ${p'}^Tp''=0$.)
             
  \item
%  question 2.22 from Bazaraa, Sherali, Shetty
Let $A$ be a $p \times n$ matrix and $B$ be a $q \times n$ matrix.
Show that exactly one of the following systems has a solution:
\begin{quote}
{\bf System 1:} $Ax < 0$, $Bx=0$ for some $x \in \re^n$.  \\
{\bf System 2:} $A^Tu+B^Tv=0$ for some $u \in \re^p$, $v \in \re^q$,
with $u \geq 0$ and $u \neq 0$.
\end{quote}
\end{enumerate}

\bigskip

\bigskip

\vfill

\begin{tabular}{@{\hspace{.5in}}l}
   John Mitchell  \\
   Amos Eaton 325  \\
   x6915.  \\
   mitchj@rpi.edu  \\
   Office hours:  
   Tuesday: 2pm -- 3.30pm.
\end{tabular}

\end{document}
