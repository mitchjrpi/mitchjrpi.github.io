\documentclass[12pt]{article}

\pagestyle{empty}

\oddsidemargin -.5in
\textwidth 7.5in
\textheight 10in
\topmargin -30pt
\headsep 0in
\headheight 0in

\newcommand{\re}{I \!\! R}

\begin{document}

\begin{center}
  \begin{large}
     MATP6640/DSES6770 Linear Programming, Homework 5.
  \end{large}
\end{center}

\begin{flushright}
   Due:  Friday, April 21, 2006.
\end{flushright}


%\vspace{\baselineskip}

%Note: In the following, $P_M$ denotes projection onto the nullspace
%of the matrix~$M$. If $M$ has full row rank, then $P_M$ is given by
%\begin{displaymath}
%P_M = I - M^T (MM^T)^{-1} M.
%\end{displaymath}
%Note that if $M$ is a $m \times n$ matrix then $P_M$ is an $n \times n$
%matrix.

\begin{enumerate}
  \item  Let $x$ be a nondegenerate basic feasible solution for the standard
    form linear programming problem $\min\{c^Tx:Ax=b, x\geq 0 \}$.
    Let $\bar{A} = AX$, where $X$ is a diagonal matrix with $X_{ii}=x_i$ for
    $i=1,...,n$. Simplify $P_{\bar{A}}$, the projection matrix onto the
    nullspace of $\bar{A}$.
    Hence show that $P_{\bar{A}} Xc = 0$.
  \item  \label{MAd1}
The Monteiro-Adler short step algorithm presented in class (Algorithm SPF
on page 86 of {\em Wright}) could be used as an infeasible interior point
method, starting with $x=s=e$, $y=0$, and $\mu=1$.
Give an example to show that in this case we may not be able
to take a step of length 1 and we may need to take a linesearch to
preserve nonnegativity.
(Note: We solve (1.20) to find the directions given in (1.25)
at each iteration. Hint: you only need look at {\bf small} examples.)
%  \item
%Show that we may not get $\Delta x^T \Delta s=0$ in the setting of 
%Question~\ref{MAd1}, even if~$r_c=0$.
  \item
Consider the primal-dual pair of linear programming problems:
\begin{displaymath}
\begin{array}{lrclllrcll}
\min & c^Tx &&&& \max & b^Ty \\
\mbox{subject to} & Ax & = & b & \quad (P) \qquad &
   \mbox{subject to} & A^Ty & \leq & c  & \quad (D)  \\
& x & \geq & 0
\end{array}
\end{displaymath}
where $x$ and $c$ are $n$-vectors, $b$ and $y$ are $m$-vectors,
and $A$ is an $m \times n$ matrix.
Let $\hat{b}=b-Ae$ and $\hat{c}=c-e$,
where $e$ denotes the vector where every component is equal to one.
Let $\hat{d}=c^Te+1$.
Now consider the linear programming problem
\begin{displaymath}
\begin{array}{lrcrcrcccl}
\min &&&&&&& (n+1) w \\
\mbox{subject to} &&& Ax & - & bt & + & \hat{b} w & = & 0  \\
& -A^Ty &&& + & ct & - & \hat{c} w & \geq & 0 \qquad (HLP)  \\
& b^Ty & - & c^T x &&& + & \hat{d} w & \geq & 0 \\
& -\hat{b}^Ty & + & \hat{c}^Tx & - & \hat{d} t &&& = & -(n+1)  \\
&&& x,t & \geq & 0, && y,w &&\mbox{free,}
\end{array}
\end{displaymath}
where $t$ and $w$ are scalars.
An interior point method can be used to find a {\em strictly complementary}
optimal solution to $(HLP)$.
In such a solution, if the primal variable is equal to zero then
the corresponding dual slack is strictly positive.
Use this result to show that if $t=0$ in a strictly complementary
optimal solution then either there exists a vector $x\geq 0$ with $Ax=0$ and
$c^Tx<0$ or there exists a vector $y$ with $A^Ty\leq 0$ and $b^Ty>0$.
\end{enumerate}


\vfill

\begin{tabular}{@{\hspace{.5in}}l}
   John Mitchell  \\
   Amos Eaton 325  \\
   x6915.  \\
   mitchj@rpi.edu  \\
   Office hours: Tuesday: 2 -- 3.30pm.
\end{tabular}

\end{document}
