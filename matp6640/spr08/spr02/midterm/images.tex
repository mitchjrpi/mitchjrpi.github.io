\batchmode
\documentclass[12pt]{article}
\makeatletter


\oddsidemargin -.5in
\textwidth 7in
\textheight 8.5in
\headsep 0in
\headheight 0in


\renewcommand{\baselinestretch}{1.2}


\usepackage[dvips]{color}
\pagecolor[gray]{.7}



\makeatletter

\makeatletter
\count@=\the\catcode`\_ \catcode`\_=8 
\newenvironment{tex2html_wrap}{}{} \catcode`\_=\count@
\makeatother
\let\mathon=$
\let\mathoff=$
\ifx\AtBeginDocument\undefined \newcommand{\AtBeginDocument}[1]{}\fi
\newbox\sizebox
\setlength{\hoffset}{0pt}\setlength{\voffset}{0pt}
\addtolength{\textheight}{\footskip}\setlength{\footskip}{0pt}
\addtolength{\textheight}{\topmargin}\setlength{\topmargin}{0pt}
\addtolength{\textheight}{\headheight}\setlength{\headheight}{0pt}
\addtolength{\textheight}{\headsep}\setlength{\headsep}{0pt}
\setlength{\textwidth}{349pt}
\newwrite\lthtmlwrite
\makeatletter
\let\realnormalsize=\normalsize
\global\topskip=2sp
\def\preveqno{}\let\real@float=\@float \let\realend@float=\end@float
\def\@float{\let\@savefreelist\@freelist\real@float}
\def\end@float{\realend@float\global\let\@freelist\@savefreelist}
\let\real@dbflt=\@dbflt \let\end@dblfloat=\end@float
\let\@largefloatcheck=\relax
\def\@dbflt{\let\@savefreelist\@freelist\real@dbflt}
\def\adjustnormalsize{\def\normalsize{\mathsurround=0pt \realnormalsize
 \parindent=0pt\abovedisplayskip=0pt\belowdisplayskip=0pt}\normalsize}%
\def\lthtmltypeout#1{{\let\protect\string\immediate\write\lthtmlwrite{#1}}}%
\newcommand\lthtmlhboxmathA{\adjustnormalsize\setbox\sizebox=\hbox\bgroup}%
\newcommand\lthtmlvboxmathA{\adjustnormalsize\setbox\sizebox=\vbox\bgroup%
 \let\ifinner=\iffalse }%
\newcommand\lthtmlboxmathZ{\@next\next\@currlist{}{\def\next{\voidb@x}}%
 \expandafter\box\next\egroup}%
\newcommand\lthtmlmathtype[1]{\def\lthtmlmathenv{#1}}%
\newcommand\lthtmllogmath{\lthtmltypeout{l2hSize %
:\lthtmlmathenv:\the\ht\sizebox::\the\dp\sizebox::\the\wd\sizebox.\preveqno}}%
\newcommand\lthtmlfigureA[1]{\let\@savefreelist\@freelist
       \lthtmlmathtype{#1}\lthtmlvboxmathA}%
\newcommand\lthtmlfigureZ{\lthtmlboxmathZ\lthtmllogmath\copy\sizebox
       \global\let\@freelist\@savefreelist}%
\newcommand\lthtmldisplayA[1]{\lthtmlmathtype{#1}\lthtmlvboxmathA}%
\newcommand\lthtmldisplayB[1]{\edef\preveqno{(\theequation)}%
  \lthtmldisplayA{#1}\let\@eqnnum\relax}%
\newcommand\lthtmldisplayZ{\lthtmlboxmathZ\lthtmllogmath\lthtmlsetmath}%
\newcommand\lthtmlinlinemathA[1]{\lthtmlmathtype{#1}\lthtmlhboxmathA  \vrule height1.5ex width0pt }%
\newcommand\lthtmlinlineA[1]{\lthtmlmathtype{#1}\lthtmlhboxmathA}%
\newcommand\lthtmlinlineZ{\egroup\expandafter\ifdim\dp\sizebox>0pt %
  \expandafter\centerinlinemath\fi\lthtmllogmath\lthtmlsetinline}
\newcommand\lthtmlinlinemathZ{\egroup\expandafter\ifdim\dp\sizebox>0pt %
  \expandafter\centerinlinemath\fi\lthtmllogmath\lthtmlsetmath}
\def\lthtmlsetinline{\hbox{\vrule width.1em\vtop{\vbox{%
  \kern.1em\copy\sizebox}\ifdim\dp\sizebox>0pt\kern.1em\else\kern.3pt\fi
  \ifdim\hsize>\wd\sizebox \hrule depth1pt\fi}}}
\def\lthtmlsetmath{\hbox{\vrule width.1em\vtop{\vbox{%
  \kern.1em\kern0.8 pt\hbox{\hglue.17em\copy\sizebox\hglue0.8 pt}}\kern.3pt%
  \ifdim\dp\sizebox>0pt\kern.1em\fi \kern0.8 pt%
  \ifdim\hsize>\wd\sizebox \hrule depth1pt\fi}}}
\def\centerinlinemath{%\dimen1=\ht\sizebox
  \dimen1=\ifdim\ht\sizebox<\dp\sizebox \dp\sizebox\else\ht\sizebox\fi
  \advance\dimen1by.5pt \vrule width0pt height\dimen1 depth\dimen1 
 \dp\sizebox=\dimen1\ht\sizebox=\dimen1\relax}

\def\lthtmlcheckvsize{\ifdim\ht\sizebox<\vsize\expandafter\vfill
  \else\expandafter\vss\fi}%
\makeatletter \tracingstats = 1 


\begin{document}
\pagestyle{empty}\thispagestyle{empty}%
\lthtmltypeout{latex2htmlLength hsize=\the\hsize}%
\lthtmltypeout{latex2htmlLength vsize=\the\vsize}%
\lthtmltypeout{latex2htmlLength hoffset=\the\hoffset}%
\lthtmltypeout{latex2htmlLength voffset=\the\voffset}%
\lthtmltypeout{latex2htmlLength topmargin=\the\topmargin}%
\lthtmltypeout{latex2htmlLength topskip=\the\topskip}%
\lthtmltypeout{latex2htmlLength headheight=\the\headheight}%
\lthtmltypeout{latex2htmlLength headsep=\the\headsep}%
\lthtmltypeout{latex2htmlLength parskip=\the\parskip}%
\lthtmltypeout{latex2htmlLength oddsidemargin=\the\oddsidemargin}%
\makeatletter
\if@twoside\lthtmltypeout{latex2htmlLength evensidemargin=\the\evensidemargin}%
\else\lthtmltypeout{latex2htmlLength evensidemargin=\the\oddsidemargin}\fi%
\makeatother

% !!! IMAGES START HERE !!!

{\newpage\clearpage
\lthtmldisplayA{displaymath17}%
\begin{displaymath}
\begin{array}{lrclr}
\min & c^Tx \\
\mbox{subject to } & Ax & = & b & (Pb) \\
& 0 \; \leq \;  x & \leq & u
\end{array}
\end{displaymath}%
\lthtmldisplayZ
\hfill\lthtmlcheckvsize\clearpage}

{\newpage\clearpage
\lthtmldisplayA{displaymath23}%
\begin{displaymath}
\begin{array}{lrclrlrcrclr}
\min & c^Tx &&&& \max & b^Ty \\
\mbox{subject to } & Ax & = & b & (P) \qquad & \mbox{subject to } & A^Ty & + & s & = & c & (D) \\
& x & \geq & 0 &&&&& s & \geq & 0
\end{array}
\end{displaymath}%
\lthtmldisplayZ
\hfill\lthtmlcheckvsize\clearpage}

{\newpage\clearpage
\lthtmlinlinemathA{tex2html_wrap_inline223}%
$\bar{B}$%
\lthtmlinlinemathZ
\hfill\lthtmlcheckvsize\clearpage}

{\newpage\clearpage
\lthtmldisplayA{displaymath33}%
\begin{displaymath}
          \bar{B}=B+(a-b)e_p^T.
        \end{displaymath}%
\lthtmldisplayZ
\hfill\lthtmlcheckvsize\clearpage}

{\newpage\clearpage
\lthtmldisplayA{displaymath36}%
\begin{displaymath}
          \bar{L}\bar{L}^T = LL^T - bb^T +aa^T,
        \end{displaymath}%
\lthtmldisplayZ
\hfill\lthtmlcheckvsize\clearpage}

{\newpage\clearpage
\lthtmlinlinemathA{tex2html_wrap_inline241}%
$\bar{L}$%
\lthtmlinlinemathZ
\hfill\lthtmlcheckvsize\clearpage}

{\newpage\clearpage
\lthtmlinlinemathA{tex2html_wrap_inline243}%
$\bar{B}=\bar{L}\bar{Q}$%
\lthtmlinlinemathZ
\hfill\lthtmlcheckvsize\clearpage}

{\newpage\clearpage
\lthtmlinlinemathA{tex2html_wrap_inline245}%
$\bar{Q}$%
\lthtmlinlinemathZ
\hfill\lthtmlcheckvsize\clearpage}

{\newpage\clearpage
\lthtmlinlinemathA{tex2html_wrap_inline251}%
${\mathit QR}$%
\lthtmlinlinemathZ
\hfill\lthtmlcheckvsize\clearpage}

{\newpage\clearpage
\lthtmldisplayA{displaymath51}%
\begin{displaymath}
\begin{array}{lrclr}
\min & c^Tx \\
\mbox{subject to } & Ax & = & b & (P) \\
& x & \geq & 0
\end{array}
\end{displaymath}%
\lthtmldisplayZ
\hfill\lthtmlcheckvsize\clearpage}

{\newpage\clearpage
\lthtmlinlinemathA{tex2html_wrap_inline283}%
$R_{\epsilon}$%
\lthtmlinlinemathZ
\hfill\lthtmlcheckvsize\clearpage}

{\newpage\clearpage
\lthtmldisplayA{displaymath58}%
\begin{displaymath}
\begin{array}{lrclr}
\max & e^Tx \\
\mbox{subject to } & Ax & = & b & (P) \\
& c^Tx & \leq & z^* + \epsilon \\
& x & \geq & 0
\end{array}
\end{displaymath}%
\lthtmldisplayZ
\hfill\lthtmlcheckvsize\clearpage}

{\newpage\clearpage
\lthtmlinlinemathA{tex2html_wrap_inline285}%
$\epsilon$%
\lthtmlinlinemathZ
\hfill\lthtmlcheckvsize\clearpage}

{\newpage\clearpage
\lthtmlinlinemathA{tex2html_wrap_inline295}%
$Q_{\epsilon}:=\{x \geq 0: Ax=b, c^Tx \leq z^* + \epsilon \}$%
\lthtmlinlinemathZ
\hfill\lthtmlcheckvsize\clearpage}

{\newpage\clearpage
\lthtmlinlinemathA{tex2html_wrap_inline297}%
$\epsilon' > \epsilon$%
\lthtmlinlinemathZ
\hfill\lthtmlcheckvsize\clearpage}

{\newpage\clearpage
\lthtmldisplayA{displaymath66}%
\begin{displaymath}
R_{\epsilon'} \leq \frac{\epsilon'}{\epsilon} R_{\epsilon}.
\end{displaymath}%
\lthtmldisplayZ
\hfill\lthtmlcheckvsize\clearpage}

{\newpage\clearpage
\lthtmldisplayA{displaymath72}%
\begin{displaymath}
\begin{array}{lrcrcrcrcrcrcl}
\min & & - & 2x_{13}& - & x_{23} & - & 2x_{24} &&& - & 10 x_{41} \\
\mbox{subject to} & -x_{12} & - & x_{13} &&&&&&& + & x_{41} & = & 0 \\
& x_{12} &&& - & x_{23} & - & x_{24} &&&&& = & 0 \\
&&& x_{13} & + & x_{23} &&& - & x_{34} &&& = & 0 \\
&&&&&&& x_{24} & + & x_{34} & - & x_{41} & = & 0 \\
& \multicolumn{11}{r}{0 \, \leq \, x_{12}, x_{13}, x_{23}, x_{24}, x_{34}} & \leq & 1 \\
&&&&&&&&& 0 & \leq & x_{41} & \leq & 10
\end{array}
\end{displaymath}%
\lthtmldisplayZ
\hfill\lthtmlcheckvsize\clearpage}

{\newpage\clearpage
\lthtmldisplayA{displaymath102}%
\begin{displaymath}
x_{12} = 1, \quad x_{34}=1, \quad x_{41}=1
\end{displaymath}%
\lthtmldisplayZ
\hfill\lthtmlcheckvsize\clearpage}

{\newpage\clearpage
\lthtmldisplayA{displaymath107}%
\begin{displaymath}
x_{13}=0, \quad x_{23}=1, \quad x_{24}=0.
\end{displaymath}%
\lthtmldisplayZ
\hfill\lthtmlcheckvsize\clearpage}


\end{document}
