\documentstyle[12pt]{article}

\newcommand{\re}{I\!\!R}

\oddsidemargin -.5in
\textwidth 7.5in
\textheight 9.8in
\topmargin -30pt
\headsep 0in
\headheight 0in

\pagestyle{empty}

\begin{document}

\begin{center}
{\bf\large\bf Subspaces, Affine sets, Convex sets, Cones} \\ John Mitchell
\end{center}

This document describes various subsets of $\re^n$.
%Throughout, we will take $A$ to be the $3 \times 4$ matrix
%\begin{displaymath}
%   A = \left[ \begin{array}{rrrr}
%     1 & 2 & 3 & 4 \\
%    -2 & 3 & -1 & 5 \\
%     3 & -1 & 4 & -1  \end{array}  \right]
%\end{displaymath}

\begin{itemize}
 \item
Let $v_1,\ldots,v_k$ be $k$ vectors in $\re^n$.
Let $\lambda_1,\ldots,\lambda_k$ be $k$ scalars.
The vector $v:=\sum_{i=1}^k\lambda_iv_i$ is a
{\bf linear combination} of $v_1,\ldots,v_k$.
 \item
Let $S$ be a subset of $\re^n$.
$S$ is a {\bf subspace} if it is closed under linear combinations.
Thus, for any $k>0$, for any vectors $v_1,\ldots,v_k \in S$,
and for any scalars $\lambda_1,\ldots,\lambda_k$, the linear
combination $v:=\sum_{i=1}^k\lambda_iv_i$ is also in~$S$.
Notice that the origin is in any nonempty subspace --- just take all
$\lambda_i=0$.
 \item
The row space, range, and null space of a matrix are all subspaces.
 \item
Let $v_1,\ldots,v_k$ be $k$ vectors in $\re^n$.
Let $\lambda_1,\ldots,\lambda_k$ be $k$ scalars
{\em satisfying $\sum_{i=1}^k\lambda_i=1$}.
(Note: some of the scalars may be negative.)
The vector $v:=\sum_{i=1}^k\lambda_iv_i$ is an
{\bf affine combination} of $v_1,\ldots,v_k$.
 \item
Let $S$ be a subset of $\re^n$.
$S$ is an {\bf affine space} if it is closed under affine combinations.
Thus, for any $k>0$, for any vectors $v_1,\ldots,v_k \in S$,
and for any scalars $\lambda_1,\ldots,\lambda_k$
satisfying $\sum_{i=1}^k\lambda_i=1$, the affine
combination $v:=\sum_{i=1}^k\lambda_iv_i$ is also in~$S$.
 \item
The set of solutions to the system of equations $Ax=b$ is an
affine space.
This is why we talk about affine spaces in this course!
 \item
An affine space is a translation of a subspace.
 \item
Any subspace is also an affine space.
 \item
Let $v_1,\ldots,v_k$ be $k$ vectors in $\re^n$.
Let $\lambda_1,\ldots,\lambda_k$ be $k$ {\em nonnegative} scalars
satisfying $\sum_{i=1}^k\lambda_i=1$.
The vector $v:=\sum_{i=1}^k\lambda_iv_i$ is a
{\bf convex combination} of $v_1,\ldots,v_k$.
 \item
Let $S$ be a subset of $\re^n$.
$S$ is an {\bf convex set} if it is closed under convex combinations.
Thus, for any $k>0$, for any vectors $v_1,\ldots,v_k \in S$,
and for any nonnegative scalars $\lambda_1,\ldots,\lambda_k$
satisfying $\sum_{i=1}^k\lambda_i=1$, the convex
combination $v:=\sum_{i=1}^k\lambda_iv_i$ is also in~$S$.
 \item
Polyhedra are convex sets.
 \item
A polytope is defined to be a bounded polyhedron.
Note that every point in a polytope is a convex combination of the
extreme points.
 \item
Any subspace is a convex set.
Any affine space is a convex set.
 \item
Let $S$ be a subset of $\re^n$. $S$ is a {\bf cone}
if it is closed under nonnegative scalar multiplication.
Thus, for any vector $v\in S$ and for any nonnegative scalar~$\lambda$,
the vector $\lambda v$ is also in $S$.
 \item
Let $S$ be a subset of $\re^n$. $S$ is a {\bf convex cone}
if it is a cone and it is convex.
It can be shown that this is equivalent to saying that $S$
is closed under nonnegative linear combinations.
Thus, for any $k>0$, for any vectors $v_1,\ldots,v_k \in S$,
and for any nonnegative scalars $\lambda_1,\ldots,\lambda_k$,
the linear combination $v:=\sum_{i=1}^k\lambda_iv_i$ is also in~$S$.
 \item
The origin is in any nonempty cone --- just take $\lambda=0$.
 \item
Any subspace is a convex cone.
\end{itemize}


\end{document}
