\documentstyle[12pt]{article}

\renewcommand{\baselinestretch}{1.4}

% My definitions of page layout
\oddsidemargin .3in
\textwidth 6in
\textheight 8.8in
\topmargin -30pt

\pagestyle{myheadings}

%\markboth{}{}

\markboth{Dantzig-Wolfe example}{Dantzig-Wolfe example}

\begin{document}

\begin{center}
   MATP6640/DSES6780 Linear Programming
\end{center}

\begin{center}
  {\large\bf An Example of Dantzig-Wolfe Decomposition}
\end{center}

Consider the linear programming problem:
\begin{displaymath}
  \begin{array}{lccl}
    \min         & c^Tx    & := & x_1 - x_2 - 2x_3 \\
    \mbox{s.t. } & Ax      & := & x_1 + x_2 + x_3 = 3 =: b \\
                 & x \in X & := & \left\{x : \begin{array}{l}
                                          0 \leq x_1 \leq 2 \\
                                          0 \leq x_2 \\
                                          0 \leq x_3 \leq 2
                                        \end{array} \right\}.
  \end{array}
\end{displaymath}
Thus, $c^T=(1,-1,-2)$, $A=(1,1,1)$ and $b=3$.
The {\em Master Problem (MP)} is:
\begin{displaymath}
  \begin{array}{lrcrcl}
    \min         & \sum_{j\in J} (c^Tx^j) \lambda_j & + &
                      \sum_{k\in K} (c^Td^k) \mu_k \\
    \mbox{s.t. } & \sum_{j\in J} (Ax^j) \lambda_j & + & 
                      \sum_{k\in K} (Ad^k) \mu_k & = & b \\
                 & \sum_{j\in J} \lambda_j &&& = & 1 \\
                 & \lambda_j \geq 0, && \mu_k \geq 0,
  \end{array}
\end{displaymath}
where $\{x^j:j \in J \}$ is the set of extreme points of $X$, and
$\{d^k:k \in K \}$ is the set of extreme rays of $X$.


\begin{picture}(400,300)
\put(200,100){\vector(1,0){200}}
\put(200,100){\vector(0,1){150}}
\put(200,100){\vector(-2,-1){150}}
\put(300,100){\line(0,1){100}}
\put(200,200){\line(1,0){100}}
\put(300,100){\line(-2,-1){150}}
\put(200,200){\line(-2,-1){150}}
\put(300,200){\line(-2,-1){150}}

\put(35,25){$x_2$}
\put(390,85){$x_1$}
\put(185,240){$x_3$}

\put(201,85){0}
\put(301,85){2}
\put(185,201){2}
\end{picture}

\newpage

\begin{itemize}
  \item {\bf Initialization}: \\
     Start with the two extreme points $x^1:=(2,0,0)^T$ and $x^2:=(2,0,2)^T$ of
     $X$. Then when using the revised simplex method, our basic variables
     are $\lambda_1$ and $\lambda_2$.
     So, in terms of the basic variables, we get the
     {\em Revised Master Problem (RMP)}:
     \begin{displaymath}
       \begin{array}{lrcrcl}
         \min        & 2\lambda_1 & - & 2\lambda_2 \\
         \mbox{s.t. }& 2\lambda_1 & + & 4\lambda_2 & = & 3 \\
                     &  \lambda_1 & + &  \lambda_2 & = & 1 \\
                     &  \lambda_j & \geq & 0.
       \end{array}
     \end{displaymath}
     Here, the coefficient of $\lambda_1$ in the objective function is
     $c^Tx^1$, the coefficient of 4 for $\lambda_2$ in the first constraint
     comes from $Ax^2$, etc.
     The second constraint is the convexity constraint
     $\sum_{j\in J} \lambda_j = 1$.
     We will denote costs in {\em (MP)}
     by $\hat{c}_j$, columns by $\hat{a}_j$, etc.

     Initial basis:
     \begin{displaymath}
       \begin{array}{c}\lambda_1: \\ \lambda_2: \end{array} \;\;\;
       \hat{B} = \left[ \begin{array}{rr}2&4\\1&1\end{array} \right],
       \hat{B}^{-1} = \left[ \begin{array}{rr}-0.5&2\\0.5&-1\end{array} \right].
     \end{displaymath}
     Thus, the initial basic feasible solution is:
     \begin{displaymath}
       \left( \begin{array}{c} \bar{\lambda}_1 \\ \bar{\lambda}_2 \end{array}
         \right)  =  \hat{B}^{-1} \hat{b}  =
            \left( \begin{array}{c} 0.5 \\ 0.5 \end{array} \right);
     \end{displaymath}
     the corresponding dual solution is:
     \begin{displaymath}
       (\pi, \sigma) = \hat{c}_B \hat{B}^{-1} = (-2, 6);
     \end{displaymath}
     the corresponding solution to the initial problem is
     \begin{displaymath}
       \bar{x} = \bar{\lambda}_1 x^1 + \bar{\lambda}_2 x^2 = (2,0,1)^T.
     \end{displaymath}
     Thus the subproblem is
     \begin{displaymath}
      \begin{array}{lc}
       \min & z := (c^T - {\pi}^T A) x = ((1,-1,-2)+2(1,1,1))x = 3x_1 + x_2 \\
         \mbox{s.t. } & x \in X.
      \end{array}
     \end{displaymath}
     By inspection, the
     optimal solution is $x=(0,0,0)^T$, giving $z^*=0<6=\sigma$.
     Therefore, our current primal solution is not optimal, and we need
     to introduce a column for $x_3:=(0,0,0)^T$ into the
     {\em Revised Master Problem}.
\newpage
  \item {\bf First iteration}: \\
     Then $\lambda_3$ enters the basis.
     Objective function value is $\hat{c}_3=c^Tx^3=0$;
     column of constraint matrix is
       $\hat{a}^3=\left(\begin{array}{r}Ax^3\\1\end{array}\right)=
          \left(\begin{array}{r}0\\1\end{array}\right)$
     In order to determine which variable leaves the basis,
     we need to calculate 
       $\hat{B}^{-1} \hat{a}^3 = \left(\begin{array}{r}2\\-1\end{array}\right)$.
     We then compare this column of ``$B^{-1}N$'' with the current bfs
     $\hat{x}_B=\left(\begin{array}{r}0.5\\0.5\end{array}\right)$
     using the minimum ratio test.
     Thus, $\lambda_1$ leaves the basis.
     The new basis matrix is:
     \begin{displaymath}
       \begin{array}{c}\lambda_3: \\ \lambda_2: \end{array} \;\;\;
       \hat{B} = \left(\begin{array}{rr}0&4\\1&1\end{array}\right);
        \hat{B}^{-1} = \left(\begin{array}{rr}-0.25&1\\0.25&0\end{array}\right)
     \end{displaymath}
     Thus, the new primal solution to {\em (MP)} is
     \begin{displaymath}
      (\bar{\lambda}_3,\bar{\lambda}_2)^T = 
       \hat{B}^{-1} \hat{b} = \hat{B}^{-1} (3,1)^T = (0.25,0.75)^T
     \end{displaymath}
     and the new dual solution is
     \begin{displaymath}
       (\pi,\sigma) = \hat{c}^T_B \hat{B}^{-1} = (0,-2) \hat{B}^{-1} = (-0.5,0).
     \end{displaymath}
     Hence, the new solution for the original LP is:
     \begin{displaymath}
       \bar{x} = \bar{\lambda}_3 x^3 + \bar{\lambda}_2 x^2 = (1.5,0,1.5)^T.
     \end{displaymath}
     The subproblem is
     \begin{displaymath}
       \begin{array}{lc}
       \min & z = (c^T - \pi A) x = ((1,-1,-2)+0.5(1,1,1))x=(1.5,-0.5,-1.5)x \\
          \mbox{s.t. }  & x \in X.
       \end{array}
     \end{displaymath}
     This subproblem has an unbounded optimal solution, and the extreme ray is
     $d=(0,1,0)^T$.

     Set $d^4:=(0,1,0)^T$; then $\hat{c}_4=c^Td^4=-1$ and
      $\hat{a}_4=\left(\begin{array}{c}Ad^4\\0\end{array}\right)=
         \left(\begin{array}{c}1\\0\end{array}\right)$.
     We introduce $\mu_4$ into the basis.

\newpage
  \item {\bf Second iteration}: \\
     We need to determine which variable leaves the basis.
     Therefore, we calculate $\hat{B}^{-1}\hat{a}_4=(-0.25,0.25)^T$.
     Using the minimum ratio test with $\hat{x}_B=(0.25,0.75)^T$ shows that
     $\lambda_2$ leaves the basis.
     The new basis matrix is:
     \begin{displaymath}
       \begin{array}{c}\lambda_3: \\ \mu_4: \end{array} \;\;\;
       \hat{B} = \left(\begin{array}{rr}0&1\\1&0\end{array}\right);
        \hat{B}^{-1} = \left(\begin{array}{rr}0&1\\1&0\end{array}\right)
     \end{displaymath}
     Thus, the new primal solution to {\em (MP)} is
     \begin{displaymath}
      (\bar{\lambda}_3,\bar{\mu}_4)^T =
       \hat{B}^{-1} \hat{b} = \hat{B}^{-1} (3,1)^T = (1,3)^T
     \end{displaymath}
     and the new dual solution is
     \begin{displaymath}
       (\pi,\sigma) = \hat{c}^T_B \hat{B}^{-1} = (0,-1) \hat{B}^{-1} = (-1,0).
     \end{displaymath}
     Hence, the new solution for the original LP is:
     \begin{displaymath}
       \bar{x} = \bar{\lambda}_3 x^3 + \bar{\mu}_4 d^4 = (0,3,0)^T.
     \end{displaymath}
     The subproblem is
     \begin{displaymath}
       \begin{array}{lc}
       \min & z = (c^T - \pi A) x = ((1,-1,-2)+1(1,1,1))x=2x_1-x_3 \\
          \mbox{s.t. }  & x \in X.
       \end{array}
     \end{displaymath}
     An optimal solution is $x=(0,0,2)^T$, giving $z^*=-2<\sigma$.
     So we do not have the optimal solution to the {\em master problem (MP)}.
     Hence, we set $x^5:=(0,0,2)^T$, and we introduce a column for this extreme
     point into the {\em Revised Master Problem}.

\newpage
  \item {\bf Third iteration}: \\
     Then $\lambda_5$ enters the basis.
     Objective function value is $\hat{c}_5=c^Tx^5=-4$;
     column of constraint matrix is
       $\hat{a}^5=\left(\begin{array}{r}Ax^5\\1\end{array}\right)=
          \left(\begin{array}{r}2\\1\end{array}\right)$.
     In order to determine which variable leaves the basis,
     we need to calculate
       $\hat{B}^{-1} \hat{a}^5 = \left(\begin{array}{r}1\\2\end{array}\right)$.
     Using the minimum ratio test with $\hat{x}_B=(1,3)^T$ shows that
     $\lambda_3$ leaves the basis.
     The new basis matrix is:
     \begin{displaymath}
       \begin{array}{c}\lambda_5: \\ \mu_4: \end{array} \;\;\;
       \hat{B} = \left(\begin{array}{rr}2&1\\1&0\end{array}\right);
        \hat{B}^{-1} = \left(\begin{array}{rr}0&1\\1&-2\end{array}\right).
     \end{displaymath}
     Thus, the new primal solution to {\em (MP)} is
     \begin{displaymath}
      (\bar{\lambda}_5,\bar{\mu}_4)^T =
       \hat{B}^{-1} \hat{b} = \hat{B}^{-1} (3,1)^T = (1,1)^T
     \end{displaymath}
     and the new dual solution is
     \begin{displaymath}
       (\pi,\sigma) = \hat{c}^T_B \hat{B}^{-1} = (-4,-1) \hat{B}^{-1} = (-1,-2).
     \end{displaymath}
     Hence, the new solution for the original LP is:
     \begin{displaymath}
       \bar{x} = \bar{\lambda}_5 x^5 + \bar{\mu}_4 d^4 = (0,1,2)^T.
     \end{displaymath}
     The subproblem is
     \begin{displaymath}
       \begin{array}{lc}
       \min & z = (c^T - \pi A) x = ((1,-1,-2)+1(1,1,1))x=2x_1-x_3 \\
          \mbox{s.t. }  & x \in X.
       \end{array}
     \end{displaymath}
     An optimal solution is $x=(0,0,2)^T$, giving $z^*=-2=\sigma$.
     Hence, our optimality criterion is satisfied,
     and $\bar{x}=(0,1,2)^T$ is an optimal solution to our original LP.
\end{itemize}




\end{document}
