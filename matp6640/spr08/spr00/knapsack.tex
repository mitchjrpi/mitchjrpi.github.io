\documentstyle[12pt]{article}

\begin{document}

\begin{center}
  {\bf\large\bf Solving a Knapsack Problem}
\end{center}

Consider the knapsack problem
\begin{displaymath}
  \begin{array}{lrcrcrcl}
    \min & -3x_1 & - & 4x_2 & - & 6x_3 \\
   \mbox{subject to} & x_1 & + & 3x_2 & + & 5x_3 & \leq & 7 \\
         &&& x_i & = & 0,1 && i=1,...,3
  \end{array}
\end{displaymath}
This is an {\bf integer} programming problem.
We solve it by solving a sequence of {\bf linear programming relaxations} of
the problem.
The initial relaxation is
\begin{displaymath}
  \begin{array}{lrcrcrcl}
    \min & -3x_1 & - & 4x_2 & - & 6x_3 \\
   \mbox{subject to} & x_1 & + & 3x_2 & + & 5x_3 & \leq & 7 \\
         & 0 & \leq & x_i & \leq & 1 && i=1,...,3
  \end{array}
\end{displaymath}
The simplex tableau for this problem is
       \begin{displaymath}
         M =
         \begin{array}{|r|rrrrrrr|}
            \multicolumn{1}{c}{\quad} & x_1 & x_2 & x_3 & s_4 & s_5 & s_6 &
                \multicolumn{1}{c}{s_7} \\ \hline
             0 & -3 & -4 & -6 & 0 & 0 & 0 & 0 \\ \hline
             7 & 1 & 3 & 5 & 1 & 0 & 0 & 0 \\
             1 & 1 & 0 & 0 & 0 & 1 & 0 & 0 \\
             1 & 0 & 1 & 0 & 0 & 0 & 1 & 0 \\
             1 & 0 & 0 & 1 & 0 & 0 & 0 & 1 \\ \hline
         \end{array}
       \end{displaymath}
The optimal tableau for this relaxation is
       \begin{displaymath}
         M^1 =
         \begin{array}{|r|rrrrrrr|}
            \multicolumn{1}{c}{\quad} & x_1 & x_2 & x_3 & s_4 & s_5 & s_6 &
                \multicolumn{1}{c}{s_7} \\ \hline
             10.6 & 0 & 0 & 0 & 1.2 & 1.8 & 0.4 & 0 \\ \hline
             0.6 & 0 & 0 & 1 & 0.2 & -0.2 & -0.6 & 0 \\
             1 & 1 & 0 & 0 & 0 & 1 & 0 & 0 \\
             1 & 0 & 1 & 0 & 0 & 0 & 1 & 0 \\
             0.4 & 0 & 0 & 0 & -0.2 & 0.2 & 0.6 & 1 \\ \hline
         \end{array}
       \end{displaymath}
Thus, the optimal solution to the initial relaxation is
\begin{displaymath}
  x_1=1, \quad x_2=1, \quad x_3=0.6
\end{displaymath}
This does obviously not solve the original relaxation,
because it has $x_3=0.6$.
Notice that no feasible solution to the integer programming problem
can have both $x_2=1$ and $x_3=1$.
Thus, every solution satisfies
\begin{displaymath}
  x_2 + x_3 \leq 1
\end{displaymath}
We can add this constraint to $M^1$, giving the tableau
       \begin{displaymath}
         M^2 =
         \begin{array}{|r|rrrrrrrr|}
            \multicolumn{1}{c}{\quad} & x_1 & x_2 & x_3 & s_4 & s_5 & s_6 &
                s_7 & \multicolumn{1}{c}{s_8} \\ \hline
             10.6 & 0 & 0 & 0 & 1.2 & 1.8 & 0.4 & 0 & 0 \\ \hline
             0.6 & 0 & 0 & 1 & 0.2 & -0.2 & -0.6 & 0 & 0 \\
             1 & 1 & 0 & 0 & 0 & 1 & 0 & 0 & 0 \\
             1 & 0 & 1 & 0 & 0 & 0 & 1 & 0 & 0 \\
             0.4 & 0 & 0 & 0 & -0.2 & 0.2 & 0.6 & 1 & 0 \\
             1 & 0 & 1 & 1 & 0 & 0 & 0 & 0 & 1 \\ \hline
         \end{array}
       \end{displaymath}
Pivoting to get an identity, we obtain
       \begin{displaymath}
         M^3 =
         \begin{array}{|r|rrrrrrrr|}
            \multicolumn{1}{c}{\quad} & x_1 & x_2 & x_3 & s_4 & s_5 & s_6 &
                s_7 & \multicolumn{1}{c}{s_8} \\ \hline
             10.6 & 0 & 0 & 0 & 1.2 & 1.8 & 0.4 & 0 & 0 \\ \hline
             0.6 & 0 & 0 & 1 & 0.2 & -0.2 & -0.6 & 0 & 0 \\
             1 & 1 & 0 & 0 & 0 & 1 & 0 & 0 & 0 \\
             1 & 0 & 1 & 0 & 0 & 0 & 1 & 0 & 0 \\
             0.4 & 0 & 0 & 0 & -0.2 & 0.2 & 0.6 & 1 & 0 \\
             -0.6 & 0 & 0 & 0 & -0.2 & 0.2 & -0.4 & 0 & 1 \\ \hline
         \end{array}
       \end{displaymath}
Solving this tableau using the {\bf dual simplex} method gives
       \begin{displaymath}
         M^3 =
         \begin{array}{|r|rrrrrrrr|}
            \multicolumn{1}{c}{\quad} & x_1 & x_2 & x_3 & s_4 & s_5 & s_6 &
                s_7 & \multicolumn{1}{c}{s_8} \\ \hline
             9 & 0 & 2 & 0 & 0 & 3 & 0 & 0 & 6 \\ \hline
             1 & 0 & 0 & 1 & 0 & 0 & 0 & 0 & 1 \\
             1 & 1 & 0 & 0 & 0 & 1 & 0 & 0 & 0 \\
             1 & 0 & -2 & 0 & 1 & -1 & 0 & 0 & -5 \\
             0 & 0 & -1 & 0 & 0 & 0 & 0 & 1 & -1 \\
             1 & 0 & 1 & 0 & 0 & 0 & 1 & 0 & 0 \\ \hline
         \end{array}
       \end{displaymath}
So, the solution to the modified relaxation is
\begin{displaymath}
  x_1=1, \quad x_2=0, \quad x_3=1
\end{displaymath}
This is integral, so it solves the original integer program.

\end{document}
