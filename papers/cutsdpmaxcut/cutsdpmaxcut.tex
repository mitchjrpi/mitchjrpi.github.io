\documentclass[12pt]{article}

\usepackage{color}
\usepackage{graphicx}

\renewcommand{\baselinestretch}{1.2}

\newcommand{\real}{I\!\!R}
\newcommand{\til}{\char '176}


\newtheorem{theorem}{Theorem}
\newtheorem{lemma}{Lemma}
\newtheorem{proposition}{Proposition}
\newtheorem{corollary}{Corollary}
\newtheorem{definition}{Definition}

% Definition of the proof-environment:
\newenvironment{proof}{{\raggedright\bf
Proof:}\quad}{\hspace*{\fill}$\fbox{ }$\\}
\newenvironment{altproof}{{\raggedright\bf Alternative
Proof:}\quad}{\hspace*{\fill}$\fbox{ }$}


\newtheorem{algorithm1}{Algorithm}
\newenvironment{algorithm}[1]{\begin{algorithm1}[#1]~\\
    \vspace{-0.5cm}}{\end{algorithm1}}


% My definitions of page layout
\oddsidemargin .3in
\textwidth 6in
\textheight 8.8in
\topmargin -30pt

\begin{document}

\bibliographystyle{plain}
%\begin{center}
%{\bigskip\Huge\bf McMaster University}\\[1cm]
%{\LARGE\bf Advanced Optimization Laboratory}\\[3.5cm]
%\begin{center}
%\centering
%% \includegraphics[width=2.5in]{advolab1}\\[2cm]
%
%{\LARGE\bf Title:}\\[0.1cm]
%
%{\LARGE\ A semidefinite programming based polyhedral cut and price
%approach for the maxcut problem} \vspace{1cm}
%
%{\Large\bf Authors:}\\[0.5cm]
%{\Large\ Kartik Krishnan, John Mitchell}\\[1cm]
%\end{center}
%\vspace{2cm} {\Large\bf AdvOL-Report No. 2004/3}\\[0.5cm]
%{\large\ May 2004, revised February 2005, Hamilton, Ontario,
%Canada}
%\end{center}

\begin{titlepage}
\begin{center}
{\bf\Large\bf A semidefinite programming based polyhedral cut and
price approach for the maxcut problem\footnote{Research supported
in part by NSF grant numbers CCR--9901822 and DMS--0317323}
\footnote{Work done as part of the first authors Ph.D.
dissertation at RPI}}
\end{center}

%\vspace{\baselineskip}

\begin{center}
{\bf\large\bf Kartik Krishnan}


  Harrelson Hall 235  \\
     Department of Mathematics  \\
     Campus Box 8205  \\
     North Carolina State University  \\
     Raleigh, NC 27695-8205  \\
     kksivara@ncsu.edu  \\
     http://www4.ncsu.edu/\til kksivara
     
\vspace{\baselineskip}

{\bf\large\bf John E. Mitchell}


    Mathematical Sciences  \\
    Rensselaer Polytechnic Institute  \\
    Troy,  NY 12180  \\
    mitchj@rpi.edu  \\
    http://www.rpi.edu/\til mitchj

\end{center}

\begin{center}
    Draft of \today
\end{center}

\begin{abstract}
We investigate solution of the maximum cut problem using a
polyhedral cut and price approach. The dual of the well-known SDP
relaxation of maxcut is formulated as a semi-infinite linear
programming problem, which is solved within an interior point
cutting plane algorithm in a dual setting; this constitutes the
pricing (column generation) phase of the algorithm. Cutting planes
based on the polyhedral theory of the maxcut problem are then
added to the primal problem in order to improve the SDP
relaxation; this is the cutting phase of the algorithm. We provide
computational results, and compare these results with a standard
SDP cutting plane scheme.
\begin{flushleft}
{\bf Keywords:} Semidefinite programming, column generation,
cutting plane methods, combinatorial optimization
\end{flushleft}
\end{abstract}

\end{titlepage}


\section{Introduction}

Let $G = (V,E)$ denote an edge weighted undirected graph without
loops or multiple edges. Let $V = \{1,\ldots,n\}$, $E \subset \{
\{i,j\} : 1 \le i < j \le n\}$, and $w \in \real^{|E|}$, with
$\{i,j\}$ the edge with endpoints $i$ and $j$, and weight
$w_{ij}$. We assume that $m = |E|$. For $S \subseteq V$, the set
of edges $\{i,j\} \in E$ with one endpoint in $S$ and the other in
$V \backslash S$ form the {\em cut} denoted by $\delta(S)$. We
define the weight of the cut as $w(\delta(S)) = \displaystyle
\sum_{\{i,j\} \in \delta(S)} w_{ij}$. The {\em maximum cut
problem}, denoted as (MC), is the problem of finding a cut for
which the total weight is maximum. (MC) can be formulated as
\begin{eqnarray}
\begin{array}{cc}
\max \{w(\delta(S)) | S \subseteq V \} & \qquad(MC)
\end{array}
\end{eqnarray}
The problem finds numerous applications in, for example, the layout of
electronic circuitry (Chang \& Du \cite{chang_du} and Pinter
\cite{pinter}), and state problems in statistical physics (Barahona et
al. \cite{barahona_et_al}). It is also a canonical problem from
which one can develop algorithms for other combinatorial
optimization problems such as min-bisection, graph partitioning
etc.

The maximum cut problem is one of the original NP complete
problems (Karp \cite{karp1}). However, there are several classes
of graphs for which the maximum cut problem can be solved in
polynomial time; for instance planar graphs, graphs with no $K_5$
minor etc. A good survey on the maxcut problem appears in Poljak
and Tuza \cite{poljak3}.

We can model the maxcut problem as an integer program as follows:
Consider a binary variable associated with each edge of the graph.
The value of the variable indicates whether the two endpoints of
the edge are on the same side of the cut or on opposite sides.
Taken together, these binary variables define a binary vector,
which represents a  feasible solution to the maxcut problem. The
convex hull of all these candidate solutions forms a polytope
known as the maxcut polytope, whose complete description is
unknown. The maxcut problem amounts to minimizing a linear
objective function over this polytope. One can also regard these
variables as the entries of a matrix which has a row and a column
for each vertex, and this matrix should be rank one (and therefore
positive semidefinite) for the values of the variables to be
consistent. This gives rise to a non-polyhedral (SDP) formulation
of the maxcut problem where one is now optimizing over this matrix
variable. These non-polyhedral formulations can be solved
efficiently and in polynomial time using interior point methods.
In an LP (SDP) cutting plane approach, one begins with an initial
polyhedral (non-polyhedral) description that contains the maxcut
polytope, and these descriptions are progressively refined using
cutting planes. The procedure is repeated until we have an optimal
solution to the maxcut problem.

Our aim in this paper is to solve the maxcut problem to optimality
using a polyhedral cut and price approach and heuristics.
Although, our approach is entirely polyhedral, it closely mimics
an SDP cutting plane approach for the maxcut problem.

The paper is organized in a manner that we have all the details at
hand to describe our cutting plane approach, which appears in
\S\ref{maxcut_as_lp}. An outline of the paper is as follows:
\S\ref{lp_formulations_of_maxcut} deals with a linear programming
(LP) formulation  while \S\ref{sdp_formulations_of_maxcut} deals
with a semidefinite programming (SDP) formulation of the maxcut
problem. In \S\ref{maxcut_as_lp} we motivate and develop our
polyhedral cut and price approach for the maxcut problem; this is
based on an SDP formulation of the maxcut problem presented in
\S\ref{sdp_formulations_of_maxcut}. The details of the algorithm
are given in \S\ref{sec_details}. We present some computational
results in \S\ref{maxcut_computational_results} and our
conclusions in \S\ref{maxcut_conclusions}.

\subsection{Notation}
A quick word on notation: we represent vectors and matrices, by
lower and upper case letters respectively. For instance $d_{ij}$
denotes the $j$th entry of the vector $d_i$ in a collection,
whereas $X_{ij}$ denotes the entry in the $i$th row and $j$th
column of matrix $X$. The requirement that a matrix $X$ be
symmetric and positive semidefinite (psd) is expressed $X \succeq 0$. We also
use MATLAB-like notation frequently in this paper. For instance
$y(n+1:m)$ will denote components $n+1$ to $m$ of vector $y$,
while $A(n+1:m,1:p)$ will denote the sub-matrix obtained by
considering rows $n+1$ to $m$, and the first $p$ columns of matrix
$A$. Finally ${d_j}.^ 2$ refers to a {\em row} vector obtained by
squaring all the components of $d_j$.

\section{Linear programming formulations of the maxcut problem}

\label{lp_formulations_of_maxcut} Consider a variable $x_{ij}$ for
each edge $\{i,j\}$ in the graph. Let $x_{ij}$ assume the value
$1$ if edge $\{i,j\}$ is in the cut, and $0$ otherwise. The maxcut
problem can then be modelled as the following integer programming
problem.
\begin{eqnarray}
\label{maxcut_as_ip}
\begin{array}{ccc}
\max & \displaystyle \displaystyle \sum_{i=1}^n \sum_{i<j, \{i,j\} \in E} w_{ij} x_{ij} \\
\mbox{subject to} & x & \mbox{is the incidence vector of a cut},
\end{array}
\end{eqnarray}
where $n$ is the number of vertices in the graph.

Let $\mbox{CUT}(G)$ denote the convex hull of the incidence
vectors of cuts. Since maximizing a linear function over a set of
points is equivalent to maximizing it over the convex hull of this
set of points, we can also write (\ref{maxcut_as_ip}) as the
following linear program
\begin{eqnarray}
\label{maxcut_as_01lp}
\begin{array}{cc}
\max & c^Tx \\
\mbox{s.t.} & x \in \mbox{CUT}(G).
\end{array}
\end{eqnarray}
Here $c = \{w_{ij} \}, \,\ x = \{x_{ij}\} \in \real^m$, where $m$
is the number of edges in the graph. We usually index $c$ and $x$
with the subscript $e$ to relate them to the edges in the graph.
However from time to time we drop the subscript $e$ as
appropriate. If we had an oracle capable of solving the separation
problem with respect to this polytope in polynomial time, we could
use this oracle in conjunction with the ellipsoid algorithm to
solve the maxcut problem in polynomial time (Gr\"otschel et al
\cite{GLSbook}). Unfortunately we do not have polynomial time
separation oracles for some of the inequalities that describe the
maxcut polytope, owing to the {\em NP complete} nature of the
problem.

It is instructive to study facets of the maxcut polytope,
since these provide good separation
inequalities in a cutting plane approach. Numerous facets of the maxcut polytope have been studied,
but the ones widely investigated in connection with a cutting plane approach are the following inequalities
\begin{eqnarray}
\label{facets_of_mcpolytope}
\begin{array}{cccc}
x_e & \ge & 0, & \forall e \in E, \\
x_e & \le & 1, & \forall e \in E, \\
x(F) - x(C \backslash F) & \le & |F| - 1 & \forall \hspace*{0.25cm} \mbox{circuits} \hspace*{0.25cm} C \subseteq E \\
& & & \mbox{and all} \hspace*{0.25cm} F \subseteq C
\hspace*{0.25cm} \mbox{with} \hspace*{0.25cm} |F| \hspace*{0.25cm}
\mbox{odd},
\end{array}
\end{eqnarray}
where $x(S) = \displaystyle \sum_{\{i,j\} \in S} x_{ij}$ for any $
S \subseteq E$. It is easy to see that these are valid
inequalities for the maxcut polytope. The first and second set of
inequalities define facets of $\mbox{CUT}(G)$, if $e$ is not
contained in a triangle. The third set known as the {\em odd-cycle
inequalities} defines facets for each $F \subseteq C$, with $|F|$
odd, if $C$ has no chord. We will call a graph {\em chordal} if
any cycle $C$ of length $\ge$ 4 has a chord. The latter
inequalities can be derived from the observation that every cycle
and cut intersect in an even number of edges. Moreover, the
integral vectors satisfying (\ref{facets_of_mcpolytope}) are
vertices of $\mbox{CUT}(G)$. So we indeed have an integer
programming formulation of the maximum cut problem, namely
\begin{eqnarray}
\label{maxcut_as_01ip}
\begin{array}{lrccc}
\max & c^Tx \\
\mbox{s.t.} & x(F) - x(C \backslash F) & \le & |F| - 1 & \forall \hspace*{0.25cm} \mbox{circuits} \hspace*{0.25cm} C \subseteq E \\
& & & & \mbox{and all} \hspace*{0.25cm} F \subseteq C \hspace*{0.25cm} \mbox{with} \hspace*{0.25cm} |F| \hspace*{0.25cm} \mbox{odd} \\
& x & \in & \{0,1\}^m.
\end{array}
\end{eqnarray}
Incidentally, the inequalities in (\ref{facets_of_mcpolytope})
define a polytope known as the {\em odd cycle polytope}. Barahona
and Mahjoub \cite{barahona4} show that we can drop the integrality
restriction in (\ref{maxcut_as_01ip}), and solve the maxcut
problem simply as an LP, if $G$ does not have any subgraph
contractible to $K_5$. This includes planar graphs, so we can
indeed solve these maxcut instances in polynomial time.

Although there is an exponential number of linear constraints in
(\ref{maxcut_as_01ip}), Barahona and Mahjoub \cite{barahona4}
describe a polynomial time separation oracle for the inequalities
(\ref{facets_of_mcpolytope}). Thus it is possible to optimize a
linear function over the odd cycle polytope in polynomial time.
This exact algorithm, together with separation heuristics, has
been used in cutting plane algorithms for the maxcut problem by
Mitchell~\cite{JEMising} and De Simone et
al.~\cite{desimone1,desimone2}, among others. The initial
relaxation used in these approaches is
\begin{eqnarray}
\label{initial_lp_relaxation}
\begin{array}{lc}
\max & c^Tx  \\
\mbox{s.t.} &  0 \hspace*{0.25cm} \le \hspace*{0.25cm} x
\hspace*{0.25cm} \le \hspace*{0.25cm} e,
\end{array}
\end{eqnarray}
where $c,x \in \real^{m}$, where $m$ is the number of edges in the
graph.

\section{Semidefinite formulations of the maxcut problem}
\label{sdp_formulations_of_maxcut} In this section we consider
semidefinite programming (SDP) formulations of the maxcut problem.
Our polyhedral cut and price algorithm presented in
\S\ref{maxcut_as_lp} works with this SDP formulation.

The maxcut problem can be modelled as an integer program using cut
vectors $d \in \{-1,1\}^n$ with $d_i = 1$ if $i \in S$, and $d_i =
-1$ for $i \in V \setminus S$. Consider the following problem
\begin{eqnarray}
\label{maxcut_as_pm1ip_1}
\begin{array}{ccc}
\max & \displaystyle \sum_{i,j=1}^n w_{ij} \frac{1-d_id_j}{4} \\
\mbox{s.t.} & d_i^2 = 1, & i=1,\ldots,n.
\end{array}
\end{eqnarray}
A factor of $\frac{1}{2}$ accounts for the fact that each edge is
considered twice. Moreover the expression $\frac{(1-d_id_j)}{2}$
is $0$ if $d_i = d_j$, i.e. if $i$ and $j$ are in the same set,
and $1$ if $d_i = -d_j$. Thus $\frac{(1-d_id_j)}{2}$ yields the
{\em incidence vector} of a cut associated with a cut vector $d$,
evaluating to $1$ if and only if edge $\{i,j\}$ is in the cut.

The Laplacian matrix of the graph $G$ is $L := \mbox{Diag}(Ae) -
A$, where $A$ is the weighted adjacency matrix of the graph. The
objective function in (\ref{maxcut_as_pm1ip_1}) can be rewritten
as $\frac{1}{4} d^TLd$. Now, consider the symmetric matrix $X =
dd^T$ of size $n$, whose $(i,j)$th entry is given by $X_{ij} =
d_id_j$. One can rewrite the maxcut problem as the following
optimization problem over the matrix variable $X$
\begin{eqnarray}
\label{maxcut_as_rank1sdp}
\begin{array}{lrcc}
\max & \frac{L}{4} \bullet X \\
\mbox{s.t.} & \mbox{diag}(X) & = & e \\
& X & \succeq & 0 \\
& \mbox{rank}(X) & = & 1.
\end{array}
\end{eqnarray}
Dropping the rank one restriction on the matrix $X$ gives the
following SDP relaxation of the maxcut problem
\begin{eqnarray}
\label{maxcut_as_sdp}
\begin{array}{lrcc}
\max & \frac{L}{4} \bullet X \\
\mbox{s.t.} & \mbox{diag}(X) & = & e \\
& X & \succeq & 0,
\end{array}
\end{eqnarray}
and its dual
\begin{eqnarray}
\label{maxcut_as_sdd}
\begin{array}{lrcc}
\min & e^Ty \\
\mbox{s.t.} & S & = & \mbox{Diag}(y) - \frac{L}{4} \\
& S & \succeq & 0,
\end{array}
\end{eqnarray}
which was originally derived in Laurent and Poljak
\cite{laurent_poljak} (see also Laurent \cite{laurent_maxcut},
Helmberg \cite{helmberg8} and Krishnan \& Terlaky
\cite{krishnan_terlaky}). We will refer to the feasible region of
(\ref{maxcut_as_sdp}) as the {\em elliptope}. It must be
emphasized that the elliptope is no longer a polytope. Thus
(\ref{maxcut_as_sdp}) is actually a non-polyhedral relaxation of
the maxcut problem.

This semidefinite formulation of the maxcut problem satisfies
strong duality, i.e., at optimality the objective values of
(\ref{maxcut_as_sdp}) and (\ref{maxcut_as_sdd}) are the same. In a
semidefinite cutting plane scheme for the maxcut problem, we would
begin with (\ref{maxcut_as_sdp}) as our starting SDP relaxation.
\begin{theorem}
\label{sdp_is_stronger} The SDP relaxation (\ref{maxcut_as_sdp})
is tighter than the LP relaxation (\ref{initial_lp_relaxation})
\end{theorem}
\begin{proof}
We will show that any feasible solution in the SDP relaxation
(\ref{maxcut_as_sdp}) gives a feasible solution in
(\ref{initial_lp_relaxation}).

Consider any feasible solution $X$ in (\ref{maxcut_as_sdp}). The
constraints $\mbox{diag}(X) = e$, and $X \succeq 0$ together imply
that $ -1 \le X_{ij} \le 1$.

We now transform the problem into $\{0,1\}$ variables using the
transformation $x_{ij} = \frac{1-X_{ij}}{2}$. This gives
$\frac{L}{4} \bullet X = \displaystyle \sum_{i=1}^n \displaystyle
\sum_{i < j, \{i,j\} \in E} w_{ij}x_{ij}$. Also, we have $ 0 \le
x_{ij} \le 1$, $\forall \{i,j\} \in E$. Such an $x$ is feasible in
(\ref{initial_lp_relaxation}).
\end{proof}
Goemans and Williamson \cite{goemans1} developed a 0.878
approximation algorithm for the maxcut problem when all the edge
weights are nonnegative. Their algorithm uses the solution $X$ to
the SDP relaxation (\ref{maxcut_as_sdp}), followed by an ingenious
randomized rounding procedure to generate the incidence vector $d$
of the cut. Since we employ the rounding scheme in our algorithm,
we briefly discuss the procedure in \S\ref{heuristics}. This 0.878
approximation ratio for the SDP formulation is a tremendous
improvement over previous known LP formulations, where the ratio
of the maxcut value to that of the LP relaxation over the odd
cycle polytope could be $\frac{1}{2}$ for various sparse random
graphs (Poljak and Tuza \cite{poljak_tuza}). This ratio improves
to $\frac{3}{4}$ if one considers dense random graphs, where it
can be further improved by considering $k$-gonal inequalities; in
fact one has a PTAS for dense graphs (Avis \& Umemoto
\cite{avis_umemoto}). On the negative side, H\r{a}stad
\cite{hastad1} showed that it is NP-hard to approximate the maxcut
problem to within a factor of 0.9412.

If we include negative edge weights and still have the Laplacian
matrix $L \succeq 0$, then Nesterov \cite{nesterov10} showed that
the GW rounding procedure gives an $\frac{2}{\pi}$ approximation
algorithm for the maxcut problem.

The solution obtained by the randomized rounding procedure can be
further improved using a Kernighan-Lin \cite{Kernighan1} (see also
Papadimitriou and Steiglitz \cite{papadimitriou_steiglitz}) local
search heuristic. Since we employ this heuristic too in our
algorithm, we discuss it in \S\ref{heuristics}.

One can further improve the relaxation (\ref{maxcut_as_sdp}) using
chordless odd cycle inequalities. These are simply the odd cycle
inequalities for the LP formulation in a $\pm 1$ setting, and have
the following form
\begin{eqnarray}
\label{odd_cycle_pm1}
\begin{array}{ccc}
X(C \backslash F) - X(F) & \le & |C|-2 \\
& & \mbox{for each cycle} \hspace*{0.25cm} C, F \subset C, |F|
\hspace*{0.25cm} \mbox{odd},
\end{array}
\end{eqnarray}
where $X(S) = \displaystyle \sum_{\{i,j\} \in S} X_{ij}$ etc.
These include among others the triangle inequalities. Since $
\mbox{diag}(X) = e$, and $X \succeq 0$ imply $-1 \le X_{ij} \le
1$, the feasible region of the SDP relaxation
(\ref{maxcut_as_sdp}) intersected with the odd cycle inequalities
(\ref{odd_cycle_pm1}) is contained within the odd cycle polytope.
We should thus get tighter upper bounds on the maxcut value.

Interestingly, although the additional inequalities improve the
SDP relaxation, they do not necessarily give rise to better
approximation algorithms. On the negative side, Karloff
\cite{karloff1} exhibited a set of graphs for which the optimal
solution to (\ref{maxcut_as_sdp}) satisfies all the triangle
inequalities as well, so after the Goemans-Williamson rounding
procedure we are still left with a $0.878$ approximation
algorithm.

More recently, Anjos and Wolkowicz \cite{anjos1} presented a
strengthened semidefinite relaxation of the maxcut problem. Their
SDP relaxation is tighter than the relaxation
(\ref{maxcut_as_sdp}) together with all the triangle inequalities.
To arrive at their relaxation they add certain redundant
constraints of the maxcut problem to (\ref{maxcut_as_sdp}), and
consider the Lagrangian dual of this problem. On the negative
side, this strengthened SDP relaxation is fairly expensive to
solve. Also, see Lasserre \cite{lasserre1,lasserre2} for a
hierarchy of increasingly tighter SDP relaxations for the maxcut
problem.

\section{An SDP based polyhedral cut and price method for the maxcut problem}
\label{maxcut_as_lp} This section deals with the major
contribution of the paper which is a polyhedral cut and price
algorithm, which is based on the SDP formulation of the maxcut
problem presented in \S\ref{sdp_formulations_of_maxcut}.

At the outset, we must mention that one could develop an SDP
cutting plane approach for solving the maxcut problem. The
approach sets up a series of primal SDP relaxations of the maxcut
problem beginning with (\ref{maxcut_as_sdp}), which are
progressively tightened by adding cutting planes, such as those
presented in \S\ref{sdp_formulations_of_maxcut}, to the primal SDP
relaxation. Such an SDP cutting plane approach for the maxcut
problem is discussed in Helmberg and Rendl \cite{helmberg1}.

Such an approach requires effective techniques for solving the
SDPs that arise in the algorithm. The main tools for solving SDPs
are interior point methods. Interior point methods, however, are
fairly limited in the size of problems they can handle. Problems
with more than 3000 constraints are considered quite hard (see
Helmberg \cite{helmberg8}).

Recently, several large scale approaches have been developed to
try to overcome this drawback. For example, Helmberg
\cite{helmberg10} has incorporated the spectral bundle method in a
cutting plane approach to solving the maxcut problem.

In this paper, we will solve each SDP in an interior point LP
cutting plane framework. Such an approach is discussed in Krishnan
and Mitchell \cite{kartik3,kartik2,kartik4,kartik5} and we present
some details in \S\ref{pricing_phase}. The overall approach is a
polyhedral cut and price approach for the maxcut problem that
mimics an SDP cutting plane approach. We note that in the past
such cut and price LP approaches have been applied with a great
degree of success in solving large scale integer programs (see
Wolsey \cite{wolsey}).

We motivate our polyhedral cut and price approach as follows. The
matrix variable $X$ is equal to $dd^T$ at optimality for
(\ref{maxcut_as_rank1sdp}), where $d$ is a $\pm 1$ vector
representing a cut. Let $d_i$ be a $\pm 1$ vector representing a
cut. We can write $X$ as a convex combination of all such vectors,
and so the maxcut problem (\ref{maxcut_as_pm1ip_1}) can also be
written as follows
\begin{eqnarray}
\label{maxcut_as_pm1ip_2}
\begin{array}{ccc}
\max & \frac{L}{4} \bullet (\displaystyle \sum_{i \in \mathcal{I}} x_id_id_i^T) \\
\mbox{s.t.} & \displaystyle \sum_{i \in \mathcal{I}} x_i d_{ij}^2
= 1, &
j=1,\ldots,n \\
%& \displaystyle \sum_i x_i = 1 \\
& x_i \ge 0, & i \in \mathcal{I},
\end{array}
\end{eqnarray}
where $\mathcal{I}$ is the index set of all the $\pm 1$ incidence
cut vectors. Note that all the equality constraints in
(\ref{maxcut_as_pm1ip_2}) are equivalent to the single constraint
$\displaystyle \sum_i x_i = 1$ for this choice of index set
$\mathcal{I}$. Moreover, $d_i^Td_i = n$, and there is an {\em
exponential} number of such vectors. If we allow $d_i$ instead to
lie on the $n$ dimensional sphere of radius $\sqrt{n}$, we obtain
the following relaxation of the maxcut problem
\begin{eqnarray}
\label{maxcut_as_inflp} \begin{array}{ccc} \displaystyle \max &
\frac{L}{4}
\bullet (\displaystyle \sum_{i \in \mathcal{J}} x_i d_id_i^T) \\
\mbox{s.t.} & \displaystyle \sum_{i \in \mathcal{J}} x_i d_{ij}^2
= 1, &
j=1,\ldots,n \\
%& \displaystyle \sum_i x_i = 1 \\
& x_i \ge 0, & \forall i \in \mathcal{J}, \end{array}
\end{eqnarray} where $\mathcal{J}$ is the index set of all the
vectors with Euclidean norm $\sqrt{n}$. In fact,
(\ref{maxcut_as_inflp}) is exactly the SDP formulation
(\ref{maxcut_as_sdp}) for the maxcut problem. Note that there are
now {\em infinitely} many terms in the summation, so
(\ref{maxcut_as_inflp}) is really a semi-infinite LP relaxation of
the maxcut problem. Note that summing the constraints
$\displaystyle \sum_{i \in \mathcal{J}} x_i d_{ij}^2 = 1$ implies
$\displaystyle \sum_{i \in \mathcal{J}} x_i = 1$, so the feasible
region is contained in the set of convex combinations of the
vectors~$d_i$.

We can solve (\ref{maxcut_as_inflp}) in a polyhedral column
generation (pricing) scheme, with a small number of $d_i \in
\real^n$ and generating others as necessary. This is reminiscent
of the Dantzig-Wolfe decomposition approach applied to
(\ref{maxcut_as_inflp}) and constitutes the pricing phase of the
algorithm. The key to this pricing phase is the existence of a
polynomial time separation oracle for the SDP relaxations. Of
course, we eventually want to solve the problem over $d_i$ that
are $\pm 1$ vectors. To do so, we introduce linear cutting planes
in (\ref{maxcut_as_inflp}), that cut off some of the non $\pm 1$
vectors not in the maxcut polytope; this is the cutting phase of
the algorithm. We elaborate on these two phases below:
\begin{enumerate}
\item {\bf Cutting phase:} The input to this phase is a feasible
point $X$ in (\ref{maxcut_as_sdp}). This is given to an oracle,
which verifies whether this point is also within the maxcut
polytope in a $\pm 1$ setting. If this point is not in this
polytope, the oracle returns linear cutting planes which are
facets of the maxcut polytope. The cutting planes are added to the
primal SDP relaxation, thereby strengthening it. This is identical
to the cutting phase of an SDP cutting plane approach, and the
hope is that the added cutting planes enforce the rank one
constraint (see the formulation (\ref{maxcut_as_rank1sdp}) of the
maxcut problem). \item {\bf Pricing phase:} The input to this
phase is an SDP relaxation of the maxcut problem. We approximately
solve the dual version of this SDP in a polyhedral interior point
cutting plane scheme. In particular, one sets up LP relaxations of
this dual SDP formulation, and approximately solves these
relaxations using interior point methods. The solution is then
given to a separation oracle for the SDP problem. The oracle
verifies whether this point is also in the dual SDP feasible
region. If this is not the case, the oracle returns linear cutting
planes, which are supporting hyperplanes of the dual SDP feasible
region. These cutting planes are added to the LP relaxation and
the process is repeated. The reader familiar with linear
programming decomposition will recognize the primal linear program
as the reformulation obtained when Dantzig-Wolfe decomposition is
applied to (\ref{maxcut_as_sdp}). Also, adding linear cutting
planes in the dual LP relaxation amounts to pricing (column
generation) in the primal LP setting. This is why we call this
phase our pricing phase. We present some details in
\S\ref{pricing_phase}.
\end{enumerate}
We now describe our entire algorithm in a nutshell. Each of these
steps will be elaborated in more detail in~\S\ref{sec_details}.

\begin{algorithm}{SDP cut-and-price algorithm for maxcut}
\label{algo_cnp}
\begin{enumerate}
\item{\bf Initialize}

\item{\bf Approximately solve the dual
maxcut SDP as an LP} using the pricing phase discussed in
\S\ref{pricing_phase}. This gives an upper bound on the maxcut
value. Construct the primal matrix $X$.

\item{\bf Generate good integer cut vectors} by running the
Goemans-Williamson and Kernighan-Lin heuristics on the primal
matrix $X$. These heuristics are presented in \S\ref{heuristics}.
Add these integer vectors into the LP approximation, and solve
this LP to optimality to find the new $X$.

\item{\bf Check for termination:} If the difference
between the upper bound and the value of the best cut is small,
return the best cut as our solution to the maxcut problem. STOP.

\item{\bf Finding violated odd cycle inequalities:} Feed the
matrix $X$ to a separation oracle that returns violated odd cycle
inequalities. This oracle is described in Barahona and Mahjoub
\cite{barahona4} and involves the solution of $n$ shortest path
problems on an auxiliary graph having twice the number of nodes
and four times the number of edges. Bucket sort the resulting
violated inequalities and add a subset of constraints to the
primal LP approximation. This changes the dual slack matrix $S$.

\item{\bf Update the LP relaxation} by choosing the most important
constraints in the primal polyhedral approximation including the
best integer cut vector, constraints based on the spectral
factorization of $X$, and the initial box constraints. Return to
Step 2.
\end{enumerate}
\end{algorithm}
Assuming that we do not resort to branch and bound, at the
termination of the cutting plane scheme we are approximately
solving
\begin{eqnarray}
\label{maxcut_as_sdp_oc}
\begin{array}{lrcc}
\max & \frac{L}{4} \bullet X \\
\mbox{s.t.} & \mbox{diag}(X) & = & e, \\
& \displaystyle \sum_{ij \in C \backslash F} X_{ij} - \displaystyle \sum_{ij \in F} X_{ij} & \le & |C|-2, \\
& \mbox{C} \hspace*{0.25cm} \mbox{a cycle}, & F \subseteq C, & |F| \hspace*{0.25cm} \mbox{odd}, \\
& X & \succeq & 0,
\end{array}
\end{eqnarray}
which is still a relaxation of the maxcut problem.

We briefly motivate the steps in the algorithm: In Step 2 we are
solving a relaxation of (\ref{maxcut_as_sdp_oc}) in a polyhedral
column generation scheme. We utilize this solution to get integer
cut vectors in Step 3; we then add these integer vectors to the
polyhedral approximation and resolve this LP. This ensures that
some of the $d_i$ in the LP are $\pm 1$ vectors representing the
best incidence cuts currently obtained by the algorithm. Moreover,
Step 3 ensures that the feasible region of the LP approximation
has a non-empty intersection with the maxcut polytope.  In Step 5,
we add violated odd cycle inequalities to the primal LP
approximation; these odd cycle inequalities are cutting off some
of the non $\pm 1$ $d_i$ vectors outside the maxcut polytope.
Since we are employing an IPM to solve the LP approximations it is
imperative that we keep the size of the LP approximations bounded.
Step 6 realizes this objective by aggregating the useful
information contained in the current polyhedral approximation.

We must mention that Gruber \cite{gruber3} has a similar approach,
where he solves a semi-infinite formulation of the primal SDP
relaxation (\ref{maxcut_as_sdp}). However, he does not motivate it
in this way; his aim is rather to strengthen the LP cutting plane
approach by adding cutting planes valid for the SDP approach. An
advantage of his approach is that he deals exclusively with the
primal problem. If the graph is chordal then it suffices to force
every submatrix of $X$ corresponding to a maximal clique to be
positive semidefinite, so it is not necessary to consider the
other entries in the primal matrix $X$ explicitly. Grone et
al.~\cite{grone1} showed that values for the missing entries can
be chosen to make $X$ positive semidefinite provided all these
submatrices are positive semidefinite.

The approach is advantageous only if the underlying graph is
chordal and sparse. If the graph is not chordal, then one needs to
construct a chordal extension of it, by adding redundant edges of
weight zero. This could lead to a fairly dense graph. Moreover,
since one is confined to the edge set of the original graph in
this approach, it is not clear how one could add hypermetric
inequalities (see, eg, \cite{deza_laurent}) without enlarging the
support of $X$.

\subsection{An LP pricing approach for SDP}
\label{pricing_phase} We describe the pricing phase (step 2 of
Algorithm~\ref{algo_cnp}) in detail in this section.

Consider the following SDP
\begin{eqnarray}
\label{sdp_relax}
\begin{array}{lrccc}
\max & C \bullet X \\
\mbox{s.t.} & A_i \bullet X & = & b_i, & i=1,\ldots,m \\
& X & \succeq & 0,
\end{array}
\end{eqnarray}
and its dual
\begin{eqnarray}
\label{sdd_relax}
\begin{array}{lrcc}
\min & b^Ty \\
\mbox{s.t.} & S & = & \displaystyle \sum_{i=1}^my_iA_i - C, \\
& S & \succeq & 0,
\end{array}
\end{eqnarray}
which are some of the relaxations of the maxcut problem
encountered during the course of the algorithm.

Our method is based on the semi-infinite formulation of
~(\ref{sdd_relax}), namely
\begin{eqnarray}
\begin{array}{lrcc}
\min & b^Ty \\
\mbox{s.t.} & S & = & \displaystyle \sum_{i=1}^my_iA_i - C, \\
& d^TSd & \geq & 0, \,\ \forall d \in B,
\end{array}
\end{eqnarray}
where $B$ is a compact set, typically $\{d : ||d||_2 \le 1\}$ or
$\{d : ||d||_{\infty} \le 1\}$. We described a cutting plane
approach to solving this relaxation in Krishnan and Mitchell
\cite{kartik3,kartik2,kartik4,kartik5}. This approach sets up a
relaxation to (\ref{sdd_relax}),
\begin{eqnarray}
\label{lp_relax_sdd}
\begin{array}{lccclr}
\min & b^Ty \\
\mbox{subject to} &  \displaystyle \sum_{j=1}^my_j(d_i^TA_jd_i) &
\geq & d_i^TCd_i &
     \mbox{for } i=1,\ldots,k,
\end{array}
\end{eqnarray}
with a corresponding constrained version of (\ref{sdp_relax}),
\begin{eqnarray}
\label{lp_relax_sdp}
\begin{array}{lrclr}
\max & C \bullet (\displaystyle \sum_{j=1}^k x_j d_j d_j^T)  \\
\mbox{subject to} & A_i \bullet (\displaystyle \sum_{j=1}^k x_j d_j d_j^T) & = & b_i & i=1,\ldots,m  \\
& x & \geq & 0.
\end{array}
\end{eqnarray}
We briefly present the pricing phase below:
\begin{algorithm}{Pricing procedure}
\begin{enumerate}
\item {\bf Initialize:} Start with an initial relaxation of the
form (\ref{lp_relax_sdd}). \item {\bf Solve the LP relaxation:}
Approximately solve (\ref{lp_relax_sdd}) and (\ref{lp_relax_sdp})
using a primal-dual interior point method to a prescribed
tolerance for the solution $(x,y)$. \item {\bf Call the SDP
separation oracle:} Compute the slack matrix $S$ at the
approximate solution $y$ to (\ref{lp_relax_sdd}). If $S$ is psd
then we proceed to Step 4. Else, if $S$ is indefinite, we add the
following cutting plane
\begin{displaymath}
\begin{array}{ccc}
\displaystyle \sum_{j=1}^my_j(d^TA_jd) & \geq & (d^TCd),
\end{array}
\end{displaymath}
where $d$ is an eigenvector corresponding to the minimum
eigenvalue of $S$ to (\ref{lp_relax_sdd}), update $k$, and return
to Step 2. \item {\bf Tighten the tolerance:} If the prescribed
tolerance is still above the minimum threshold, tighten the
tolerance, and proceed to Step 2. Else, the solution $(X,y)$,
where $X = \displaystyle \sum_{i=1}^kx_id_id_i^T$ is optimal for
the current SDP relaxation. STOP.
\end{enumerate}
\end{algorithm}
This cutting plane approach will be used as a subroutine to solve
the SDP relaxations which arise while solving the maxcut problem.

An important point needs to be emphasized here: We use a
primal-dual IPM to solve the LP relaxations approximately; the
query points at which we call the oracle are more central iterates
in the feasible region of (\ref{lp_relax_sdd}), where this oracle
returns better cuts. One could use the simplex method to solve the
LP relaxations or even an interior point method solving these
relaxations to optimality. This leads to the classical Kelley LP
cutting plane scheme \cite{Kelley} for convex optimization (or the
traditional Dantzig-Wolfe decomposition in the primal setting);
here the query points are extreme points of the LP feasible
region. It is known that the Kelley cutting plane method suffers
from tailing effects, and we observed in Krishnan \cite{kartik3}
(see also Elhedhli \& Goffin \cite{elhedhli_goffin}) that a
simplex implementation performed very badly. Some convergence
estimates for Kelley's method can be found in Lemarechal
\cite{lemarechal}.

\subsection{Heuristics for generating good incidence cut vectors}
\label{heuristics} We briefly mention the Goemans-Williamson
rounding procedure and the Kernighan-Lin local search heuristic in
this section. These heuristics will be employed in generating good
incidence cut vectors in step 3 of Algorithm~\ref{algo_cnp}.

\begin{algorithm}
{The Goemans-Williamson rounding procedure}
\begin{enumerate}
\item We assume we have a feasible solution $X$ in
(\ref{maxcut_as_sdp}) to start with. \item Compute $Z$ such that
$X = Z^TZ$. Let $z_i$, $i=1,\ldots,n$ be the $i$th column of the
matrix $Z$. \item Generate a random vector $r$ on the unit sphere
in $\real^n$. Assign vertex $i$ to $S$ if $z_i^Tr \ge 0$, and $V
\backslash S$ otherwise. Repeat this procedure for all the
vertices in the graph.
\end{enumerate}
\end{algorithm}
In practice, one repeats Step 3 a number of times, and picks the
best cut that was generated. We refer the reader to Goemans and
Williamson \cite{goemans1} for more details.

\begin{algorithm}
{\bf The Kernighan and Lin (KL) heuristic}
\begin{enumerate}
\item We assume that we have a cut to start with; this is given by
the bipartition $S$ and $V \backslash S$ of the vertex set $V$.
For each vertex $i$, $i=1,\ldots,n$, compute the external and
internal costs $E(i)$ and $I(i)$. For a vertex $a$ in $S$ this is
the following (the other case is treated similarly):
\begin{displaymath}
\begin{array}{ccc}
E(a) & = & \displaystyle \sum_{i \in V \backslash S} w_{ai}, \\
I(a) & = & \displaystyle \sum_{i \in S} w_{ai},
\end{array}
\end{displaymath}
where $w_{ij}$ is the weight of edge $\{i,j\}$. \item For each
vertex $i$, compute $D^1(i) = E(i) - I(i)$, $i=1,\ldots,n$. This
can be interpreted as follows: Sending the vertex $a$ from $S$ to
$V \backslash S$ will reduce the objective value of the cut by a
factor of $D(a)$. \item In the $k$ iteration, choose a vertex $b$
such that $D^k(b) = \min_i D^k(i)$ (not necessarily negative).
Suppose that $b$ goes from $S$ to $ V \backslash S$ (the other
case can be handled similarly). Let $g(k) = D^k(b)$. \item Mark
$b$ as {\em tabu}, implying that this vertex will not be
considered in future interchanges. Update $D(v)$ for all vertices
$v$ adjacent to $b$ as follows :
\begin{displaymath}
\begin{array}{cccc}
D^{k+1}(v) & = & D^k(v) + 2w_{bv}, & \forall v \in S, \\
D^{k+1}(v) & = & D^k(v) - 2w_{bv}, & \forall v \in V \backslash S.
\end{array}
\end{displaymath}
Also, $D^{k+1}(v) = D^k(v)$, for all vertices $v$ not adjacent to
$b$. \item If $k < n$ return to step 3, else go to step 6. \item
When $k=n$, we are done considering all the vertices. We are back
to the original cut, since we have swapped all the vertices, i.e.
roles of $S$ and $V \backslash S$ are reversed. Thus, we have
$\sum_{i=1}^ng(i) = 0$. \item Examine the cumulative function
$G(k) = \sum_{i=1}^kg(i)$, $k=1,\ldots,n$, and see where it is a
minimum. If $G(k) \ge 0$, $k=1,\ldots,n$ we stop. Else, choose $k$
such that $G(k)$ is most negative (break ties arbitrarily), and
perform the $k$ first interchanges. This will give a cut, which is
better than what we started with. \end{enumerate}\end{algorithm}
 One can repeat
this procedure until it runs out of steam. In our implementation,
we repeat the procedure $5$ times in all. We refer the reader to
Papadimitriou and Steiglitz \cite{papadimitriou_steiglitz} for
more details on the K-L heuristic.

\section{Details of the algorithm   \label{sec_details}}
In this section, we describe the steps of Algorithm~\ref{algo_cnp} in more detail.

Our initial dual LP relaxation is
\begin{eqnarray}
\label{dual_lp_initial}
\begin{array}{lrccc}
\min & \displaystyle \sum_{i=1}^n y_i  \\
\mbox{s.t.} & y_i & \ge & \frac{L_{ii}}{4}, & i=1,\ldots,n.
\end{array}
\end{eqnarray}
The initial constraints in (\ref{dual_lp_initial}) define
one-sided box constraints. They suffice for the maxcut problem
since the objective function is the all ones vector, and an
optimal solution is $(x^0,y^0,z^0) =
(e,\mbox{diag}(\frac{L}{4}),0) \in \real^n \times \real^n \times
\real^n$. This corresponds to initial matrices $X^0 = I$, and $S^0
= \frac{L}{4} - \mbox{Diag}(\frac{L}{4})$. We note that for other
problems we will in general need two sided box constraints to
ensure that the initial relaxation has a bounded objective value.

The current dual maxcut SDP relaxation is not solved to optimality
in step 2 of Algorithm~\ref{algo_cnp}; we mentioned some reasons in \S\ref{pricing_phase}. We
use a dynamically altered tolerance on the required accuracy of
the solution (for specifics, see
\S\ref{maxcut_computational_results}). However, if we do not solve
the SDP accurately enough, then the objective value of the LP
relaxation is not guaranteed to be an upper bound on the maxcut
value. In fact we typically have the following inequalities on the
objective values of the various relaxations.
\begin{displaymath}
\begin{array}{cccc}
\mbox{Cheap LP approximation to SDP} & \le & \mbox{Optimal Maxcut value} & \le \\
\mbox{Good LP approximation to SDP} & \le & \mbox{SDP objective
value}.
\end{array}
\end{displaymath}
The accuracy to which the SDP is solved also affects the
performance of the Goemans-Willamson rounding procedure. Using the
same analysis as Goemans and Williamson \cite{goemans1}, we can
show that the value of the cut generated by the procedure is at
least $0.878$ times the value of the current LP approximation, if
all the edge weights are nonnegative. However, we cannot always
guarantee that the value of our LP approximation is an upper bound
on the maxcut value. Thus, we do not have a performance guarantee
out here. Nevertheless, we can say in a weak sense that the more
accurately we solve the SDP as an LP in the cutting plane scheme,
the more likely that a better cut is produced.

It should be noted that we add cutting planes in both the primal
and dual spaces. We add cutting planes of the form
(\ref{odd_cycle_pm1}) (based on the polyhedral structure of the
maxcut polytope) in the primal space, while the cutting planes
added in the dual are sometimes low dimensional faces of the dual
SDP cone (in $y$ space), and not always facets (see also~ Krishnan
and Mitchell \cite{kartik5}); this amounts to column generation
(pricing) in the primal space.

As regards the primal, we cannot guarantee that our primal LP
feasible region always contains the entire maxcut polytope; in
most cases it does not. The primal LP feasible region increases,
when we solve the dual SDP as an LP (pricing), and decreases when
we add cutting planes in the primal.

When we are adding cutting planes in the dual, we update the
primal matrix $X = \displaystyle \sum_{i=1}^mx_id_id_i^T$. On the
other hand, when we add cutting planes in the primal we are
updating the dual slack matrix $ S = \mbox{Diag}(y) +
\displaystyle \sum_{i=1}^p y_{n+i}A_i - \frac{L}{4}$, where $A_i$
is the coefficient matrix associated with $i$th odd cycle
inequality. In each iteration of the algorithm, we have an LP
formulation of the dual maxcut SDP of the form
\begin{eqnarray}
\label{lp_formulation_sdd}
\begin{array}{cc}
\min & \displaystyle \sum_{i=1}^n y_i + \displaystyle \sum_{i=1}^{p} (|C_i|-2)y_{n+i} \\
\mbox{s.t.} & \left[\begin{array}{cccc}
                    I & 0 & \ldots & 0 \\
                    d_{n+1}.^2 & d_{n+1}^TA_1d_{n+1} & \ldots & d_{n+1}^TA_{p}d_{n+1} \\
                    \vdots & \vdots & \ddots & \vdots \\
                    d_{m}.^2 & d_{m}^T A_1d_{m} & \ldots & d_{m}^T A_{p}d_{m}
                    \end{array} \right] \left[\begin{array}{c}
                                              y_1 \\
                                              \vdots \\
                                              y_n \\
                                              y_{n+1} \\
                                              \vdots \\
                                              y_{n+p}
                                              \end{array}\right] \\
& \ge \hspace*{0.2cm}
\left[\begin{array}{c}
      \mbox{diag}(\frac{L}{4}) \\
       d_{n+1}^T \frac{L}{4} d_{n+1} \\
       \vdots \\
       d_{m}^T \frac{L}{4} d_{m}
       \end{array}\right] \\
& y_{n+i} \hspace*{0.2cm} \ge \hspace*{0.2cm}  0, \hspace*{0.25cm}
i=1,\ldots,p,
\end{array}
\end{eqnarray}
and its dual
\begin{eqnarray}
\label{lp_formulation_sdp}
\begin{array}{lrccc}
\max & \frac{L}{4} \bullet (\displaystyle \sum_{j=1}^{m}x_jd_jd_j^T) \\
\mbox{s.t.} & \mbox{diag}(\displaystyle \sum_{j=1}^{m}x_jd_jd_j^T) & = & e \\
& A_i \bullet (\displaystyle \sum_{j=1}^{m}x_jd_jd_j^T) & \le & (|C_i|-2), & i=1,\ldots,p \\
& x_j & \ge & 0, & j=1,\ldots,m.
\end{array}
\end{eqnarray}
An odd cycle inequality (\ref{odd_cycle_pm1}) is written as $ A
\bullet X \le (|C|-2)$, where $A$ is a symmetric matrix, and $|C|$
is the cardinality of the cycle. Also $x_j$ is the primal variable
corresponding to the $j$th dual constraint.

In step 2  of Algorithm~\ref{algo_cnp}, we improve the LP approximation to the
SDP using the pricing strategy outlined in \S\ref{pricing_phase}.
The LPs are solved approximately using a primal-dual interior
point method. Cutting planes are added to
(\ref{lp_formulation_sdd}), which try and force the dual slack
matrix $ S = \mbox{Diag}(y(1:n)) + \displaystyle
\sum_{i=1}^{p}A_iy_{n+i} - \frac{L}{4} $ to be psd. Appropriate
cuts can be found by looking for eigenvectors of the current dual
slack matrix with negative eigenvalues. At the end of step 2, we
have a solution $(X,y)$, where $X$ is feasible in the current
primal SDP relaxation of the maxcut problem.

To ensure that the feasible region of the current LP approximation
has a nonempty intersection with the maxcut polytope, we add good
incidence cut vectors to the LP approximation in step~3 of Algorithm~\ref{algo_cnp}. We run
the Goemans and Williamson rounding procedure on $X$, and generate
a number of good cut vectors. We then improve these cuts using the
Kernighan and Lin (KL) heuristic. We then add the $5$ best
incidence vectors as cuts in (\ref{lp_formulation_sdd}). These
heuristics were described in \S\ref{heuristics}. We resolve the
new LP for $(X,y)$.

The value of the best cut found so far gives a lower bound on the
optimal value of the maxcut problem. The value of any dual
feasible vector $\hat{y}$ for which the corresponding slack matrix
$\hat{S}$ is psd gives an upper bound on this optimal value. Such
a vector can be constructed from the current solution $y$ by
increasing its first $n$ components appropriately.  To compute
$\hat{y}$ we proceed as follows:  Compute the dual slack matrix
$S$ at the point $y$. Let $\lambda$ be the minimum eigenvalue of
$S$. Then set
\begin{displaymath}
\begin{array}{cccc}
\hat{y}_i & = & y_i - \lambda, & i=1,\ldots,n, \\
\hat{y}_i & = & y_i, & i=n+1,\ldots,n+p. \end{array}
\end{displaymath}
The objective value at $\hat{y}$ gives us our current upper bound.
The best such upper bound is stored, and when the upper bound and
lower bound are close enough the algorithm is terminated in step
4 of Algorithm~\ref{algo_cnp}.

Cutting planes for (\ref{lp_formulation_sdp}) are found in step 5
of Algorithm~\ref{algo_cnp}.
These odd cycle inequalities are violated by the current solution
$X$. We use the exact Barahona-Mahjoub separation oracle which
returns violated odd cycle inequalities (\ref{odd_cycle_pm1}),
which are facets of the maxcut polytope. These odd cycle
inequalities are bucket sorted by violation, and the most violated
ones are added to (\ref{lp_formulation_sdp}).

In order to keep the size of the relaxations manageable, the
constraint matrix is modified in step 6 of Algorithm~\ref{algo_cnp}. The rule is to drop those
constraints whose primal variables $x_j$ are zero or small. We
sort these constraints based on their $x$ values and retain the
$q$ (an upper bound is given below) best constraints ensuring that
the best integer vector is among these constraints. The modified
LP formulation (\ref{lp_formulation_sdd}) has the following
constraints :
\begin{enumerate}
\item The box constraints, i.e. the first $n$ constraints in the
earlier (\ref{lp_formulation_sdd}). \item The $q$ best constraints
from the earlier (\ref{lp_formulation_sdd}). \item The
eigenvectors $v_i$, $i=1,\ldots,r$ corresponding to nonzero
eigenvalues of $X$ ($r$ here is the rank of $X$). In a sense, the
eigenvectors corresponding to the nonzero eigenvalues {\em
aggregate} the important information contained in all the
constraints added so far.
\end{enumerate}
The number $q$ is chosen as $q = \sqrt{2n} - r$. Note that,
$\sqrt{2n}$ is an upper bound for the rank of at least one extreme
point optimal solution for (\ref{maxcut_as_sdp_oc}) with $p$ odd
cycle inequalities (see Pataki~\cite{pataki1}).
%When the algorithm loops back to step 2 after adding odd cycle
%inequalities, warm start information should be exploited in
%solving the new $(LDRC)$ and $(LPRC)$. Since the odd cycle
%inequalities returned in step 5 are designed to cut off the primal
%matrix $\hat{X}$, the old solution $(\hat{x},\hat{y})$ is not
%strictly feasible in the modified $(LDRC)$ and $(LPRC)$. We
%consider the primal LP relaxation first.

%A new primal matrix $\bar{X}$ is constructed as
%\begin{displaymath}
%\begin{array}{ccc}
%\bar{X} & := & \frac{1}{2}\{\displaystyle \sum_{j=1}^q \hat{x}_j
%d_{n+j}d_{n+j}^T + \displaystyle \sum_{i=1}^r \lambda_iv_iv_i^T\}
%\end{array}
%\end{displaymath}
%This matrix is positive semidefinite, but it may not satisfy the
%linear constraints of~(\ref{maxcut_as_sdp}). The old feasible $X$
%however differs from $\bar{X}$ by a positive sum of outer products
%of $d_j$ vectors, and this difference has a non-negative diagonal.
%
%Thus, one can add positive linear combinations of $e_ie_i^T$,
%$i=1,\ldots,n$ to $\bar{X}$ until
%\begin{displaymath}
%\tilde{X} := \displaystyle \sum_{i=1}^n \alpha_i e_ie_i^T +
%\bar{X}
%\end{displaymath}
%satisfies the equality constraints, i.e. $\mbox{diag}(\tilde{X}) =
%e$. To strictly satisfy the odd cycle inequalities $A_i \bullet X
%\le (|C_i|-2)$, $i=1,\ldots,p$, we backtrack towards $I$ along the
%straight line between $\tilde{X}$ and $I$. Since $I$ is the
%analytic center of the feasible region $\{X : \mbox{diag}(X) = e,
%X \succeq 0\}$, which contains the max cut polytope, a line search
%in this direction should enable us to strictly satisfy the
%violated odd cycle inequalities. The final primal matrix $X$
%enables us to find a strictly primal feasible solution~$\bar{x}$
%in $(LPRC)$.
%
%As regards the dual, perturb $\hat{S}$ until it is psd as follows.
%Set the new dual variables corresponding to the newly added odd
%cycle inequalities to~$1$. Calculate the dual slack matrix
%$\hat{S}$ with the old values of $\hat{y}$ together with these new
%components, so
%\begin{displaymath}
%\hat{S}=\mbox{Diag}(y(1:n)) + \displaystyle
%\sum_{i=1}^{p}A_iy_{n+i} - \frac{L}{4}
%\end{displaymath}
%where $p$ has been updated to reflect the added primal constraints.
%Let $\lambda$ be the magnitude of the most negative eigenvalue of $\hat{S}$.
%Compute
%\begin{displaymath}
%\begin{array}{ccll}
%\bar{y}_i & = & \bar{y}_i + \lambda  & \mbox{for } i=1,\ldots,n \\
%\end{array}
%\end{displaymath}
%and leave the other dual variables at their previous values.
%This gives a strictly feasible dual solution, since the $(n+i)$th dual constraint
%is equivalent to requiring $d_{n+i}^TS d_{n+i}$ be nonnegative.
%The desired strictly feasible starting point is
%$(\bar{x},\bar{y})$.
The overall SDP cutting plane method for maxcut is best summarized
in Figure \ref{ip_via_lp_fig}. The figure illustrates how the
primal LP feasible region evolves during the course of the
algorithm. Since we are solving a {\em constrained} version of the
primal SDP relaxation, the LP feasible region is always within the
corresponding primal SDP cone. The polytope shaded is the maxcut
polytope. Suppose we are in the $k$th iteration, and the LP
feasible region is the polytope $abcdea$. Since we have solved the
current dual SDP only approximately as an LP, the polytope
$abcdea$ does not contain the entire maxcut polytope. The Barahona
separation oracle returns an odd cycle inequality, that is
violated by the current LP solution $x^{k+1}$. The new LP feasible
region is now the polytope $abcgha$. We then try and grow this LP
feasible region, by adding cuts in the dual to try and force the
new dual slack matrix $S^{k+1}$ to be positive semidefinite. At
the end of the process, let $abefgh$ be the new LP feasible
region. Although, $abefgha$ does not contain the entire maxcut
polytope either, it is a better approximation than the earlier
polytope $abcdea$. Proceeding in this manner, we eventually hit
the optimal vertex $i$.
\begin{figure}[htbp]
\centering
%\rotatebox{270}{
\includegraphics[height=5in,width=5in]{ip_via_lp}
\caption{\label{ip_via_lp_fig} The polyhedral cut and price
algorithm for the maxcut problem}
\end{figure}

\section{Computational results}
\label{maxcut_computational_results} In this section we report our
preliminary computational experience. All results were obtained on
a {\em Sun Ultra 5.6}, {\em 440 MHz} machine with {\em 128 MB} of
memory. We carried out our tests on a number of maxcut problems
from {\em SDPLIB} \cite{SDPLIB}, the {\em DIMACS Implementation
Challenge} \cite{pataki4}, and some randomly generated Ising spin
glass problems, i.e. maxcut problems with $\pm 1$ edge weights
from Mitchell \cite{JEMising}. We compare our results with a
traditional SDP cutting plane scheme that uses SDPT3 (Toh et al
\cite{toh1}) to solve the SDP relaxations.

We implemented our entire approach within the {\em MATLAB}
environment. The LP solver was Zhang's {\em LIPSOL} \cite{zhang8}.
The SDP separation oracle used in step 2 of Algorithm~\ref{algo_cnp} employs MATLAB's Lanczos
solver. Finally, we used the Barahona-Mahjoub separation oracle to
generate valid cutting planes in the primal; our implementation
was based on the {\em CPLEX} \cite{cplex1} network optimization
solver to solve the shortest path problem as an LP. In practice,
one uses heuristics (see Berry and Goldberg \cite{berry_goldberg})
to generate the violated cuts, but we used the exact separation
oracle in our experiments.

The computational results are summarized in Table
\ref{maxcut_test_results}. The columns in the table represent
\begin{center}
\begin{tabular}{ll}
Name: &Problem name
\\  $n$: & Number of nodes
\\  $m$: & Number of edges
\\  Maxcut: & Objective value of the best incidence vector found.  \\
& If not optimal, the optimal value is noted.
\\  SDP1: & Objective value over the elliptope (\ref{maxcut_as_sdp})
\\  SDP2: & Objective value over the elliptope and odd cycles (\ref{maxcut_as_sdp_oc})
\\  UB: & The upper bound returned by the cutting plane scheme
\end{tabular}
\end{center}

\begin{table}[htbp]
\begin{center}
\begin{tabular}{|c|c|c|c|c|c|c|} \hline
Name & n & m & Maxcut(Opt) & SDP1 & SDP2& UB \\
\hline \hline gpp100 \footnotemark[1] & 100 & 269 & 210 & 221.69 &
211.27 & 212.56  \\ \hline gpp1241 \footnotemark[1] & 124 & 149 &
137 & 141.91 & 137 & 137.45  \\ \hline gpp1242 \footnotemark[1] &
124 & 318 & 256 & 269.64 & 257.66 & 258.60  \\ \hline gpp1243
\footnotemark[1] & 124 & 620 & 446 & 467.68 & 458.35 & 458.90  \\
\hline gpp1244 \footnotemark[1] & 124 & 1271 & 834 & 864.26 &
848.92 & 850.08 \\ \hline gpp2501 \footnotemark[1] & 250 & 331 &
305 & 317.23 & 305 & 307.83  \\ \hline gpp2502 \footnotemark[1] &
250 & 612 & 502 & 531.78 & 511.12 & 520.19  \\ \hline gpp5001
\footnotemark[1] & 500 & 625 & 574 & 598.11 & 576.64
\footnotemark[4] & 584.91 \\ \hline toruspm-8-50 \footnotemark[2]
& 512 & 1536 & 456 & 527.81 & 464.70 \footnotemark[4] & 472.28 \\
\hline torusg3-8 \footnotemark[2] & 512 & 1536 & 412.75 (416.84) &
457.36 & 417.69 \footnotemark[4] & 428.18 \\ \hline ising10
\footnotemark[3] & 100 & 200 & 70 & 79.22 & 70 & 70.70 \\ \hline
ising$20_1$ \footnotemark[3] & 400 & 800 & 282 & 314.25 & 282 &
288.04  \\ \hline ising$20_2$ \footnotemark[3] & 400 & 800 & 268 &
305.63 & 268 & 271.82 \\ \hline ising30 \footnotemark[3] & 900 &
1800 & 630 (632) & 709.00 & 632 & \\ \hline
\end{tabular}
\caption{\label{maxcut_test_results} Test Results on Maxcut}
\end{center}
\end{table}

%\footnotetext{All results for 10 iterations of the SDP scheme}
\footnotetext[1]{SDPLIB}
\footnotetext[2]{DIMACS}
\footnotetext[3]{Random Ising Spin glass problems (Mitchell)}
\footnotetext[4]{Best results obtained using an interior point approach}

We ran $10$ cutting plane iterations in all. In our LP subroutine
for the SDP, we add $5$ cutting planes in each iteration, our
starting tolerance is $1$, and this is lowered by a factor of $0.9$ in each
iteration. Initially, we solve our SDP relaxations very cheaply,
i.e. perform only $5$ LP cutting plane iterations, and we increase
this number if we are unable to improve the best incidence cut
vector. Also, we add the $\frac{n}{4}$ most violated odd cycle
inequalities returned by the Barahona-Mahjoub approach in each
iteration.
%
%We must mention that our technique of estimating upper bounds does
%not performs very well. Therefore to obtain the upper bounds
%mentioned above, we solve the final primal SDP relaxation with the
%odd cycle inequalities to a tolerance of 1e-8 using SDPT3 (Toh et
%al \cite{toh1}).
Interestingly, we were able to obtain the optimal integer solution
in most cases (except problems torusg3-8 and ising30). The former
problem is not easy, since the maxcut solution is not integer,
whereas the latter problem is currently as big as we can handle.
For most of the problems we do not have a proof of optimality,
since the SDP relaxation over the elliptope and odd cycles is not
enough, and there is yet some gap involved. For the Ising spin
glass problems optimizing over the intersection of the elliptope
and the odd cycle polytope is sufficient to give the optimal
solution.

To give an estimate of the times involved we performed the following experiment:
\begin{enumerate}
\item We ran our code against a standard SDP cutting plane scheme
that employed $SDPT3$ \cite{toh1} version 2.3 to solve the SDP
relaxations. \item We chose eight problems from SDPLIB. \item We
ran $10$ cutting plane iterations in all. \item In our approach,
we solve the SDP relaxations very cheaply ($5$ LP cutting plane
iterations), and in every $5$th iteration we solve this SDP
relaxation more accurately ($25$ iterations). In $SDPT3$ we solve
the SDP relaxations to a moderate tolerance $10^{-2}$, and in every
$5$th iteration we solve the SDP more accurately ($10^{-6}$). \item
We add the $\frac{n}{4}$ most violated odd cycle inequalities
returned by the Barahona-Mahjoub approach in each iteration.
\end{enumerate}
The results can be found in table \ref{maxcut_comparison}.
\begin{table}[htbp]
\begin{center}
\begin{tabular}{|c|c|c|c|c|c|c|} \hline
Problem & \multicolumn{3}{|c|}{SDPT3} & \multicolumn{3}{|c|}{Our approach} \\
Name & UB & LB & Time & UB & LB & Time \\ \hline \hline
gpp100 & 212.26 & 210 & 321 & 216.86 & 210 & 485 \\ \hline
gpp1241 & 137 & 137 & 197 %\footnotemark[1]
 & 138.33 & 137 & 488 \\ \hline
gpp1242 & 258.4 & 256 & 423 & 263.73 & 256 & 620 \\ \hline
gpp1243 & 458.86 & 446 & 409 & 461.92 & 446 & 832 \\ \hline
gpp1244 & 851.01 & 834 & 468 & 857.35 & 834 & 902 \\ \hline
gpp2501 & 305 & 305 & 2327 & 309.95 & 304 & 1073 \\ \hline
gpp2502 & 512.06 & 502 & 1739 & 521.06 & 502 & 1687 \\ \hline
gpp5001 & - & - & - & 583.65 & 574 & 2893 \\ \hline
\end{tabular}
\caption{\label{maxcut_comparison} Comparing two SDP cutting plane schemes for maxcut}
\end{center}
\end{table}
%\footnotetext[1]{Converged in 5 iterations}
Table
\ref{maxcut_comparison} shows how our cutting plane approach
scales with the problem size, and the increase in runtime with
problem size compares quite well with a traditional interior point
SDP cutting plane approach. The interior point approach solved the
problem gpp1241 to optimality in just $5$ iterations. On the other
hand on problem gpp5001, the interior point method was unable to
complete the stipulated $10$ iterations.


\section{Conclusions}
\label{maxcut_conclusions} We have presented a polyhedral cut and
price technique for approximating the maxcut problem; the
polyhedral approach is based on a SDP formulation of the maxcut
problem which is much tighter than traditional LP relaxations of
the maxcut problem. This linear approach allows one to potentially
solve large scale problems. Here are some conclusions based on our
preliminary computational results
\begin{enumerate}
\item It is worth reiterating that although our scheme behaves
like an SDP approach for the maxcut problem, it is entirely a
polyhedral LP approach. The SDP relaxations themselves are solved
within a cutting plane scheme, so our scheme is really a cutting
plane approach within another cutting plane approach. We refer to
the inner cutting plane approach as our pricing phase. \item The
cutting plane approach is able to pick the optimal maxcut
incidence vector quickly, but the certificate of optimality takes
time. \item We are currently working on a number of techniques to
improve our upper bounding procedure using information from the
eigenvector based on the most negative eigenvalue. Nevertheless,
we believe that if we are able to handle this difficulty, then our
approach can be effective on large scale maxcut problems. \item
The main computational task in our approach is the time taken by
the SDP separation oracle. Another difficulty is that the oracle
returns cutting planes that lead to dense linear programs. One way
is to use a variant of the {\em matrix completion} idea in Gruber
and Rendl \cite{gruber3}, and examine the positive
semidefiniteness of $S_K$ for any clique $K \subseteq V$ (here
$S_K$ is a sub-matrix of $S$ corresponding to only those entries
in the clique). This should speed up the separation oracle, which
will also return sparser constraints, since any cutting plane for
$S_k$ can be lifted to $\mathcal{S}^n$ by assigning zero entries
to all the components of the cutting plane not in~$K$. \item We
are also trying to incorporate this approach in a branch and cut
approach to solving the maxcut problem, i.e. we resort to
branching when we are not able to improve our best cut values
obtained so far, or our upper bounds. \item Helmberg
\cite{helmberg10} shows that better SDP relaxations can be
obtained by enlarging the number of cycles in the graph, i.e.
adding edges of weights $0$ between non-adjacent nodes. This gives
rise to a complete graph, and the only odd cycles now are triangle
inequalities, which can be inspected by complete enumeration.
Again, this in sharp contrast to the LP case, where adding these
redundant edges does not improve the relaxation. We hope to
utilize this feature in our cutting plane approach, in the near
future. \item We hope to extend this approach to other
combinatorial optimization problems such as min bisection, $k$ way
equipartition, and max stable set problems based on a Lovasz theta
SDP formulation.
\end{enumerate}

\section{Acknowledgements}
The authors would like to thank three anonymous referees whose
comments greatly improved the presentation in the paper.

\begin{thebibliography}{10}

\bibitem{anjos1}
M.~F. Anjos and H.~Wolkowicz.
\newblock {\em Strengthened semidefinite relaxations via a second lifting for the
  max-cut problem}.
\newblock Discrete Applied Mathematics, 119:79--106, 2002.

\bibitem{avis_umemoto}
D. Avis and J. Umemoto.
\newblock {\em Stronger linear programming relaxations of
max-cut}.
\newblock Mathematical Programming, 97:451--469, 2003.

\bibitem{barahona_et_al}
F.~Barahona, M.~Gr\"{o}tschel, M. J\"{u}nger, and G. Reinelt.
\newblock{\em An application of combinatorial optimization to
statistical physics and circuit layout design}.
\newblock Operations Research, 36:493--513, 1998.

\bibitem{barahona4}
F.~Barahona and A.~R. Mahjoub.
\newblock {\em On the cut polytope}.
\newblock Mathematical Programming, 36:157--173, 1986.

\bibitem{berry_goldberg}
J.~Berry and M.~Goldberg.
\newblock {\em Path optimization for graph partitioning problems}.
\newblock Discrete Applied Mathematics, 90:27--50, 1999.

\bibitem{SDPLIB}
B.~Borchers.
\newblock {\em SDPLIB 1.2, a library of semidefinite programming test problems}.
\newblock Optimization Methods and Software, 11:683--690, 1999.

\bibitem{chang_du}
K.~C. Chang and D.~Z. Du.
\newblock{\em Efficient algorithms for layout assignment
problems}.
\newblock IEEE Transactions on Computer Aided Design, 6:67--78,
1987.

\bibitem{cplex1}
{CPLEX Optimization Inc}.
\newblock {\em CPLEX Linear Optimizer and Mixed Integer Optimizer}.
\newblock Suite 279, 930 Tahoe Blvd. Bldg 802, Incline Village, NV 89541.

\bibitem{desimone1}
C.~{De Simone}, M.~Diehl, M.~J\"{u}nger, P.~Mutzel, G.~Reinelt,
and G.~Rinaldi.
\newblock {\em Exact ground states of {Ising} spin glasses: {New} experimental
  results with a branch and cut algorithm}.
\newblock Journal of Statistical Physics, 80:487--496, 1995.

\bibitem{desimone2}
C.~{De Simone}, M.~Diehl, M.~J\"{u}nger, P.~Mutzel, G.~Reinelt,
and G.~Rinaldi.
\newblock {\em Exact ground states of two-dimensional {$\pm J$} {Ising} spin
  glasses}.
\newblock Journal of Statistical Physics, 84:1363--1371, 1996.

%\bibitem{deza_grishukhin_laurent}
%M.~M. Deza, V.~P. Grishukhin, and M. Laurent.
%\newblock {\em The hypermetric cone is polyhedral}.
%\newblock Combinatorica, 13:397-411, 1993.
%

\bibitem{deza_laurent}
M.~M. Deza and M. Laurent.
\newblock {\em Geometry of cuts and metrics}.
\newblock Springer Verlag, Berlin Heidelberg, 1997.

\bibitem{elhedhli_goffin}
S.~Elhedhli and J.~L. Goffin,
\newblock {\em The integration of an interior-point cutting plane
method within a branch-and-price algorithm}.
\newblock Mathematical Programming, 100:267--294, 2004.

\bibitem{goemans1}
M.~X. Goemans and D.~P. Williamson.
\newblock {\em Improved Approximation Algorithms for Maximum Cut and Satisfiability
  Problems Using Semidefinite Programming}.
\newblock J. Assoc. Comput. Mach., 42:1115--1145, 1995.

\bibitem{grone1}
B.~Grone, C.R. Johnson, E.~Marques de~Sa, and H.~Wolkowicz.
\newblock {\em Positive definite completions of partial {Hermitian} matrices}.
\newblock Linear Algebra and its Applications, 58:109--124, 1984.

\bibitem{GLSbook}
M.~Gr{\"{o}}tschel, L.~Lovasz, and A.~Schrijver.
\newblock {\em Geometric Algorithms and Combinatorial Optimization}.
\newblock Springer-Verlag, Berlin, Germany, 1988.

\bibitem{gruber3}
G.~Gruber.
\newblock {\em On semidefinite programming and applications in combinatorial
  optimization}.
\newblock PhD thesis, University of Technology, Graz, Austria, April 2000.

\bibitem{hastad1}
J.~H\r{a}stad.
\newblock {\em Some optimal inapproximability results}.
\newblock Journal of the ACM, 48(4):798--859, 2001.

\bibitem{helmberg8}
C.~Helmberg.
\newblock {\em Semidefinite programming for combinatorial optimization}.
\newblock Technical Report ZR-00-34, TU Berlin, Konrad-Zuse-Zentrum, Berlin,
  October 2000.
\newblock Habilitationsschrift.

\bibitem{helmberg10}
C.~Helmberg.
\newblock {\em A cutting plane algorithm for large scale semidefinite relaxations}.
\newblock In {\em The Sharpest Cut}, edited by M. Gr\"{o}tschel, Festschrift in
  honor of M. Padberg's 60th birthday, MPS-SIAM 2004, pp. 233-256.

\bibitem{helmberg1}
C.~Helmberg and F.~Rendl.
\newblock {\em Solving quadratic (0,1)-problems by semidefinite programs and cutting
  planes}.
\newblock Mathematical Programming, 82:291--315, 1998.

\bibitem{karloff1}
H.~Karloff.
\newblock {\em How good is the {Goemans}-{Williamson} {MAX} {CUT}
algorithm?}
\newblock SIAM Journal on Computing, 29:336--350, 1999.

\bibitem{karp1}
R.~M. Karp.
\newblock {\em Reducibility among combinatorial problems}.
\newblock In Complexity of
  Computer Computations, edited by R.~E. Miller and J.~W. Thatcher,
  Plenum Press, New York, 1972, pp. 85-103.

\bibitem{Kelley}
J.~E. Kelley.
\newblock {\em The cutting plane method for solving convex
programs}.
\newblock Journal of SIAM, 8:703--712, 1960.

\bibitem{Kernighan1}
B.~W. Kernighan and S.~Lin.
\newblock {\em An efficient heuristic procedure for partitioning graphs}.
\newblock Bell System Technical Journal, 49:291--307, 1970.

\bibitem{kartik3}
K.~Krishnan.
\newblock {\em Linear programming approaches to semidefinite programming
  problems}.
\newblock PhD thesis, Mathematical Sciences, Rensselaer Polytechnic Institute,
  Troy, NY 12180, July 2002.

\bibitem{kartik2}
K.~Krishnan and J.~E. Mitchell.
\newblock {\em Semi-infinite linear programming approaches to semidefinite
  programming ({SDP}) problems}.
\newblock Fields Institute Communications
  Series, Volume 37, {\em Novel approaches to hard discrete optimization
  problems}, edited by P. Pardalos and H. Wolkowicz, pages 121--140, 2003.

\bibitem{kartik4}
K.~Krishnan and J.~E. Mitchell.
\newblock {\em An unifying framework for several cutting plane methods for
   semidefinite programming}.
\newblock AdvOL-Report No. 2004/2, Advanced Optimization
Laboratory, McMaster University, August 2004 (to appear in
Optimization Methods and Software, 2005).

\bibitem{kartik5}
K.~Krishnan and J.~E. Mitchell.
\newblock {\em Properties of a cutting plane method for semidefinite programming}.
\newblock Technical report, Mathematical Sciences, Rensselaer Polytechnic
  Institute, Troy, NY 12180, May 2003.

\bibitem{krishnan_terlaky}
K.~Krishnan and T.~Terlaky.
\newblock {\em Interior point and semidefinite approaches in
combinatorial optimization}.
\newblock AdvOL-Report No. 2004/1, Advanced Optimization
Laboratory, McMaster University, April 2004 (to appear in the
GERAD 25th Anniversary Volume on {\em Graphs and Combinatorial
Optimization}, edited by D. Avis, A. Hertz, and O. Marcotte,
Kluwer Academic Publishers, 2005).

\bibitem{lasserre1}
J.~B. Lasserre.
\newblock {\em Global optimization with polynomials and the problem of moments}.
\newblock SIAM Journal on Optimization, 11:796--817, 2001.

\bibitem{lasserre2}
J.~B. Lasserre.
\newblock {\em An explicit equivalent semidefinite program for
nonlinear 0-1 programs}.
\newblock SIAM Journal on Optimization, 12:756-769, 2002.

\bibitem{laurent_maxcut}
M. Laurent.
\newblock {\em A journey from some classical results to the
Goemans Williamson approximation algorithm for maxcut}.
\newblock Lecture given at CORE, Louvain La Neuve, September 1-12
2003. Available from http://homepages.cwl.nl/\til monique/

\bibitem{laurent_poljak}
M. Laurent and S. Poljak.
\newblock {\em On a positive semidefinite
relaxation of the cut polytope}.
\newblock Linear Algebra and its Applications, 223:439--461, 1995.

\bibitem{lemarechal}
C. Lemarechal.
\newblock {\em Non-differentiable Optimization}
\newblock in Optimization, G.L. Nemhauser, A.H.G. Rinnooy Kan, and M.J.
Todd, eds., North-Holland, New York, 529--572, 1989.

\bibitem{JEMising}
J.~E. Mitchell.
\newblock {\em Computational experience with an interior point cutting plane
  algorithm}.
\newblock SIAM Journal on Optimization, 10:1212--1227, 2000.

\bibitem{nesterov10}
Y.~E. Nesterov.
\newblock {\em Semidefinite relaxation and nonconvex quadratic optimization}.
\newblock Optimization Methods and Software, 9:141--160, 1998.

\bibitem{papadimitriou_steiglitz}
C.~H. Papadimitriou and K. Steiglitz.
\newblock {\em Combinatorial Optimization: Algorithms and Complexity}.
\newblock Dover Publications Inc., Mineola, New York, 1998.

\bibitem{pataki1}
G.~Pataki.
\newblock {\em On the rank of extreme matrices in semidefinite programs and the
  multiplicity of optimal eigenvalues}.
\newblock Mathematics of Operations Research, 23:339--358, 1998.

\bibitem{pataki4}
G. Pataki and S. H. Schmieta.
\newblock {\em The {DIMACS} library of mixed semidefinite, quadratic, and linear
programs}.
\newblock Available at
http://dimacs.rutgers.edu/Challenges/Seventh/Instances/

\bibitem{pinter}
R.~Y. Pinter
\newblock {\em Optimal layer assignment for interconnect}.
\newblock Journal of VLSI Computational Systems, 1:123--137, 1984.

\bibitem{poljak3}
S.~Poljak and Z.~Tuza.
\newblock {\em Maximum cuts and large bipartite subgraphs}.
\newblock In {\em Combinatorial Optimization}, volume~20 of DIMACS Series
  in Discrete Mathematics and Theoretical Computer Science, AMS/DIMACS, pp.
  181--244, 1995.

\bibitem{poljak_tuza}
S.~Poljak and Z.~Tuza.
\newblock {\em The expected relative error of
the polyhedral approximation of the maxcut problem}.
\newblock Operations Research Letters, 16:191-198, 1994.

\bibitem{toh1}
K.~C. Toh, M.~J. Todd, and R.~Tutuncu.
\newblock {\em SDPT3-- a {Matlab} software package for semidefinite programming}.
\newblock Optimization Methods and Software, 11:545--581, 1999.

\bibitem{wolsey}
L.~Wolsey.
\newblock {\em Integer Programming}.
\newblock John Wiley \& Sons, Inc., New York 1998.

\bibitem{zhang8}
Y.~Zhang.
\newblock {\em User's guide to LIPSOL: Linear programming interior point solvers
  v0.4}.
\newblock Optimization Methods and Software, 11:385--396, 1999.

\end{thebibliography}

\end{document}
