\documentclass[12pt]{article}

%\oddsidemargin -.5in
%\textwidth 7.5in
%\textheight 9in
%\topmargin -30pt

\usepackage[myheadings]{fullpage}

\newcommand{\re}{I\!\!R}

\newcommand{\xab}{\mbox{$x_{1,2}$}}
\newcommand{\xac}{\mbox{$x_{1,3}$}}
\newcommand{\xad}{\mbox{$x_{1,4}$}}
\newcommand{\xbc}{\mbox{$x_{2,3}$}}
\newcommand{\xbd}{\mbox{$x_{2,4}$}}
\newcommand{\xcd}{\mbox{$x_{3,4}$}}
%\pagestyle{myheadings}

\markboth{COIP Final 2005}{COIP Final 2005}

\begin{document}


\begin{center}
  {\large
  MATP6620/DSES6770 \\
  {\bf Combinatorial Optimization and Integer Programming } \\
  Fall 2005}
\end{center}

\begin{center}
  Final Exam, Friday, December 16, 2005.
\end{center}

Please do all three problems. Show all work. No books or calculators allowed.
You may use any result from class, the homeworks, or the texts, except where
stated.
You may use one sheet of handwritten notes.
The exam lasts three hours.

\begin{enumerate}
%  \item (20 points)
%You wish to choose from a set of possible investments $\{1,\ldots,7\}$,
%subject to the following constraints:
%\begin{enumerate}
%\item You cannot invest in all of them.
%\item Investment 1 must be chosen if investment 3 is chosen.
%\item Investment 5 can be chosen only if investment 2 is also chosen.
%\item You must choose either both investments 1 and 6 or neither.
%\item You must choose either at least one of the investments 1,2,3,
%and/or at least two investments from 2,5,6,7.
%\end{enumerate}
%Model this as a 0-1 integer programming feasibility problem.
%
%\pagebreak
%
%\addtocounter{page}{-1}
%(intentionally left blank)
%
%\pagebreak

%  \newpage
%  \setcounter{page}{1}  \quad \vspace*{\fill}  \quad
%  \newpage

   \item (30 points; each part is worth 10 points) \\
Consider the knapsack problem
\begin{displaymath}
\begin{array}{lrcrcrcrcrclr}
\max & 23x_1 & + & 17x_2 & + & 30x_3 & + & 14x_4 & + & 9x_5 \\
\mbox{subject to} & 6x_1 & + & 5x_2 & + & 10x_3 & + & 7x_4 & + & 5x_5 & \leq & 14 & \quad (KP) \\
& \multicolumn{11}{c}{x_i \in \{0,1\},  \; i=1,\ldots, 5}
\end{array}
\end{displaymath}
Let $P$ denote the feasible region of $(KP)$.

\begin{enumerate}
\item
Recall that a {\em cover} is a set of variables $x_j$ whose constraint coefficients $a_j$
add up to more than~$b$. Any feasible solution cannot have all of these $x_j=1$.
Give three minimal cover inequalities satisfied by all feasible solutions to~$(KP)$.
(Note: there are more than three minimal cover inequalities for this problem.)

\item
The inequality
\begin{displaymath}
x_1 + x_2 + x_4 + x_5 \leq 2
\end{displaymath}
is valid for $P \cap \{x \in \re^5: x_3=0\}$
Lift this inequality to give a valid inequality for~$P$.


\item
Let $s$ denote the slack in the original knapsack constraint,
and
let $s_i=1-x_i$
%denote a slack variable in the LP relaxation of $(KP)$,
%so $x_i+s_i=1$
for $i=1,\ldots,5$.
The optimal solution to the LP relaxation is to take $x_1=x_2=1$,
$x_3=0.3$, and $x_4=x_5=0$.
One row of the optimal simplex tableau
can be written
\begin{displaymath}
x_3 + 0.7x_4 + 0.5 x_5 - 0.6 s_1 - 0.5 s_2 + 0.1s =  0.3.
\end{displaymath}
Show that the Gomory cutting plane procedure applied to this constraint
gives a constraint equivalent to a cover inequality.
Is it a minimal cover inequality?
\end{enumerate}





  \item (50 points, each part is worth 10 points)  \\
Let $G=(V,E)$ be a graph with edge weights $w_e$
and let $R \subseteq V$ be a set of required vertices.
The {\bf Steiner tree problem} is to find the minimum weight tree
in~$G$ that includes every vertex in~$R$.
In the following complete graph on four vertices,
vertices $1$, $2$, and $3$ are required, so $R=\{1,2,3\}$.
Edge lengths are indicated.

\begin{center}
\begin{picture}(200,200)
\put(100,180){\circle{15}}
\put(97,176){1}
\put(100,80){\circle{15}}
\put(97,76){4}
\put(00,30){\circle{15}}
\put(-3,26){2}
\put(200,30){\circle{15}}
\put(197,26){3}

\put(4,36){\line(2,3){92}}
\put(196,36){\line(-2,3){92}}
\put(8,30){\line(1,0){184}}

\put(100,88){\line(0,1){84}}
\put(94,76){\line(-2,-1){88}}
\put(106,76){\line(2,-1){88}}

\put(40,100){5}
\put(155,100){5}
\put(100,32){5}

\put(102,120){3}
\put(50,60){3}
\put(140,60){3}
\end{picture}
\end{center}

       \begin{enumerate}
         \item
         A feasible solution to a Steiner tree problem can be obtained by finding a
         minimum spanning tree on the subgraph of $G$ induced by the vertices in~$R$.
         Show that the minimum spanning tree on this subgraph is not optimal
         for the Steiner tree problem on this graph.
         
         
         \item
         A first attempt to formulate
         the Steiner tree problem on this graph as an
         integer programming problem is the following:
         \begin{displaymath}
         \begin{array}{lrcrcrcrcrcrclr}
         \min & 5\xab & + & 5\xac & + & 5\xbc & + & 3\xad & + & 3 \xbd & + & 3\xcd  \\
         \mbox{s.t.} & \xab & + & \xac &&& + & \xad &&&&& \geq & 1 \\
         & \xab &&& + & \xbc &&& + & \xbd &&& \geq & 1 & \qquad (IP)  \\
         &&& \xac & + & \xbc &&&&& + & \xcd & \geq & 1  \\
         & \multicolumn{11}{c}{\xab, \xac, \xbc, \xad, \xbd, \xcd \mbox{ binary}}
         \end{array}
         \end{displaymath}
         The upper bound constraints $x_e \leq 1$ are redundant in the LP relaxation,
         so the dual to the LP relaxation can be written
         \begin{displaymath}
         \begin{array}{lrcrcrclr}
         \max & y_1 & + & y_2 & + & y_3 \\
         \mbox{s.t.} & y_1 & + & y_2 &&& \leq & 5  \\
         & y_1 &&& + & y_3 & \leq & 5  \\
         &&& y_2 & + & y_3 & \leq & 5 & \qquad (D) \\
         & y_1 &&&&& \leq & 3 \\
         &&& y_2 &&& \leq & 3 \\
         &&&&& y_3 & \leq & 3 \\
         & \multicolumn{5}{r}{y_1, y_2, y_3} & \geq & 0
         \end{array}
         \end{displaymath}
         Show that the optimal value of the LP relaxation is 7.5, by finding primal and dual
         feasible solutions with this value.
         

        \item   \label{part14vs23}
        \begin{enumerate}
        \item
        Why are the following constraints valid for the Steiner Tree Problem:
        \begin{displaymath}
        \begin{array}{rcrcrcrcrcrcl}
        \xab & + & \xac &&&&& + & \xbd & + & \xcd & \geq & 1  \\
        \xab &&& + & \xbc & + & \xad &&& + & \xcd & \geq & 1 \\
        && \xac & + & \xbc & + & \xad & + & \xbd &&& \geq & 1
        \end{array}
        \end{displaymath}
        \item
        Give a feasible integral solution to $(IP)$ that violates one of these constraints
        and does not correspond to a Steiner tree.
        \end{enumerate}

         \item  \label{part.sumall}
         Let $S$ be the set of incidence vectors of feasible solutions to the Steiner tree problem.
         The constraint
         \begin{equation}   \label{eqn.sumall}
         3\xab + 3\xac + 3\xbc + 2 \xad + 2 \xbd + 2 \xcd \geq 6
         \end{equation}
         is valid for~$S$.
         Show that adding this constraint to the LP relaxation of $(IP)$ results in
         an optimal value of~9, and hence gives a proof that the optimal Steiner tree
         uses the edges $(1,4)$, $(2,4)$, and $(3,4)$.
%         Show that the constraint $\sum_{e \in E} x_e \geq 2$ is valid for the Steiner tree problem.
%         Show that adding this constraint to the LP relaxation results in an
%         optimal value of~8.
%         (Note:
%         The primal solution you give should be fractional and should
%         (strictly) satisfy all the inequalities in part~\ref{part14vs23}.
%         The dual solution you give will only contain positive dual values for
%         the original variables $y_1$, $y_2$, $y_3$ and the dual variable
%         corresponding to the constraint $\sum_{e \in E} x_e \geq 2$.)
         
        
         \item
          By considering the valid constraints
          \begin{displaymath}
          \begin{array}{rcrcrcrcrcrcl}
          \xab & + & \xac & + & \xbc & + & \xad & + & \xbd &&& \geq & 2  \\
          \xab & + & \xac & + & \xbc & + & \xad &&& + & \xcd & \geq & 2  \\
          \xab & + & \xac & + & \xbc &&& + & \xbd & + & \xcd & \geq & 2  \\
          \end{array}
          \end{displaymath}
          or otherwise,
          show that constraint~(\ref{eqn.sumall}) does not define a facet
          of~$S$.
          (You may assume that $S$ has dimension equal to~6.)
%         Find another valid constraint that is violated by the solution to
%         part~\ref{part.sumall}.
       \end{enumerate}


   \item (20 points; each part is worth 10 points)  \\
   We have an integer program of the form
   \begin{displaymath}
   \begin{array}{lrclr}
   \min & c^Tx \\
   \mbox{s.t.} & Ax & \geq & b & \qquad (IP)  \\
   & x && \mbox{binary}
   \end{array}
   \end{displaymath}
   where $b>0$ and $c>0$, and each nonzero element $a_{ij}$ of $A$ satisfies
   $a_{ij} \geq b_i$.
         \begin{enumerate}
           \item
           Show that the optimal solution to the LP relaxation of $(IP)$
           can be used to find a feasible integer solution.
         
        
            \item
           Let $p$ be the maximum number of nonzeroes in any row of~$A$.
           Give a polynomial time algorithm to find a feasible integer
           solution that has value within a factor of $p$ of the optimal value.
         \end{enumerate}


\end{enumerate}

\end{document}
