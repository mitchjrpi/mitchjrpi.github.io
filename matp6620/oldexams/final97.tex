\documentstyle[12pt]{article}

%\pagestyle{empty}

\oddsidemargin -.5in
\textwidth 7in
\textheight 9.5in
\topmargin -30pt
\headsep 0in
\headheight 0in

\begin{document}

\begin{center}
  \begin{large}
     67.612/92.672/66.612 \\
 {\bf Combinatorial Optimization and Integer Programming}
  \end{large}
\end{center}
\begin{center}
  \begin{large}
                Final Exam, Spring 1997
  \end{large}
\end{center}

Take Home   \hfill    Due: 12 noon, Friday, 9 May, 1997.

\vspace{\baselineskip}

This is to be all your own work.  You may use any result from class,
homeworks, or the books and papers on reserve in the library.
Do not consult anybody or anything else. 
I can dispense hints to help you if you are stuck.
My phone numbers are 276--6915(O) and 346--2811(H).
You can also reach me by email at mitchj@rpi.edu.
I will have office hours each Thursday from 2--5pm.
I will be out of town between Friday, 2 May and Wednesday, 7 May,
inclusive.
The exam consists of four questions and is worth 100 points.

In order that I can display grades, please write a 4 digit number
on the front of your solution set.

\vspace{\baselineskip}

\begin{enumerate}
   \item (25 points)
         Consider the complete undirected graph $G=(V,E)$ on $n$ vertices.
         A travelling salesman tour on this graph
         can be described as a collection of edges
         $T$ such that the graph $G'=(V,T)$ is connected,
         $T$ contains $n$ edges, and each vertex is adjacent to two edges.
         Let $c_e$ be the cost of edge $e$ in~$E$.
         The travelling salesman problem is to find the minimum weight tour,
         that is
         \begin{displaymath}
            \min\{\sum_{e \in T}c_ex_e : \mbox{$T$ is a tour}\}.
         \end{displaymath}
         Let $\delta(i)$ denote the edges which are incident to vertex~$i$.
         \begin{enumerate}
           \item (10 points)
              Verify that the following problem is a Lagrangian
              relaxation of the travelling salesman problem:
              \begin{displaymath}
                \begin{array}{llr}
                   \min      & \sum_{e \in  T}c_ex_e +
         \sum_{i \in V}\lambda_i(2-\sum_{e \in \delta(i)}x_e)
               &     \qquad  \qquad  LR(\lambda)  \\
                   \mbox{st } & \mbox{$T$ is a spanning tree together with
                                        one additional edge.}
                \end{array}
              \end{displaymath}
           \item (5 points)
              What would it imply if the optimal solution to $LR(\lambda)$
              was a tour?
           \item (10 points)
              Solve $LR(\lambda)$ for the graph with edge lengths as in
              the following table,
%              with vertex weights $\lambda_1=1$, $\lambda_2=2$,
%              $\lambda_3=3$, $\lambda_4=2$, $\lambda_5=0$, and~$\lambda_6=1$.
              with vertex weights $\lambda_1=2$, $\lambda_2=4$,
              $\lambda_3=6$, $\lambda_4=4$, $\lambda_5=0$, and~$\lambda_6=2$.
              (You may assume that the solution to a problem of the form
              \begin{displaymath}
                 \min\{\sum_{e \in T}w_ex_e :
                    \mbox{$T$ is a spanning tree together with
                                        one additional edge}\}
              \end{displaymath}
              is obtained by finding the minimum weight spanning tree and then
              adding in the remaining shortest edge.)
\begin{displaymath}
%   \left[ \begin{array}{rrrrrr} -- & 6  & 10 & 10 & 9  & 7 \\
%                                6  & -- & 9  & 10 & 10 & 10 \\
%                                10 & 9  & -- & 10 & 12 & 10 \\
%                                10 & 10 & 10 & -- & 8  & 9 \\
%                                9  & 10 & 12 & 8  & -- & 4 \\
%                                7  & 10 & 10 & 9  & 4  & -- 
   \left[ \begin{array}{rrrrrr} -- & 12 & 20 & 20 & 17 & 15 \\
                                12 & -- & 18 & 21 & 17 & 20 \\
                                20 & 18 & -- & 20 & 24 & 21 \\
                                20 & 19 & 20 & -- & 16 & 19 \\
                                17 & 17 & 24 & 16 & -- &  8 \\
                                15 & 20 & 21 & 19 &  8 & -- 
           \end{array}  \right]  \quad
    \left. \begin{array}{c} v_1 \\ v_2 \\ v_3 \\ v_4 \\ v_5 \\ v_6 \end{array}
             \right.
\end{displaymath}
          \end{enumerate}
   \item (25 points; each part is worth 5 points.)
%The {\em Clustering Problem} is defined as follows:
%\begin{quote}
%  {\bf The Clustering Problem:}
Given the complete graph $K_n=:(V,E)$ on $n$ vertices
and edge weights $w_e$ on the edges, a {\em clustering}
of the vertices is obtained by choosing an integer $p$
and a partition of the vertices into $p$ sets $V_1,\ldots,V_p$ satisfying:
\begin{itemize}
 \item $V_i \cap V_j = \emptyset$ for $1\leq i < j \leq p$.
 \item $\cup_{i=1}^p V_i=V$.
\end{itemize}
The {\em incidence vector} of this clustering is defined by
\begin{displaymath}
  x_e = \left\{ \begin{array}{ll} 1 & \mbox{if the two endpoints of $e$
               are in the same set $V_j$}  \\  0 & \mbox{otherwise}
        \end{array}  \right.
\end{displaymath}
The {\em clustering problem} for this set of edge weights is then
\begin{displaymath}
\begin{array}{ll}
 \max & z:=\sum_{e\in E}w_ex_e \\
  \mbox{subject to } & x \mbox{ is the incidence vector of a}
                         \mbox{ clustering}
\end{array}
\end{displaymath}
The edge weights $w_e$ can be positive or negative.
If all the edge weights are positive, the optimal solution
is to set $p=1$ and put all the vertices in one set,
giving a value $z=\sum_{e \in E}w_e$.
If all the edge weights are negative, the optimal solution is
to set $p=n$ and put each vertex in its own set,
giving a value $z=0$.
The weights $w_e$ measure a ``distance'' between two vertices:
the larger this value, the more likely the vertices should be
in the same subset~$V_j$, and if $w_e$ is very negative,
the vertices will probably be in different sets in the optimal
solution.

Let $Q$ be the set of incidence vectors of clusterings
for $K_n$.
\begin{enumerate}
 \item Show that the dimension of $Q$ is $n(n-1)/2$, ie, $Q$
is full dimensional.
 \item Show that the inequality
\begin{equation}  \label{eqn.cluster}
  x_{ij} + x_{jk} - x_{ik}  \leq  1, \;\;  1 \leq i <j<k \leq n
\end{equation}
is valid for any point in Q.
 \item Show that inequality $(\ref{eqn.cluster})$ defines
a facet of the convex hull of~$Q$.
 \item \label{part.k4}
Take $n=4$.
Choose edge weights $w_e$ so that
the optimal solution to the following relaxation of the
clustering problem is not in the convex hull of~$Q$:
\begin{displaymath}
 \begin{array}{lccll}
  \max & \sum_{e \in E} w_e x_e \\
  \mbox{subject to} & x_{ij}+x_{jk}-x_{ik} & \leq & 1 & \mbox{ for }
      1 \leq i < j < k \leq 4 \\
  & 0 \;\;\; \leq \;\;\;  x_e & \leq & 1 & \mbox{ for } e \in E
 \end{array}
\end{displaymath}
 \item
Find a valid inequality that cuts off the point you found in
part~(\ref{part.k4}).
\end{enumerate}

   \item (25 points)
         Consider the travelling salesman problem (TSP) on the complete
         undirected graph. Let $X$ be an instance of the travelling salesman
         problem, let $OPT(X)$ be the optimal value of this instance, and
         let $A$ be a polynomial time algorithm for the TSP which finds a
         tour of length~$A(X)$. We know that (unless $\cal{P}=\cal{NP}$)
         algorithm $A$ will not find the optimal solution to every instance~$X$.
         Let $r$ be a given constant, $1 < r < \infty$.
         Show that we can not even guarantee $A(X) \leq rOPT(X)$,
         unless $\cal{P}=\cal{NP}$.

         If the edge lengths $c_{ij}$ satisfy the triangle inequality
         $c_{ij}+c_{jk}\geq c_{ik}$ for every three vertices $i$, $j$,
         and $k$ then Christofides Heuristic generates a tour that is
         guaranteed to be no more than $1.5 OPT(X)$.
         How does your earlier proof break down if the edge lengths
         must satisfy the triangle inequality?

   \item (25 points)
Consider the mixed integer programming problem $(MIP)$:
     \begin{displaymath}
       \begin{array}{lrrrrrrcl}
         \max         & 8x_1 & +9x_2 & +5x_3 & +6x_4 & -15y_1 & -10y_2 \\
         \mbox{s.t. } & x_1 &&  +x_3 &&&& \leq & 1 \\
                      & x_1 &&& +x_4  &&& \leq & 1 \\
                      && x_2 &  +x_3 &&&& \leq & 1 \\
                      && x_2 && +x_4  &&& \leq & 1 \\
                      &&&&& -y_1 & -y_2 & \leq & -1 \\
                      &    x_1 &&&&  -y_1 && \leq & 0 \\
                      &&   x_2  &&&  -y_1 && \leq & 0 \\
                      &&&   x_3   &&& -y_2 & \leq & 0 \\
                      &&&&  x_4    && -y_2 & \leq & 0 \\
                      &&&&&& x_j & \geq & 0  \\
                      &&&&&& y_i && \mbox{binary} \\
       \end{array}
     \end{displaymath}
     Solve this problem using Bender's decomposition.
     (Take the problem
     \begin{displaymath}
       \begin{array}{lrrrcl}
         \max         & z & -15y_1 & -10y_2 \\
         \mbox{s.t}   & z &&& \leq & 28 \\
                      &&& y_i && \mbox{binary}
       \end{array}
     \end{displaymath}
     as the initial relaxation $(RMP)$.)
\end{enumerate}


\end{document}
