\batchmode
\documentstyle[12pt]{article}
\makeatletter


\oddsidemargin -.5in
\textwidth 7in
\textheight 9.5in
\topmargin -30pt
\headsep 0in
\headheight 0in







\makeatletter
\AtBeginDocument{\input /afs/rpi.edu/home/93/mitchj/public_html/matp6620/final97.aux }

\makeatletter
\count@=\the\catcode`\_ \catcode`\_=8 
\newenvironment{tex2html_wrap}{}{} \catcode`\_=\count@
\makeatother
\let\mathon=$
\let\mathoff=$
\ifx\AtBeginDocument\undefined \newcommand{\AtBeginDocument}[1]{}\fi
\newbox\sizebox
\setlength{\hoffset}{0pt}\setlength{\voffset}{0pt}
\addtolength{\textheight}{\footskip}\setlength{\footskip}{0pt}
\addtolength{\textheight}{\topmargin}\setlength{\topmargin}{0pt}
\addtolength{\textheight}{\headheight}\setlength{\headheight}{0pt}
\addtolength{\textheight}{\headsep}\setlength{\headsep}{0pt}
\setlength{\textwidth}{349pt}
\newwrite\lthtmlwrite
\makeatletter
\let\realnormalsize=\normalsize
\global\topskip=2sp
\def\preveqno{}\let\real@float=\@float \let\realend@float=\end@float
\def\@float{\let\@savefreelist\@freelist\real@float}
\def\end@float{\realend@float\global\let\@freelist\@savefreelist}
\let\real@dbflt=\@dbflt \let\end@dblfloat=\end@float
\let\@largefloatcheck=\relax
\def\@dbflt{\let\@savefreelist\@freelist\real@dbflt}
\def\adjustnormalsize{\def\normalsize{\mathsurround=0pt \realnormalsize
 \parindent=0pt\abovedisplayskip=0pt\belowdisplayskip=0pt}\normalsize}%
\def\lthtmltypeout#1{{\let\protect\string\immediate\write\lthtmlwrite{#1}}}%
\newcommand\lthtmlhboxmathA{\adjustnormalsize\setbox\sizebox=\hbox\bgroup}%
\newcommand\lthtmlvboxmathA{\adjustnormalsize\setbox\sizebox=\vbox\bgroup%
 \let\ifinner=\iffalse }%
\newcommand\lthtmlboxmathZ{\@next\next\@currlist{}{\def\next{\voidb@x}}%
 \expandafter\box\next\egroup}%
\newcommand\lthtmlmathtype[1]{\def\lthtmlmathenv{#1}}%
\newcommand\lthtmllogmath{\lthtmltypeout{l2hSize %
:\lthtmlmathenv:\the\ht\sizebox::\the\dp\sizebox::\the\wd\sizebox.\preveqno}}%
\newcommand\lthtmlfigureA[1]{\let\@savefreelist\@freelist
       \lthtmlmathtype{#1}\lthtmlvboxmathA}%
\newcommand\lthtmlfigureZ{\lthtmlboxmathZ\lthtmllogmath\copy\sizebox
       \global\let\@freelist\@savefreelist}%
\newcommand\lthtmldisplayA[1]{\lthtmlmathtype{#1}\lthtmlvboxmathA}%
\newcommand\lthtmldisplayB[1]{\edef\preveqno{(\theequation)}%
  \lthtmldisplayA{#1}\let\@eqnnum\relax}%
\newcommand\lthtmldisplayZ{\lthtmlboxmathZ\lthtmllogmath\lthtmlsetmath}%
\newcommand\lthtmlinlinemathA[1]{\lthtmlmathtype{#1}\lthtmlhboxmathA  \vrule height1.5ex width0pt }%
\newcommand\lthtmlinlineA[1]{\lthtmlmathtype{#1}\lthtmlhboxmathA}%
\newcommand\lthtmlinlineZ{\egroup\expandafter\ifdim\dp\sizebox>0pt %
  \expandafter\centerinlinemath\fi\lthtmllogmath\lthtmlsetinline}
\newcommand\lthtmlinlinemathZ{\egroup\expandafter\ifdim\dp\sizebox>0pt %
  \expandafter\centerinlinemath\fi\lthtmllogmath\lthtmlsetmath}
\def\lthtmlsetinline{\hbox{\vrule width.1em\vtop{\vbox{%
  \kern.1em\copy\sizebox}\ifdim\dp\sizebox>0pt\kern.1em\else\kern.3pt\fi
  \ifdim\hsize>\wd\sizebox \hrule depth1pt\fi}}}
\def\lthtmlsetmath{\hbox{\vrule width.1em\vtop{\vbox{%
  \kern.1em\kern0.8 pt\hbox{\hglue.17em\copy\sizebox\hglue0.8 pt}}\kern.3pt%
  \ifdim\dp\sizebox>0pt\kern.1em\fi \kern0.8 pt%
  \ifdim\hsize>\wd\sizebox \hrule depth1pt\fi}}}
\def\centerinlinemath{%\dimen1=\ht\sizebox
  \dimen1=\ifdim\ht\sizebox<\dp\sizebox \dp\sizebox\else\ht\sizebox\fi
  \advance\dimen1by.5pt \vrule width0pt height\dimen1 depth\dimen1 
 \dp\sizebox=\dimen1\ht\sizebox=\dimen1\relax}

\def\lthtmlcheckvsize{\ifdim\ht\sizebox<\vsize\expandafter\vfill
  \else\expandafter\vss\fi}%
\makeatletter \tracingstats = 1 


\begin{document}
\pagestyle{empty}\thispagestyle{empty}%
\lthtmltypeout{latex2htmlLength hsize=\the\hsize}%
\lthtmltypeout{latex2htmlLength vsize=\the\vsize}%
\lthtmltypeout{latex2htmlLength hoffset=\the\hoffset}%
\lthtmltypeout{latex2htmlLength voffset=\the\voffset}%
\lthtmltypeout{latex2htmlLength topmargin=\the\topmargin}%
\lthtmltypeout{latex2htmlLength topskip=\the\topskip}%
\lthtmltypeout{latex2htmlLength headheight=\the\headheight}%
\lthtmltypeout{latex2htmlLength headsep=\the\headsep}%
\lthtmltypeout{latex2htmlLength parskip=\the\parskip}%
\lthtmltypeout{latex2htmlLength oddsidemargin=\the\oddsidemargin}%
\makeatletter
\if@twoside\lthtmltypeout{latex2htmlLength evensidemargin=\the\evensidemargin}%
\else\lthtmltypeout{latex2htmlLength evensidemargin=\the\oddsidemargin}\fi%
\makeatother

% !!! IMAGES START HERE !!!

{\newpage\clearpage
\lthtmldisplayA{displaymath15}%
\begin{displaymath}
            \min\{\sum_{e \in T}c_ex_e : \mbox{$T$\space is a tour}\}.
         \end{displaymath}%
\lthtmldisplayZ
\hfill\lthtmlcheckvsize\clearpage}

{\newpage\clearpage
\lthtmlinlinemathA{tex2html_wrap_inline133}%
$\delta(i)$%
\lthtmlinlinemathZ
\hfill\lthtmlcheckvsize\clearpage}

{\newpage\clearpage
\lthtmldisplayA{displaymath20}%
\begin{displaymath}
                \begin{array}{llr}
                   \min      & \sum_{e \in  T}c_ex_e +
         \sum_{i \in V}\lambda_i(2-\sum_{e \in \delta(i)}x_e)
               &     \qquad  \qquad  LR(\lambda)  \\
                   \mbox{st } & \mbox{$T$\space is a spanning tree together with
                                        one additional edge.}
                \end{array}
              \end{displaymath}%
\lthtmldisplayZ
\hfill\lthtmlcheckvsize\clearpage}

{\newpage\clearpage
\lthtmlinlinemathA{tex2html_wrap_inline139}%
$LR(\lambda)$%
\lthtmlinlinemathZ
\hfill\lthtmlcheckvsize\clearpage}

{\newpage\clearpage
\lthtmlinlinemathA{tex2html_wrap_inline143}%
$\lambda_1=2$%
\lthtmlinlinemathZ
\hfill\lthtmlcheckvsize\clearpage}

{\newpage\clearpage
\lthtmlinlinemathA{tex2html_wrap_inline145}%
$\lambda_2=4$%
\lthtmlinlinemathZ
\hfill\lthtmlcheckvsize\clearpage}

{\newpage\clearpage
\lthtmlinlinemathA{tex2html_wrap_inline147}%
$\lambda_3=6$%
\lthtmlinlinemathZ
\hfill\lthtmlcheckvsize\clearpage}

{\newpage\clearpage
\lthtmlinlinemathA{tex2html_wrap_inline149}%
$\lambda_4=4$%
\lthtmlinlinemathZ
\hfill\lthtmlcheckvsize\clearpage}

{\newpage\clearpage
\lthtmlinlinemathA{tex2html_wrap_inline151}%
$\lambda_5=0$%
\lthtmlinlinemathZ
\hfill\lthtmlcheckvsize\clearpage}

{\newpage\clearpage
\lthtmlinlinemathA{tex2html_wrap_inline153}%
$\lambda_6=2$%
\lthtmlinlinemathZ
\hfill\lthtmlcheckvsize\clearpage}

{\newpage\clearpage
\lthtmldisplayA{displaymath30}%
\begin{displaymath}
                 \min\{\sum_{e \in T}w_ex_e :
                    \mbox{$T$\space is a spanning tree together with
                                        one additional edge}\}
              \end{displaymath}%
\lthtmldisplayZ
\hfill\lthtmlcheckvsize\clearpage}

{\newpage\clearpage
\lthtmldisplayA{displaymath34}%
\begin{displaymath}
\left[ \begin{array}{rrrrrr} -- & 12 & 20 & 20 & 17 & 15 \\
                                12 & -- & 18 & 21 & 17 & 20 \\
                                20 & 18 & -- & 20 & 24 & 21 \\
                                20 & 19 & 20 & -- & 16 & 19 \\
                                17 & 17 & 24 & 16 & -- &  8 \\
                                15 & 20 & 21 & 19 &  8 & -- 
           \end{array}  \right]  \quad
    \left. \begin{array}{c} v_1 \\v_2 \\v_3 \\v_4 \\v_5 \\v_6 \end{array}
             \right.
\end{displaymath}%
\lthtmldisplayZ
\hfill\lthtmlcheckvsize\clearpage}

{\newpage\clearpage
\lthtmlinlinemathA{tex2html_wrap_inline167}%
$V_1,\ldots,V_p$%
\lthtmlinlinemathZ
\hfill\lthtmlcheckvsize\clearpage}

{\newpage\clearpage
\lthtmlinlinemathA{tex2html_wrap_inline169}%
$V_i \cap V_j = \emptyset$%
\lthtmlinlinemathZ
\hfill\lthtmlcheckvsize\clearpage}

{\newpage\clearpage
\lthtmlinlinemathA{tex2html_wrap_inline171}%
$1\leq i < j \leq p$%
\lthtmlinlinemathZ
\hfill\lthtmlcheckvsize\clearpage}

{\newpage\clearpage
\lthtmlinlinemathA{tex2html_wrap_inline173}%
$\cup_{i=1}^p V_i=V$%
\lthtmlinlinemathZ
\hfill\lthtmlcheckvsize\clearpage}

{\newpage\clearpage
\lthtmldisplayA{displaymath48}%
\begin{displaymath}
  x_e = \left\{ \begin{array}{ll} 1 & \mbox{if the two endpoints of $e$                are in the same set $V_j$ }  \\0 & \mbox{otherwise}
        \end{array}  \right.
\end{displaymath}%
\lthtmldisplayZ
\hfill\lthtmlcheckvsize\clearpage}

{\newpage\clearpage
\lthtmldisplayA{displaymath56}%
\begin{displaymath}
\begin{array}{ll}
 \max & z:=\sum_{e\in E}w_ex_e \\
  \mbox{subject to } & x \mbox{ is the incidence vector of a}
                         \mbox{ clustering}
\end{array}
\end{displaymath}%
\lthtmldisplayZ
\hfill\lthtmlcheckvsize\clearpage}

{\newpage\clearpage
\lthtmlinlinemathA{tex2html_wrap_inline183}%
$z=\sum_{e \in E}w_e$%
\lthtmlinlinemathZ
\hfill\lthtmlcheckvsize\clearpage}

{\newpage\clearpage
\lthtmldisplayA{displaymath67}%
\begin{displaymath}
  x_{ij} + x_{jk} - x_{ik}  \leq  1, \;\;  1 \leq i <j<k \leq n
\end{displaymath}%
\lthtmldisplayZ
\hfill\lthtmlcheckvsize\clearpage}

{\newpage\clearpage
\lthtmlinlinemathA{tex2html_wrap_inline205}%
% latex2html id marker 205
$(\ref{eqn.cluster})$%
\lthtmlinlinemathZ
\hfill\lthtmlcheckvsize\clearpage}

{\newpage\clearpage
\lthtmldisplayA{displaymath75}%
\begin{displaymath}
 \begin{array}{lccll}
  \max & \sum_{e \in E} w_e x_e \\
  \mbox{subject to} & x_{ij}+x_{jk}-x_{ik} & \leq & 1 & \mbox{ for }
      1 \leq i < j < k \leq 4 \\
  & 0 \;\;\; \leq \;\;\;  x_e & \leq & 1 & \mbox{ for } e \in E
 \end{array}
\end{displaymath}%
\lthtmldisplayZ
\hfill\lthtmlcheckvsize\clearpage}

{\newpage\clearpage
\lthtmlinlinemathA{tex2html_wrap_inline223}%
$\cal{P}=\cal{NP}$%
\lthtmlinlinemathZ
\hfill\lthtmlcheckvsize\clearpage}

{\newpage\clearpage
\lthtmlinlinemathA{tex2html_wrap_inline231}%
$1 < r < \infty$%
\lthtmlinlinemathZ
\hfill\lthtmlcheckvsize\clearpage}

{\newpage\clearpage
\lthtmlinlinemathA{tex2html_wrap_inline233}%
$A(X) \leq rOPT(X)$%
\lthtmlinlinemathZ
\hfill\lthtmlcheckvsize\clearpage}

{\newpage\clearpage
\lthtmlinlinemathA{tex2html_wrap_inline239}%
$c_{ij}+c_{jk}\geq c_{ik}$%
\lthtmlinlinemathZ
\hfill\lthtmlcheckvsize\clearpage}

{\newpage\clearpage
\lthtmldisplayA{displaymath97}%
\begin{displaymath}
       \begin{array}{lrrrrrrcl}
         \max         & 8x_1 & +9x_2 & +5x_3 & +6x_4 & -15y_1 & -10y_2 \\
         \mbox{s.t. } & x_1 &&  +x_3 &&&& \leq & 1 \\
                      & x_1 &&& +x_4  &&& \leq & 1 \\
                      && x_2 &  +x_3 &&&& \leq & 1 \\
                      && x_2 && +x_4  &&& \leq & 1 \\
                      &&&&& -y_1 & -y_2 & \leq & -1 \\
                      &    x_1 &&&&  -y_1 && \leq & 0 \\
                      &&   x_2  &&&  -y_1 && \leq & 0 \\
                      &&&   x_3   &&& -y_2 & \leq & 0 \\
                      &&&&  x_4    && -y_2 & \leq & 0 \\
                      &&&&&& x_j & \geq & 0  \\
                      &&&&&& y_i && \mbox{binary} \\
       \end{array}
     \end{displaymath}%
\lthtmldisplayZ
\hfill\lthtmlcheckvsize\clearpage}

{\newpage\clearpage
\lthtmldisplayA{displaymath104}%
\begin{displaymath}
       \begin{array}{lrrrcl}
         \max         & z & -15y_1 & -10y_2 \\
         \mbox{s.t}   & z &&& \leq & 28 \\
                      &&& y_i && \mbox{binary}
       \end{array}
     \end{displaymath}%
\lthtmldisplayZ
\hfill\lthtmlcheckvsize\clearpage}


\end{document}
