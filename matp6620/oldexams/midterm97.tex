\documentstyle[12pt]{article}

\oddsidemargin -.5in
\textwidth 7.5in
\textheight 9in
\topmargin -30pt

\pagestyle{myheadings}

\markboth{COIP Midterm 97}{COIP Midterm 97}

\begin{document}


\begin{center}
  {\large
  67.612/92.672 \\
  {\bf Combinatorial Optimization and Integer Programming } \\
  Spring 97}
\end{center}

\begin{center}
  Midterm Exam, Wednesday, April 9, 1997.
\end{center}

Please do all three problems. Show all work. No books or calculators allowed.
You may use any result from class, the homeworks, or the texts, except where
stated.
You may use one sheet of handwritten notes.
The exam lasts two hours.



\begin{enumerate}
  \item (25 points)
     The {\em max clique} feasibility problem can be described
     as follows:
     \begin{quote}
       An {\em instance} consists of a graph $G=(V,E)$ and an integer $k$.
       The answer is {\em YES} if the graph contains a clique with at
       least $k$ vertices; otherwise, the answer is~{\em NO}.
     \end{quote}
     Show that the {\em max clique} problem is $NP$-complete.
     (Hint: You may assume the following problems are $NP$-complete:
     {\sc Satisfiability}, {\sc 3-Satisfiability},
     {\sc Node Packing}, {\sc Hamiltonian Path}, {\sc Hamiltonian Circuit}.
     It should be possible to use one of these in a reduction.
     Instances of these problems are defined as follows:
     \begin{quote}
       {\sc Satisfiability:} Given a set of boolean variables
        and a set of clauses consisting of
        literals (each of which is a variable or a negated variable)
        does there exist
        an assignment for the variables so that at least one literal
        in each clause is {\em true}?  \\
       {\sc 3-Satisfiability:}
        An instance of satisfiability where each clause contains exactly
        three literals.  \\
       {\sc Node Packing:}
        Given a graph $G=(V,E)$ and a integer $k$,
        does there exist a subset $U \subseteq V$
        with $\mid \! U \! \mid = k$
        where no two of the vertices in $U$ share an edge?  \\
       {\sc Hamiltonian Path:}
        Given a graph $G=(V,E)$, does there exist a path which visits
        every vertex?  \\
       {\sc Hamiltonian Circuit:}
        Given a graph $G=(V,E)$, does there exist a circuit which visits
        every vertex?)
     \end{quote}
  \item (50 points)
      \begin{enumerate}
         \item (42 points; each part is worth six points.) \label{particular}
            Consider the binary knapsack problem:
            \begin{displaymath}
              \begin{array}{lccrcrcrcrcrcl}
     \max  & z & :=      & 10x_1 & + & 8x_2 & + & 5x_3 & + & 2x_4 & + & x_5 \\
     \mbox{st }  &&&  2x_1 & + & 4x_2 & + & 5x_3 & + & 3x_4 & + & 4x_5 &
                                 \leq & 10
                                 \qquad \qquad (KP)  \\
                 &&&&&&&&& x_i && \mbox{binary.}
              \end{array}
            \end{displaymath}
            Let $P=\{x \in {\Re}^5_+:2x_1+4x_2+5x_3+3x_4+4x_5\leq 10,
                    0 \leq x_i \leq 1\}$ and~$S=P\cap{\cal Z}^5$.
            \begin{enumerate}
              \item
                 Solve the LP-relaxation
                 $\max\{z : x \in P \}$.
                 (Hint: You can solve the relaxation by ordering the variables
                 in decreasing order of $c_j/a_j$ and then successively
                 using as much as
                 possible of the next available variable until all of the
                 resource $b$ is consumed. Here, $c_j$ are
                 the objective function coefficients, $a_j$ are the
                 constraint coefficients, and $b$ is the right hand side
                 of the constraint.)
              \item
                 What is the dimension of $S$?
              \item
                 Prove that $x_1+x_2+x_3 \leq 2$ is a valid inequality for~$S$.
              \item
                 Prove that $x_1+x_2+x_3 \leq 2$ is a facet of conv($S$),
                 the convex hull of~$S$.
              \item
                 What is the Chvatal-Gomory rank of the inequality
                 $x_1+x_2+x_3 \leq 2$?
              \item
                 Prove that $x_3+x_4+x_5 \leq 2$ is a valid inequality for~$S$.
              \item
                 Show that the inequality $x_3+x_4+x_5 \leq 2$ can be lifted
                 to give a stronger valid inequality for~$S$.
            \end{enumerate}
         \item (8 points)
            Consider the general binary knapsack problem:
            \begin{displaymath}
              \begin{array}{lrcl}
                \max          & \sum_{i=1}^n c_ix_i \\
                \mbox{s.t. }  & \sum_{i=1}^n a_ix_i & \leq & b \\
                              &            x_i  && \mbox{binary,}
              \end{array}
            \end{displaymath}
            where $b>0$, and $a_i>0$, $c_i>0$ for $i=1,\ldots,n$.
            Generalize the results of part~\ref{particular} to give a class of
            valid inequalities for this problem.
      \end{enumerate}
  \item (25 points)
          Consider the 0-1 equality knapsack problem
          \begin{displaymath}
             \max \{-x_{n+1} : 2x_1 + 2x_2 + \ldots + 2x_n + x_{n+1} = n,
                         x \in B^n \},  \qquad\qquad (KE)
          \end{displaymath}
          where $n$ is an odd integer.
          We wish to solve this problem using a branch and bound algorithm
          with linear programming relaxations.
       \begin{enumerate}
         \item (5 points)
          Show that $x_{n+1}=1$ in any feasible solution to~$(KE)$. 
         \item (10 points)
          Assume
          we always separate using variable dichotomies
          (ie, add the constraints $x_i = 0$ or $x_i = 1$ for some $i$).
          Show that if no more than $\lfloor n/2 \rfloor$
          of the variables $x_1,\ldots,x_n$
          have been fixed at 0 or 1 then the relaxation has value~0.
          Hence show that an exponential number of nodes of the
          branch and bound tree must be considered to solve the problem.
         \item (10 points)
          What is the minimum number of nodes we need if we can separate
          using {\em any} linear inequality?
          (Assume that you obtain an extreme point optimal solution to any
          relaxation that you set up, provided such a solution exists.)
       \end{enumerate}
\end{enumerate}

\end{document}
