\documentclass[12pt]{article}

\oddsidemargin -.5in
\textwidth 7.5in
\textheight 9in
\topmargin -30pt

\pagestyle{myheadings}

\markboth{COIP Final 2003}{COIP Final 2003}

\begin{document}

\setcounter{page}{0}

\begin{tabular}{cl}
  \hspace{5in} & Name:
\end{tabular}

\begin{center}
  {\large
  MATP6620/DSES6770 \\
  {\bf Combinatorial Optimization and Integer Programming } \\
  Spring 2003}
\end{center}

\begin{center}
  Final Exam, Tuesday, May 6, 2003.
\end{center}

Please do all five problems. Show all work. No books or calculators allowed.
You may use any result from class, the homeworks, or the texts, except where
stated.
You may use one sheet of handwritten notes.
The exam lasts three hours.

\vspace{2in}

\begin{center}
\begin{tabular}{c|@{\hspace*{1in}}cr}
  Q1 & \qquad & /20 \\ \hline
  Q2 & \qquad & /20 \\ \hline
  Q3 & \qquad & /25 \\ \hline
  Q4 & \qquad & /15 \\ \hline
  Q5 & \qquad & /20 \\ \hline \hline
  Total && /100
\end{tabular}
\end{center}

\pagebreak

\begin{enumerate}
%  \item (20 points)
%You wish to choose from a set of possible investments $\{1,\ldots,7\}$,
%subject to the following constraints:
%\begin{enumerate}
%\item You cannot invest in all of them.
%\item Investment 1 must be chosen if investment 3 is chosen.
%\item Investment 5 can be chosen only if investment 2 is also chosen.
%\item You must choose either both investments 1 and 6 or neither.
%\item You must choose either at least one of the investments 1,2,3,
%and/or at least two investments from 2,5,6,7.
%\end{enumerate}
%Model this as a 0-1 integer programming feasibility problem.
%
%\pagebreak
%
%\addtocounter{page}{-1}
%(intentionally left blank)
%
%\pagebreak

%  \newpage
%  \setcounter{page}{1}  \quad \vspace*{\fill}  \quad
%  \newpage

   \item (20 points) \\
     The {\em max clique} feasibility problem can be described
     as follows:
     \begin{quote}
       An {\em instance} consists of a graph $G=(V,E)$ and an integer $k$.
       The answer is {\em YES} if the graph contains a clique with at
       least $k$ vertices; otherwise, the answer is~{\em NO}.
     \end{quote}
     Show that the {\em max clique} problem is $NP$-complete.

     (Hint: You may want to look at the {\em complement} of $G=(V,E)$,
     that is, the graph $G'=(V,E')$ where $e$ is in $E'$ if and only if
     it is not in~$E$.
     You may also want to look at the        {\em node packing} problem:
    \begin{quote}
        Given a graph $G=(V,E)$ and a integer $k$,
        does there exist a subset $U \subseteq V$
        with $\mid \! U \! \mid = k$
        where no two of the vertices in $U$ share an edge?)
   \end{quote}


\pagebreak

\addtocounter{page}{-1}
(intentionally left blank)

\pagebreak






  \item (20 points, each part is worth five points)  \\
          Consider the 0-1 equality knapsack problem
          \begin{displaymath}
             \min \{x_{n+1} : 2x_1 + 2x_2 + \ldots + 2x_n + x_{n+1} = n,
                         x \in B^n \},  \qquad\qquad (KE)
          \end{displaymath}
          where $n$ is an odd integer.
          We wish to solve this problem using a branch and bound algorithm
          with linear programming relaxations.
       \begin{enumerate}
         \item
          Show that $x_{n+1}=1$ in any feasible solution to~$(KE)$. 
          What is the optimal value of~$(KE)$?
         \item
          Assume
          we always separate using variable dichotomies
          (ie, add the constraints $x_i = 0$ or $x_i = 1$ for some $i$).
          Show that if no more than $\lfloor n/2 \rfloor$
          of the variables $x_1,\ldots,x_n$
          have been fixed at 0 or 1 then the relaxation has value~0.
          Hence show that an exponential number of nodes of the
          branch and bound tree must be considered to solve the problem.
         \item
          Derive the valid inequality
\begin{displaymath}
             x_1 + x_2 + \ldots + x_n \leq \lfloor n/2 \rfloor
\end{displaymath}
          using the Chvatal-Gomory rounding procedure.
         \item
          What is the minimum number of nodes we need in the branch and bound
          process if we can separate using {\em any} linear inequality?
          (Assume that you obtain an extreme point optimal solution to any
          relaxation that you set up, provided such a solution exists.)
       \end{enumerate}

\pagebreak

\addtocounter{page}{-1}
(intentionally left blank)

\pagebreak



   \item (25 points; each part is worth 5 points)  \\
         We have $n$ objects and
         we want to find an {\em ordering} of the objects,
         so, given two objects $i$ and $j$, either $i$ is before $j$
         or $j$ is before~$i$.
         The ordering should be {\em consistent};
         that is, if $i$ is before $j$ and $j$ is before $k$ then $i$ should
         be before~$k$.
         We get a reward $w_{ij}$ if $i$ is before $j$
         and a reward $w_{ji}$ if $j$ is before~$i$.
         The objective is to maximize
         \begin{displaymath}
           \begin{array}{cc}
              \sum  & w_{ij}.  \\
              \mbox{all pairs $i$, $j$,}  &  \\
              \mbox{$i$ before $j$}
           \end{array}
         \end{displaymath}
         We can model this as an integer programming problem by introducing
         variables $x_{ij}$ defined as follows:
         \begin{displaymath}
            x_{ij} = \left\{ \begin{array}{ll}
                        1 & \mbox{if $i$ is before $j$} \\
                        0 & \mbox{otherwise}
                     \end{array}  \right.
         \end{displaymath}
         Let $S$ be the set of points $x$ which correspond to orderings;
         let $conv(S)$ be the convex hull of~$S$.
        This formulation has $n(n-1)$ variables.
        You may assume that the dimension of $conv(S)$ is~$n(n-1)/2$.
         \begin{enumerate}
           \item Show that $x_{ij}+x_{ji}=1$ is valid for every ordering.
              \label{valideq}
%           \item Show that the dimension of $conv(S)$ is~$n(n-1)/2$.
%              (Hint: This means that only linear combinations of the
%              equalities given in part~\ref{valideq} are valid throughout~$S$.
%              Assume the contrary and derive a contradiction.)
           \item Show that the inequality $x_{ij}+x_{jk}+x_{ki}\leq 2$
              is valid for $S$ for any $i$, $j$, and~$k$.   \label{dicycle}
           \item For this part only, let $n=3$. Show that
              $x_{12}+x_{23}+x_{31}\leq 2$ is a facet of~$conv(S)$.
%           \item By lifting, or otherwise, show that the inequality
%              $x_{ij}+x_{jk}+x_{ki}\leq 2$ is a facet of $conv(S)$ for any
%              $i$, $j$, and~$k$.
           \item Let $U=\{u_1,\ldots,u_k\}$ and $W=\{w_1,\ldots,w_k\}$
              be two subsets of the objects. Show that the inequality
              \begin{displaymath}
                \sum_{u_i \in U, w_j \in W, i \neq j} x_{w_j,u_i}
                   + \sum_{i=1}^k x_{u_i,w_i} \leq k^2-k+1
              \end{displaymath}
              is valid for~$S$. (Note that the inequality includes $k$ edges pointing
up in the picture and $k(k-1)=k^2-k$ edges pointing down.)
 \label{fence.ineq}

\begin{center}
\begin{picture}(340,200)
\put(20,50){\circle*{5}}
\put(20,150){\circle*{5}}
\put(120,50){\circle*{5}}
\put(120,150){\circle*{5}}
\put(220,50){\circle*{5}}
\put(220,150){\circle*{5}}
\put(320,50){\circle*{5}}
\put(320,150){\circle*{5}}

\put(20,35){$u_1$}
\put(120,35){$u_2$}
\put(220,35){$u_3$}
\put(320,35){$u_4$}
\put(20,160){$w_1$}
\put(120,160){$w_2$}
\put(220,160){$w_3$}
\put(320,160){$w_4$}

\put(20,50){\vector(0,1){97}}
\put(120,50){\vector(0,1){97}}
\put(220,50){\vector(0,1){97}}
\put(320,50){\vector(0,1){97}}

\put(20,150){\vector(1,-1){98}}
\put(20,150){\vector(2,-1){198}}
\put(20,150){\vector(3,-1){298}}

\put(120,150){\vector(1,-1){98}}
\put(120,150){\vector(2,-1){198}}
\put(120,150){\vector(-1,-1){98}}

\put(220,150){\vector(1,-1){98}}
\put(220,150){\vector(-2,-1){198}}
\put(220,150){\vector(-1,-1){98}}

\put(320,150){\vector(-1,-1){98}}
\put(320,150){\vector(-2,-1){198}}
\put(320,150){\vector(-3,-1){298}}
\end{picture}
\end{center}

           \item Find a point $\bar{x}$, $0\leq \bar{x}_i \leq 1$,
              which satisfies all the equalities in part~\ref{valideq}
              and all the inequalities in part~\ref{dicycle},
              but which violates the inequality in part~\ref{fence.ineq}.
             (Hint: such a point exists when $k=3$.)
         \end{enumerate}

 \pagebreak

\addtocounter{page}{-1}
(intentionally left blank)

\pagebreak

\addtocounter{page}{-1}
(intentionally left blank)

\pagebreak

\item (15 points; each part is worth 5 points) \\
   Consider the integer programming problem
     \begin{displaymath}
      \begin{array}{lrcrcrcrcrcl}
       \min & -x_1 & + & 2x_2 & + & 3x_3 &
                   - & 10x_4 & + & 7x_5 \\
       \mbox{subject to} & 2x_1 & + & x_2 &&& + & 5x_4 & - & 3x_5 & = & 1 \\
                         & -2x_1 &&& + & x_3 &- & 6x_4 & + & 4x_5 & = & 1 \\
                     &&&&&&&   x  & \geq & 0, & \mbox{and} & \mbox{integer}
       \end{array}
     \end{displaymath}
One feasible integral solution is $x=(0,1,1,0,0)$, which has value~5.
The linear programming relaxation of this problem is obtained by ignoring
the condition that the variables $x$ should be integral.
The optimal tableau for the LP relaxation is
\begin{displaymath}
\begin{tabular}{|rrrrr|r|}
\hline
2 & 1 & 0.5 & 0 & 0 & -3.5 \\  \hline
1 & 2 & 1.5 & 1 & 0 & 3.5 \\
1 & 3 & 2.5 & 0 & 1 & 5.5 \\ \hline
\end{tabular}
\end{displaymath}
\begin{enumerate}
 \item What is the dual problem to the LP relaxation?
 \item Show that any feasible solution to the LP relaxation with
$x_1 \geq 1$ has value at least 5.5.
%(Hint: There are many ways to show this.
%For example, you could add the constraint $x_1 \geq 1$ to the
%given tableau and reoptimize using the dual simplex method.)
 \item What can you say about the value of $x_1$ in any optimal solution?
\end{enumerate}

\pagebreak

\addtocounter{page}{-1}
(intentionally left blank)

\pagebreak

\item (20 points) \\
When we looked at tabu search, we considered the problem of finding
the minimum spanning tree in a graph subject to some side constraints.
The tabu search process gives a feasible solution, and hence an upper bound
on the optimal value.
How would you find a {\em lower} bound on the optimal value?
(Note: please include details. You will not get much credit for
a two-word answer!)

\pagebreak

\addtocounter{page}{-1}
(intentionally left blank)



\end{enumerate}

\end{document}
