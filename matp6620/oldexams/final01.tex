\documentstyle[12pt]{article}

%\pagestyle{empty}

\oddsidemargin -.5in
\textwidth 7in
\textheight 9.5in
\topmargin -30pt
\headsep 0in
\headheight 0in

\newcommand{\til}{\char '176}

\begin{document}

\begin{center}
  \begin{large}
     MATP6620/DSES6760 \\
 {\bf Combinatorial Optimization and Integer Programming}
  \end{large}
\end{center}
\begin{center}
  \begin{large}
                Final Exam, Spring 2001
  \end{large}
\end{center}

Take Home   \hfill    Due: 5pm, Monday, 30 April, 2001.

\vspace{\baselineskip}

This is to be all your own work.  You may use any result from class,
homeworks, or the books and papers on reserve in the library.
Do not consult anybody or anything else. 
I can dispense hints to help you if you are stuck.
%My phone numbers are 276--6915(O) and 346--2811(H).
My office extension is 276--6915 and
you can also reach me by email at mitchj@rpi.edu.
I will have office hours Wednesday from 1--3pm
and Friday 2--4pm.
The exam consists of four questions and is worth 100 points.

In order that I can display grades, please write a 4 digit number
on the front of your solution set.

\vspace{\baselineskip}

\begin{enumerate}
  \item (25 points)
    Consider a bipartite graph $G=(V,E)$, where $V$ can be broken into
    two parts $U$ and $W$, and every edge in $E$ has one endpoint in $U$
    and one in~$W$.
    (Formally, $V=U\cup W$, $U\cap W =\emptyset$, and $e \in E$ implies that
    $e=(i,j)$, where $i\in U$ and $j\in W$.)
    %Let $k=\mid\!U\!\mid$ and $l=\mid\!W\!\mid$.
    Note that we have not assumed %that $\mid\!U\!\mid=\mid\!W\!\mid$, or
    that every vertex in $U$ is adjacent to every vertex in~$W$.
    Because this is a bipartite graph, the maximum cardinality matching
    problem can be solved by solving its linear programming relaxation
    \begin{displaymath}
      \begin{array}{lrcllr}
        \max & \sum_{(i,j) \in E} x_{ij} \\
        \mbox{subject to } & \sum_{j \in W: (i,j) \in E} x_{ij}
                                & \leq & 1, & \forall i \in U & \qquad (P) \\
                           & \sum_{i \in U: (i,j) \in E} x_{ij}
                                & \leq & 1, & \forall j \in W \\
                           & x_{ij} & \geq & 0, & (i,j) \in E
      \end{array}
    \end{displaymath}
    where $x_{ij}$ is one if edge $(i,j)$ is in the matching,
    and zero otherwise.
    Every basic feasible solution to both $(P)$ and its dual $(D)$ is integral.
    \begin{enumerate}
      \item (5 points)
        What is the dual $(D)$ to the linear program~$(P)$?
        (Hint: The dual of a linear program of the form
        $\max\{c^Tx:Ax\leq b,x\geq 0\}$ is $\min\{b^Ty:A^Ty\geq c, y\geq 0\}$.)
      \item (10 points)
        A {\em node cover} of $G$ is a subset $S$ of $V$ such that every
        edge in $E$ is incident to at least one vertex in~$S$.
        What do the integral solutions to $(D)$ correspond to?
        What do you conclude from strong duality?
      \item (10 points)
        What are the complementary slackness conditions for the
        pair $(P)$ and $(D)$? Interpret these conditions.
    \end{enumerate}
  \item (20 points)
    \begin{enumerate}
      \item (10 points)
        Let $M=(N,F)$ be a matroid defined on the finite set $N$ and with
        independent sets $F$.
        The {\bf dual matroid} $\bar{M}$ 
        to $M$ can be defined as the independence
        system on the finite set $N$ with its maximal independent sets
        equal to the complements of the maximal independent sets in~$M$.
        Show that $\bar{M}$ is a matroid.
        (Note: The dual matroid is defined in a different way in Nemhauser
        and Wolsey. I want you to use the definition I've given you here
        to prove this result, and not to use the definition in the text.)
      \item (10 points)
        A {\em matric matroid} $M_1=(N_1,F_1)$
        can be represented using a matrix:
        elements of the finite set $N_1$ correspond to columns of the matrix,
        and the independent sets in $F_1$ correspond to
        linearly independent subsets of the columns.
        A {\em graphic matroid} $M_2=(N_2,F_2)$ can be represented
        using a graph:
        elements of the finite set $N_2$ correspond to edges of the graph,
        and the independent sets in $F_2$ correspond to acyclic subsets
        of the edges.
        Show that any graphic matroid is also a matric matroid.
    \end{enumerate}

   \item (30+10 points)   \label{equi_lp}
       Let $G$ be a complete graph on $2s$ vertices, for some integer~$s\geq 3$.
       Let each edge $e$ have weight~$c_e$.
       The {\em equipartition problem} on this graph is to divide the vertices
       into two sets of size $s$ so as to minimize the sum of the weights of
       the edges that have one endpoint in each set. One polyhedral
       representation of this problem requires defining variables
       \begin{displaymath}
         x_e := \left\{ \begin{array}{ll}
           1 & \mbox{if the endpoints are in different sets}  \\
           0 & \mbox{otherwise}
         \end{array}  \right.
       \end{displaymath}
       for each edge~$e$.
       \begin{enumerate}
         \item (5 points)  \label{degree}
            Show that $x$ must satisfy the equality
            $\sum_{e \in \delta(v)}x_e=s$
            for each vertex $v$, where $\delta(v)$ denotes the set of
            edges incident to vertex~$v$.
         \item (5 points)
            Show that the dimension of the feasible region is no more than
            \begin{displaymath}
              \mbox{dim}(Q) \leq \left( \begin{array}{c}2s\\2\end{array}
                 \right) - 2s.
            \end{displaymath}
         \item (Extra credit: 10 points)
            Show that the dimension of the feasible region is exactly
            \begin{displaymath}
              \mbox{dim}(Q) = \left( \begin{array}{c}2s\\2\end{array}
                 \right) - 2s.
            \end{displaymath}
         \item (5 points)  \label{tri}
            Let $C$ be a cycle of length 3 and let $E(C)$ denote the
            edges of this cycle.
            Show that any feasible solution satisfies
            \begin{displaymath}
              \sum_{e \in E(C)} x_e \leq 2.
            \end{displaymath}
         \item (5 points)  \label{cycle}
            Let $C$ be a cycle of length $s+1$ and let $E(C)$ denote the
            edges of this cycle.
            Show that any feasible solution satisfies
            \begin{displaymath}
              \sum_{e \in E(C)} x_e \geq 2.
            \end{displaymath}
         \item (10 points)
            Solve the instance of the equipartition problem contained in
            \begin{quote}
              http://www.rpi.edu/\til mitchj/matp6620/final/equi.mod
            \end{quote}
            and
            \begin{quote}
              http://www.rpi.edu/\til mitchj/matp6620/final/equi8.dat
            \end{quote}
            using a cutting plane method.
       \end{enumerate}

  \item (25 points)
Another approach to the equipartition problem is to define a variable
\begin{displaymath}
w_{kj} := \left\{ \begin{array}{ll}
1 & \mbox{if vertex $j$ is on side $k$}  \\
0 & \mbox{otherwise}
\end{array}  \right.
\end{displaymath}
where $k$ takes the values 1 and 2, corresponding to the two sides
of the equipartition.
Let $g^{(p)}$ denote the $p$-vector with every entry equal to one.
\begin{enumerate}
\item (8 points)
Let $W$ denote the $2 \times 2s$ matrix whose $(k,j)$th entry is $w_{kj}$.
Show that the entries of $W$ satisfy $Wg^{(2s)}=sg^{(2)}$ and
$W^Tg^{(2)}=g^{(2s)}$.
\item (8 points)
Let $X$ denote the $2s \times 2s$ matrix whose $(i,j)$th entry is
the variable $x_{ij}$ defined in Question~\ref{equi_lp} corresponding
to the edge $e=(i,j)$.
Show that $E-X=W^TW$ for any feasible equipartition, where $E$ denotes
a matrix whose every entry is one.
\item (9 points)
Show that an SDP relaxation of the equipartition problem is:
\begin{displaymath}
\begin{array}{lrcl}
\min_Z & \frac{1}{2}C \bullet E - \frac{1}{2} C \bullet Z \\
\mbox{subject to} & Zg^{(2s)} & = & sg^{(2s)} \\
& Z_{ii} & = & 1 \\
& Z & \succeq & 0 \mbox{ and symmetric}
\end{array}
\end{displaymath}
where $Z$ is a $2s \times 2s$ matrix and $C_{ij}=c_e$ when the edge
$e=(i,j)$.
How does the matrix $Z$ relate to the matrices $X$ and $W$ given earlier?
How can we exploit the results of Question~\ref{equi_lp} in this
SDP formulation?
What have we relaxed to arrive at this formulation?
\end{enumerate}


\end{enumerate}


\end{document}
