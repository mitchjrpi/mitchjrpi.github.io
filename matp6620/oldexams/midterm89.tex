\documentstyle[12pt]{article}

\oddsidemargin -.5in
\textheight 9in
\textwidth 7.2in
\topmargin -30pt
\headsep 0pt
\headheight 0pt

\begin{document}

\begin{center}
   67.612 Integer Programming and Combinatorial Optimization
\end{center}

\begin{center}
   Thursday, March 16, 1989. \hspace{1in}  2 hours.
\end{center}

Answer all questions.

\begin{enumerate}
   \item  (20 points.)
          Find the minimum weight spanning tree in the graph below.
          Specify the algorithm you use and the order in which you
          construct your tree.

   \newpage

   \item  (15 points.)
          Consider using the Bellman-Ford algorithm to find the shortest
          path from $s$ to $t$ in a graph with edge lengths as follows:

\begin{tabular}{lc|rrrrrr|}
   \multicolumn{4}{c}{} & TO &  \multicolumn{3}{c}{}  \\
   \multicolumn{2}{c}{}  &  \multicolumn{1}{c}{s}
                       &  \multicolumn{1}{c}{1}
                       &  \multicolumn{1}{c}{2}
                       &  \multicolumn{1}{c}{3}
                       &  \multicolumn{1}{c}{4}
                       &  \multicolumn{1}{c}{t}   \\
  \cline{3-8}
   &  s  &  ---  &   3   &   2   &  100  &  100  &  100  \\
   &  1  &  100  &  ---  &  -2   &   3   &  100  &  100  \\
  FROM
   &  2  &  100  &   4   &  ---  &   2   &   6   &  100  \\
   &  3  &  100  &  100  &  100  &  ---  &   3   &  -4   \\
   &  4  &  100  &  100  &  -4   &  100  &  ---  &  -8   \\
   &  t  &  100  &  100  &  100  &  100  &  100  &  ---  \\
  \cline{3-8}
\end{tabular}

(So, for example, the length of the arc from node $s$ to node 2 is 2.
`---' indicates the corresponding edge does not exist.)

Let $a_i^k$ be the length of the shortest path from $s$ to $i$ which uses
at most $k$ edges. How would you calculate $a_i^{k+1}$?

We find $a_s^3=0$, $a_1^3=3$, $a_2^3=1$, $a_3^3=3$, $a_4^3=7$, and
$a_t^3=0$.  What are $a_i^4$, $i=s,1,2,3,4,t$?

\newpage

   \item  (20 points.)
          Consider the problem of finding the maximum cardinality matching
          in a bipartite graph.
       \begin{enumerate}
          \item  Express this problem as a linear programming problem~P.
          \item  What is the dual linear program D to~P?
          \item  How do node covers relate to feasible solutions to~D?
          \item  What are the complementary slackness conditions for the
                 dual pair of linear programming problems D and~P?
       \end{enumerate}

\newpage

   \item  (30 points.)
          Let $T$ be a finite set.
          Let $S_i\subseteq{T}$, $i=1,2,\ldots,m$, and let
          $\Sigma=\{S_1,S_2,\ldots,S_m\}$.
          A list $(f_1,f_2,\ldots,f_m)$ of distinct elements
          of $T$ which satisfy 
          $f_i{\in}S_i,i=1,\ldots,m$
          is called a {\em system of distinct representatives} (SDR)
          of $\Sigma$.

          For example:
          \begin{quotation}
             Let $T=\{a,b,c,d,e\}$.

             If $\Sigma=\{\{a,b,c\},\{a,d,e\},\{a,d,e\},\{a,e\}\}$ then
               $(c,d,e,a)$ is an SDR for $\Sigma$.

             However, if 
             $\Sigma=\{\{a,b,c\},\{a,d,e\},\{a,d,e\},\{a,e\},\{d,e\}\}$ then
             there does not exist an SDR for $\Sigma$.
          \end{quotation}

       \begin{enumerate}
          \item  Show that if there exists an SDR for $\Sigma$ then every
                 ${\Sigma}'=\{S_{i_1},\ldots,S_{i_k}\}\subseteq\Sigma$
                 must contain at least $k$
                 distinct elements of $T$, ie, the cardinality of
                 ${\cup}_{j=1}^k S_{i_j}$ is greater than or equal to $k$.
          \item  Use the max-flow/min-cut theorem to show that if every
                 ${\Sigma}'=\{S_{i_1},\ldots,S_{i_k}\}\subseteq\Sigma$
                 contains at least $k$
                 distinct elements of $T$ then there exists an SDR of
                 $\Sigma$.
                 (Hint:  Construct a bipartite graph with bipartition
                 $(W_1,W_2)$.  Let the elements of $W_1$ correspond to the
                 elements of $T$. Let the elements of $W_2$ correspond to the
                 elements of $\Sigma$.
                 Show that if there exists a flow of size $m$ then there
                 exists an SDR.
                 What can you conclude about the value of the maximum flow
                 if there does not exist an SDR?
                 If there exists a cut of size less than $m$, consider the
                 elements of $W_2$ that are not reachable from $s$ with
                 respect to the optimal flow.)
        \end{enumerate}

\newpage
  \quad
\newpage

   \item  (15 points.)
          For each of the questions below, answer either yes or no and give
          a short justification for your answer.
      \begin{enumerate}
          \item Is the flow in the following graph maximum?
                (The pair of numbers on each edge $e$ is $(x(e),c(e))$,
                where $x(e)$ is the flow on edge $e$ and $c(e)$ is the
                capacity of edge $e$.)
            \vspace{2.3in}
          \item Is the node cover in the following graph minimum?
            \vspace{2.3in}
          \item Is the matching in the following graph maximum?
                (Squiggly lines indicate matched edges.)
      \end{enumerate}

\end{enumerate}



\end{document}
