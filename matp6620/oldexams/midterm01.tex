\documentclass[12pt]{article}

\oddsidemargin -.5in
\textwidth 7.5in
\textheight 9in
\topmargin -30pt

\pagestyle{myheadings}

\markboth{COIP Midterm 2001}{COIP Midterm 2001}

\begin{document}

\setcounter{page}{0}

\begin{tabular}{cl}
  \hspace{5in} & Name:
\end{tabular}

\begin{center}
  {\large
  MATP6620/DSES6770 \\
  {\bf Combinatorial Optimization and Integer Programming } \\
  Spring 2001}
\end{center}

\begin{center}
  Midterm Exam, Thursday, March 8, 2001.
\end{center}

Please do all five problems. Show all work. No books or calculators allowed.
You may use any result from class, the homeworks, or the texts, except where
stated.
You may use one sheet of handwritten notes.
The exam lasts two hours.

\vspace{2in}

\begin{center}
\begin{tabular}{c|@{\hspace*{1in}}c}
  Q1 & \qquad \\ \hline
  Q2 & \qquad \\ \hline
  Q3 & \qquad \\ \hline
  Q4 & \qquad \\ \hline
  Q5 & \qquad \\ \hline \hline
  Total &
\end{tabular}
\end{center}

\pagebreak

\begin{enumerate}
%  \item (20 points)
%You wish to choose from a set of possible investments $\{1,\ldots,7\}$,
%subject to the following constraints:
%\begin{enumerate}
%\item You cannot invest in all of them.
%\item Investment 1 must be chosen if investment 3 is chosen.
%\item Investment 5 can be chosen only if investment 2 is also chosen.
%\item You must choose either both investments 1 and 6 or neither.
%\item You must choose either at least one of the investments 1,2,3,
%and/or at least two investments from 2,5,6,7.
%\end{enumerate}
%Model this as a 0-1 integer programming feasibility problem.
  \item (20 points)
         An instance $d$ of a feasibility problem $X \in NP$
         depends upon two positive integer parameters $m$ and $n$.
         Assume $d$ requires storage $m2^n$ in binary, and that we
         know an algorithm $A$ which solves $d$ in time $2^{m+n}$.
         \begin{enumerate}
           \item (10 points) Can we conclude $X$ is in $P$?
           \item (10 points) Assume we know in addition that $m\leq n$ for every
             instance $d$ of $X$.  What can we conclude now?
         \end{enumerate}
  \newpage
  \setcounter{page}{1}  \quad \vspace*{\fill}  \quad
  \newpage
\item (20 points)
   Using the {\em Hamiltonian path} problem, or otherwise,
           show that the following
           problem is ${\cal N}{\cal P}$-complete.
           \begin{quote}
              Given a graph $G=(V,E)$ and an integer $k$, is there a
              spanning tree $T$ of $G$ that has exactly $k$ leaves?
           \end{quote}
           (A {\em leaf} of a tree is a vertex of degree 1.
           The Hamiltonian path problem is: Given a graph $G=(V,E)$, does there
           exist a path which visits all the vertices of $G$ exactly once?
           You may assume that this problem is
           \begin{math}{\cal N}{\cal P}\end{math}-complete.)
  \newpage
  \setcounter{page}{2}  \quad \vspace*{\fill}  \quad
  \newpage
\item
  Let $S:= \{ x \in I\!\!B^3: -x_1+x_2+3x_3 \leq 2\}$.
Let $S^0:=S \cap \{ x \in I\!\!B^3: x_1=0\}$.
\begin{enumerate}
\item (5 points) Show that $x_2 \leq 1$ defines a facet of the convex
hull of~$S^0$.
\label{part_S0}
\item \label{part_lift}
(5 points) Derive a valid inequality for $S$ by lifting the
valid inequality for $S^0$ given in part~\ref{part_S0}.
\item (5 points) Show that the inequality you derived in
part~\ref{part_lift} does not define a facet of the convex
hull of~$S$.
\item (5 points) Give an inequality description of the convex
hull of~$S$.
\end{enumerate}
%   \item (20 points) Consider the set of points $S$ given by
%     \begin{displaymath}
%       \begin{array}{rrrrrrcl}
%         x_1 & + x_2 & &&&& \geq & 1 \\
%         x_1 && + x_3 & &&& \geq & 1 \\
%         & x_2& + x_3 & &&& \geq & 1 \\
%         x_1  &&& + x_4 & + x_5 && \geq & 1 \\
%         & x_2 && + x_4 && + x_6 & \geq & 1 \\
%         && x_3 && + x_5 & + x_6 & \geq & 1 \\
%                       &&&&& x_i && \mbox{binary}
%       \end{array}
%     \end{displaymath}
%     \begin{enumerate}
%       \item (10 points) Show that the convex hull of $S$, $conv(S)$,
%         has dimension 6.
%         (Hint: Note that the point $x_i=1$, $i=1,\ldots,6$, is feasible.)
%       \item (10 points) The point  
%$x^*=(\frac{2}{3},\frac{2}{3},\frac{2}{3},
%      \frac{1}{4},\frac{1}{4},\frac{1}{4})$
%         satisfies all the inequalities above.
%         Show that it is not in $conv(S)$.
%     \end{enumerate}
  \newpage
  \setcounter{page}{3}  \quad \vspace*{\fill}  \quad
  \newpage
\item (20 points)
We wish to solve the MAXCUT problem on
a graph $G=(V,E)$ with edge weights~$c_e$.
Recall that we modeled this as an integer programming problem
by introducing binary variables $x_e$ to indicate whether an edge is
in the cut.
Let $S$ be the set of incidence vectors corresponding to cuts.
These variables satisfied the constraints
\begin{equation}
\label{eqn_maxcut}
x(F)-x(C\setminus F) \leq |F|-1
\end{equation}
for any chordless cycle $C$, where $F$ is a subset of $C$ of odd cardinality.
We are going to restrict our attention to~$K_5$,
the complete graph on 5 vertices.
\begin{enumerate}
\item (5 points)
Argue from first principles that any maxcut on $K_5$ uses at most
6 of the edges of~$K_5$.
\item (10 points)
\label{part_maxcut}
Show that the point $x_e=\frac{2}{3}$ for all edges $e$
satisfies~(\ref{eqn_maxcut}) for all chordless cycles $C$ and corresponding
subsets~$F$.
Show further that this point is not in the convex hull of the set of
incidence vectors of cuts.
What constraint does this suggest?
\item
(5 points)
How would you try to show that the constraint you defined in
part~\ref{part_maxcut}
gives a facet of the convex hull of cuts?
(Note: I do not want you to show that it is a facet;
instead, I want you to tell me what points you might consider,
and what you might try to do with those points.
You may assume $S$ is full dimensional.)
\end{enumerate}
%  \item (20 points)
%    Consider the node packing problem on the graph $G=(V,E)$.
%    This can be written
%    \begin{displaymath}
%      \begin{array}{lrcll}
%        \max & \sum_{i\in V} x_i \\
%        \mbox{subject to } & x_i + x_j & \leq & 1, & (i,j) \in E \\
%            & x_i & = & 0 \mbox{ or } 1, & i \in V
%      \end{array}
%    \end{displaymath}
%    \begin{enumerate}
%      \item (10 points)
%        We saw in class that if the vertices $v_1,...,v_{2k+1}$
%        ($k\geq 1$, $k$ integer)
%        give a cycle then the {\em odd hole inequality}
%        \begin{displaymath}
%          \sum_{i=1}^{2k+1} x_i \leq k
%        \end{displaymath}
%        is valid for the node packing polytope.
%        What is the Chvatal rank of this inequality?
%      \item (10 points)
%        We saw in class that if the vertices $v_1,...,v_5$
%        form a clique then the {\em clique inequality}
%        \begin{displaymath}
%          \sum_{i=1}^5 x_i \leq 1
%        \end{displaymath}
%        is a facet of the node packing polytope.
%        Show that the Chvatal rank of this inequality is no more than 2.
%        (Hint: A clique of size 3 is also a cycle of length~3.)
%    \end{enumerate}
  \newpage
  \setcounter{page}{4}  \quad  \vspace*{\fill}   \quad
  \newpage
  \setcounter{page}{4}  \quad  \vspace*{\fill}   \quad
\newpage
\item (20 points)
  {\em Resolution} is a technique used in logic to try to solve instances
  of SATISFIABILITY. It works by combining two clauses to generate an
extra clause that must be satisfied by any solution to the instance.
For example, consider the two clauses $x_1 \vee \bar{x}_2 \vee x_3$
and $\bar{x}_2 \vee \bar{x}_3 \vee x_4$.
If $x_3$ takes the value true then either $x_2$ must be false or $x_4$ must be
true in order for the second clause to be satisfied. If $x_3$ is false
then either $x_1$ must be true or $x_2$ must be false
in order for the first clause to be satisfied. It follows that
any truth assignment satisfying the clauses $x_1 \vee \bar{x}_2 \vee x_3$
and $\bar{x}_2 \vee \bar{x}_3 \vee x_4$
must also satisfy the clause $x_1 \vee \bar{x}_2 \vee x_4$.
We say in this case that the two clauses were {\em resolved using~$x_3$.}

In general, to resolve two clauses using variable $x_i$, the clauses have
the forms:
\begin{eqnarray}
\label{eqn_preres1}
x_i & \vee & ( \bigvee_{j \in C_1^+} x_j ) \vee
 ( \bigvee_{j \in C_1^-} \bar{x}_j )
\\
\label{eqn_preres2}
\bar{x}_i
    & \vee & ( \bigvee_{j \in C_2^+} x_j ) \vee
 ( \bigvee_{j \in C_2^-} \bar{x}_j ).
\end{eqnarray}
Any solution to these two clauses must also satisfy the {\em resolved clause}
\begin{equation}
\label{eqn_resolved}
( \bigvee_{j \in C_1^+ \cup C_2^+} x_j ) \vee
( \bigvee_{j \in C_1^- \cup C_2^-} \bar{x}_j ).
\end{equation}
Note that it is not necessary that the four sets of indices
$C_1^+$, $C_1^-$, $C_2^+$, and $C_2^-$ be distinct.

In this question, we are going to investigate the representation of
SATISFIABILITY as an integer programming problem.
Thus, we introduce a 0-1 variable $y_i$ for each logical variable $x_i$
and we introduce an inequality for each clause.
The clauses can be satisfied simultaneously if and only if the
inequalities can be satisfied simultaneously.

\begin{enumerate}
\item (5 points)
Assume the four sets $C_1^+$, $C_1^-$, $C_2^+$, and $C_2^-$ are distinct.
Show that the inequality corresponding to (\ref{eqn_resolved})
is implied by the inequalities corresponding to equations (\ref{eqn_preres1})
and~(\ref{eqn_preres2}).
\item (10 points)
Now assume that $C_i^+ \cap C_j^- = \emptyset$
for any combination of $i$ and $j$,
but that either $C_1^+ \cap C_2^+ \neq \emptyset$
and/or $C_1^- \cap C_2^- \neq \emptyset$.
Show that the resolved clause can be derived using one step of the
Chvatal-Gomory rounding procedure.
\item (5 points)
Find a fractional point that satisfies the inequalities corresponding
to the clauses
$x_1 \vee \bar{x}_2 \vee x_3$
and $\bar{x}_2 \vee \bar{x}_3 \vee x_4$
but which violates the inequality corresponding to
the clause $x_1 \vee \bar{x}_2 \vee x_4$.
\end{enumerate}
  \newpage
  \setcounter{page}{5}  \quad  \vspace*{\fill}   \quad
  \newpage
  \setcounter{page}{5}  \quad  \vspace*{\fill}   \quad
\end{enumerate}

\end{document}
