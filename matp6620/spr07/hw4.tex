\documentclass[11pt]{article}

\usepackage{fullpage}

\pagestyle{empty}

%\oddsidemargin -.5in
%\textwidth 7.5in
%\textheight 10in
%\topmargin -30pt
%\headsep 0in
%\headheight 0in

\newcommand{\til}{\char '176}
\newcommand{\re}{I\!\!R}
\newcommand{\binary}{I\!\! B}


\begin{document}

\begin{center}
  \begin{large}
     MATP6620 / DSES6760 Combinatorial Optimization and Integer Programming \\
       Homework 4.
  \end{large}
\end{center}

\begin{flushright}
   Due:  Friday, March 16, 2007.  \\
Penalty for late homeworks: 10\% for each day or part of a day.
\end{flushright}

%The {\em linear ordering problem} is the subject of questions~\ref{q.linord}--\ref{q.facet}.
         Given $n$ objects,
         we want to find an {\em ordering} (or permutation) of the objects.
%         Mathematically, $\sigma$ is a 1-1 mapping from the set of integers
%         $1,\ldots,n$ onto itself, and $i$ is before $j$ in the ordering
%         if and only if $\sigma(i)<\sigma(j)$.
%%         The ordering should be {\em consistent};
%%         that is, if $i$ is before $j$ and $j$ is before $k$ then $i$ should
%%         be before~$k$.
%         We get a reward $w_{ij}$ if $i$ is somewhere before $j$
%         and a reward $w_{ji}$ if $j$ is somewhere before~$i$.
%         The objective is to maximize the total reward.
%         \begin{displaymath}
%           \begin{array}{cc}
%              \sum  & w_{ij}.  \\
%              \mbox{all pairs $i$, $j$,}  &  \\
%              \mbox{$i$ before $j$}
%           \end{array}
%         \end{displaymath}
         We can model the set of orderings by introducing
         variables $x_{ij}$ for $1 \leq i < j \leq n$  defined as follows:
         \begin{displaymath}
            x_{ij} = \left\{ \begin{array}{ll}
                        1 & \mbox{if $i$ is before $j$} \\
                        0 & \mbox{otherwise}
                     \end{array}  \right.
         \end{displaymath}
%         The resulting model is
%         \begin{displaymath}
%           \begin{array}{lccl}
%              \max & \sum_{i=1}^n & \sum_{j=1}^n  & w_{ij}x_{ij}  \\
%              && j \neq i & \\
%             \mbox{subject to} &&& x \mbox{ is the incidence vector of an ordering}
%           \end{array}
%         \end{displaymath}
%         Let $S$ be the set of points $x$ which correspond to orderings;
%         let $conv(S)$ be the convex hull of~$S$.
%        This formulation has $n(n-1)$ variables.
%        You may assume that the dimension of $conv(S)$ is~$n(n-1)/2$.
         Given two objects $i$ and $j$, either $i$ is before $j$
         or $j$ is before~$i$.
The set of feasible solutions can be written as
\begin{displaymath}
S :=
\left\{ x \in \binary^{n(n-1)/2} :
\begin{array}{rcll}
x_{ij} + x_{jk} - x_{ik} & \leq & 1 & \mbox{for }1 \leq i < j < k \leq n  \\
- x_{ij} - x_{jk} + x_{ik} & \leq & 0 & \mbox{for } 1 \leq i < j < k \leq n  \\
\end{array}   \right\}
\end{displaymath}
The inequalities are called triangle inequalities.

\begin{enumerate}
\item Show that $S$ is full dimensional.
\item Show that the triangle inequality $x_{ij} + x_{jk} - x_{ik}  \leq  1$
defines a facet of conv$(S)$ for any $1 \leq i < j < k \leq n $.
\item Let $1 \leq i < j < k < l \leq n$.
Show that the valid inequality $x_{ij} + x_{jk} + x_{kl} - x_{il} \leq 2$ is
not a facet of~conv$(S)$.
\item Solve the integer program $\min\{c^Tx : x \in S\}$ with $n=8$ using a cutting plane
algorithm, adding violated triangle inequalities as cutting planes.
The objective function coefficients $c$ are as follows:
\begin{displaymath}
\begin{array}{l|rrrrrrrr}
    & 1 & 2 & 3 & 4 & 5 & 6 & 7 & 8  \\  \hline
1  & * &-3 & 2 &-4 & 1 & 3 &-1 &-2  \\
2  & * & * &-3 & 2 & 4 &-1 &-5 & 2  \\
3  & * & * & * & 1 &-4 & 5 & 1 &-3  \\
4  & * & * & * & * &-3 & 1 & 2 &-3  \\
5  & * & * & * & * & * & 3 &-1 &-1  \\
6  & * & * & * & * & * & * &-1 & 4  \\
7  & * & * & * & * & * & * & * & 5  \\
8  & * & * & * & * & * & * & * & *
\end{array}
\end{displaymath}
What is the optimal ordering?
(Hint: the optimal value is $-22$.)
%   \item  Consider the 0-1 equality knapsack problem
%          \begin{displaymath}
%             \max \{-x_{n+1} : 2x_1 + 2x_2 + \ldots + 2x_n + x_{n+1} = n,
%                         x \in B^{n+1} \},
%          \end{displaymath}
%          where $n$ is an odd integer.
%          We wish to solve this problem using a branch and bound algorithm
%          with linear programming relaxations.
%          Assume
%          we always branch using variable dichotomies
%          (ie, add the constraints $x_i = 0$ or $x_i = 1$ for some $i$).
%          Show that an exponential number of nodes of the
%          branch and bound tree must be considered to solve the problem.
%%          What is the minimum number of nodes we need if we can separate
%%          using any linear inequality?

%   \item  Use branch-and-bound with LP relaxations to solve the knapsack problem
%          \begin{displaymath}
%           \begin{array}{lrrrrl}
%            \max          &  20x_1 & + 30x_2 & + 40x_3 & + 42x_4  \\
%            \mbox{s.t.}   &    x_1 & +  3x_2 & +  5x_3 & +  6x_4 & \leq 7 \\
%                          &&&&     x_i & \mbox{binary,}
%           \end{array}
%          \end{displaymath}
%          branching using variable dichotomies.
%          Give a valid cover inequality violated by your solution to the root node.
%          What do you get if you lift this inequality?

%   \item \label{q.linord}
%         \begin{enumerate}
%           \item Show that $x_{ij}+x_{ji}=1$ is valid for every ordering. (Here, $i \neq j$.)
%              \label{valideq}
%%           \item Show that the dimension of $conv(S)$ is~$n(n-1)/2$.
%%              (Hint: This means that only linear combinations of the
%%              equalities given in part~\ref{valideq} are valid throughout~$S$.
%%              Assume the contrary and derive a contradiction.)
%           \item Show that the inequality $x_{ij}+x_{jk}+x_{ki}\leq 2$
%              is valid for $S$ for any $i$, $j$, and~$k$.   \label{dicycle}
%              (Here, $i \neq j \neq k \neq i$.)
%        \end{enumerate}
%%           \item By lifting, or otherwise, show that the inequality
%%              $x_{ij}+x_{jk}+x_{ki}\leq 2$ is a facet of $conv(S)$ for any
%%              $i$, $j$, and~$k$.
%           \item %This question concerns the linear ordering problem of question~\ref{q.linord}.
%           Let $U=\{u_1,\ldots,u_k\}$ and $W=\{w_1,\ldots,w_k\}$
%              be two subsets of the objects. Show that the inequality
%              \begin{displaymath}
%                \sum_{u_i \in U, w_j \in W, i \neq j} x_{w_j,u_i}
%                   + \sum_{i=1}^k x_{u_i,w_i} \leq k^2-k+1
%              \end{displaymath}
%              is valid for~$S$. (Note that the inequality includes $k$ edges pointing
%up in the picture and $k(k-1)=k^2-k$ edges pointing down.)
% \label{fence.ineq}

%\begin{center}
%\begin{picture}(340,200)
%\put(20,50){\circle*{5}}
%\put(20,150){\circle*{5}}
%\put(120,50){\circle*{5}}
%\put(120,150){\circle*{5}}
%\put(220,50){\circle*{5}}
%\put(220,150){\circle*{5}}
%\put(320,50){\circle*{5}}
%\put(320,150){\circle*{5}}

%\put(20,35){$u_1$}
%\put(120,35){$u_2$}
%\put(220,35){$u_3$}
%\put(320,35){$u_4$}
%\put(20,160){$w_1$}
%\put(120,160){$w_2$}
%\put(220,160){$w_3$}
%\put(320,160){$w_4$}

%\put(20,50){\vector(0,1){97}}
%\put(120,50){\vector(0,1){97}}
%\put(220,50){\vector(0,1){97}}
%\put(320,50){\vector(0,1){97}}

%\put(20,150){\vector(1,-1){98}}
%\put(20,150){\vector(2,-1){198}}
%\put(20,150){\vector(3,-1){298}}

%\put(120,150){\vector(1,-1){98}}
%\put(120,150){\vector(2,-1){198}}
%\put(120,150){\vector(-1,-1){98}}

%\put(220,150){\vector(1,-1){98}}
%\put(220,150){\vector(-2,-1){198}}
%\put(220,150){\vector(-1,-1){98}}

%\put(320,150){\vector(-1,-1){98}}
%\put(320,150){\vector(-2,-1){198}}
%\put(320,150){\vector(-3,-1){298}}
%\end{picture}
%\end{center}

%           \item Find a point $\bar{x}$, $0\leq \bar{x}_i \leq 1$,
%              which satisfies all the equalities in part~\ref{valideq}
%              and all the inequalities in part~\ref{dicycle},
%              but which violates the inequality in question~\ref{fence.ineq}.
%             (Hint: such a point exists when $k=3$.)

%\item  \label{q.facet}
%%Consider again the linear ordering problem of Question~\ref{q.linord}.
%We can use part~\ref{valideq} to eliminate half the variables,
%namely those $x_{ij}$ with $j<i$.
%The resulting polytope is full-dimensional.
%The valid inequality from part~\ref{dicycle} for objects 1, 2, and 3 is then
%\begin{displaymath}
%              x_{12}+x_{23}-x_{13}\leq 1.
%\end{displaymath}
%Show that this defines a facet of this full-dimensional polytope.

%Given the complete graph $K_n=:(V,E)$ on $n$ vertices,
%a {\bf clustering}
%of the vertices is obtained by choosing an integer $p$
%and a partition of the vertices into $p$ sets $V_1,\ldots,V_p$ satisfying:
%\begin{itemize}
% \item $V_s \cap V_t = \emptyset$ for $1\leq s < t \leq p$.
% \item $\cup_{s=1}^p V_s=V$.
%\end{itemize}
%Note that $p$ is {\em not fixed}.
%The {\em incidence vector} of this clustering is defined by
%\begin{displaymath}
%  x_e = \left\{ \begin{array}{ll} 1 & \mbox{if the two endpoints of $e$
%               are in the same set $V_s$}  \\  0 & \mbox{otherwise.}
%        \end{array}  \right.
%\end{displaymath}
%Let edge $e$ have weight~$w_e$.
%The {\bf clustering problem} for this set of edge weights is then
%\begin{displaymath}
%\begin{array}{ll}
% \max & z:=\sum_{e\in E}w_ex_e \\
%  \mbox{subject to } & x \mbox{ is the incidence vector of a}
%                         \mbox{ clustering}
%\end{array}
%\end{displaymath}
%The edge weights $w_e$ can be positive or negative.
%The weights $w_e$ measure the ``similarity'' between two vertices:
%the larger this value, the more likely the vertices should be
%in the same subset~$V_s$, and if $w_e$ is very negative,
%the vertices will probably be in different sets in the optimal
%solution.

%Let $Q$ be the set of incidence vectors of clusterings
%for $K_n$.
%\begin{enumerate}
% \item
%Show that if
%all the edge weights are positive, the optimal solution
%is to put all the vertices in one set,
%giving a value $z=\sum_{e \in E}w_e$.
%Show that if all the edge weights are negative, the optimal solution is
%to put each vertex in its own set, giving a value $z=0$.
% \item Show that the dimension of $Q$ is $n(n-1)/2$, that is, $Q$
%is full dimensional.
% \item Show that the inequality
%\label{q_ineq}
%\begin{equation}  \label{eqn.cluster}
%  x_{ij} + x_{jk} - x_{ik}  \leq  1, \;\;  1 \leq i<j<k \leq n
%\end{equation}
%is valid for any point in Q.
% \item Show that inequality $(\ref{eqn.cluster})$ defines
%a facet of the convex hull of~$Q$.
% \item \label{part.k4}
%%Take $n=4$.
%%Choose edge weights $w_e$ so that
%%the optimal solution to the following relaxation of the
%%clustering problem is not in the convex hull of~$Q$:
%%One LP relaxation of the clustering problem is the following:
%%\begin{displaymath}
%% \begin{array}{lccll}
%%  \max & \sum_{e \in E} w_e x_e \\
%%  \mbox{subject to} &
%It follows from question~\ref{q_ineq} that any point in~$Q$ satisfies the
%{\em triangle inequalities:}
%\begin{displaymath}
%\begin{array}{ccll}
% x_{ij}+x_{jk}-x_{ik} & \leq & 1 & \mbox{ for }
%      1 \leq i < j < k \leq n \\
% x_{ij}-x_{jk}+x_{ik} & \leq & 1 & \mbox{ for }
%      1 \leq i < j < k \leq n \\
% -x_{ij}+x_{jk}+x_{ik} & \leq & 1 & \mbox{ for }
%      1 \leq i < j < k \leq n.
% \end{array}
%\end{displaymath}
%%\item
%The point $\bar{x}$ given by
%\begin{displaymath}
%\bar{x}_e = \left\{ \begin{array}{ll}
%\frac{1}{2} & \mbox{if }
%e=(1,2), (1,3), (1,4) \\
%0 & \mbox{otherwise}
%\end{array}  \right.
%\end{displaymath}
%satisfies these constraints.
%Find a valid inequality that cuts off the point~$\bar{x}$.
% \item
%Solve the five vertex
%instance of the clustering problem given in the AMPL files
%\begin{quote}
%http://www.math.rpi.edu/\til mitchj/matp6620/hw4/hw4q.mod
%\end{quote}
%and
%\begin{quote}
%http://www.rpi.edu/\til mitchj/matp6620/hw4/hw4q.dat
%\end{quote}
%using a cutting plane approach, where the only cuts added are violated triangle
%inequalities as in question~\ref{part.k4}.
%Note that the initial relaxation contains just simple bound constraints.

%

\end{enumerate}

\vfill

\begin{tabular}{@{\hspace{.5in}}l}
   John Mitchell  \\
   Amos Eaton 325  \\
   x6915.  \\
   mitchj@rpi.edu  \\
   Office hours:
   Tuesday 2pm -- 3pm, Wednesday 11am -- noon.
\end{tabular}


\end{document}
