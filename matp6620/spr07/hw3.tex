\documentclass[12pt]{article}

\usepackage{fullpage}

\pagestyle{empty}

%\oddsidemargin -.5in
%\textwidth 7.5in
%\textheight 10in
%\topmargin -30pt
%\headsep 0in
%\headheight 0in

\newcommand{\til}{\char '176}
\newcommand{\re}{I\!\!R}


\begin{document}

\begin{center}
  \begin{large}
     MATP6620 / DSES6760 Combinatorial Optimization and Integer Programming \\
       Homework 3.
  \end{large}
\end{center}

\begin{flushright}
   Due:  Friday, March 2, 2007.  \\
Penalty for late homeworks: 10\% for each day or part of a day.
\end{flushright}

%\vspace{\baselineskip}

%The MAXCUT problem is the subject of questions
%\ref{q.ineq}, \ref{q.4cycle}, and \ref{q.comp}.
%It is defined on a graph $G=(V,E)$
%with edge weights~$c_e$.
%The vertices are partitioned into two sets with the objective of
%maximizing the total weight of the edges with one endpoint in each set.
%This can be modeled as an integer programming problem
%by introducing binary variables $x_e$ to indicate whether an edge is
%in the cut.

Each of the three questions on this homework is worth 20 points.

The following graph $G=(V,E)$ is used in questions \ref{q.nodepack} and~\ref{q.matching}.

\begin{center}
\begin{picture}(340,180)(-10,-10)
\put(0,0){\circle{20}}
\put(-4,-4){A}
\put(0,160){\circle{20}}
\put(-4,156){B}
\put(320,0){\circle{20}}
\put(316,-4){E}
\put(320,160){\circle{20}}
\put(316,156){D}
\put(160,0){\circle{20}}
\put(156,-4){F}
\put(160,160){\circle{20}}
\put(156,156){C}
\put(80,80){\circle{20}}
\put(76,76){G}
\put(160,80){\circle{20}}
\put(156,76){H}
\put(240,80){\circle{20}}
\put(236,76){J}

\put(0,10){\line(0,1){140}}
\put(320,10){\line(0,1){140}}
\put(10,0){\line(1,0){140}}
\put(170,0){\line(1,0){140}}
\put(10,160){\line(1,0){140}}
\put(170,160){\line(1,0){140}}
\put(90,80){\line(1,0){60}}
\put(170,80){\line(1,0){60}}
\put(160,10){\line(0,1){60}}
\put(160,90){\line(0,1){60}}

\put(7,7){\line(1,1){66}}
\put(313,7){\line(-1,1){66}}
\put(313,153){\line(-1,-1){66}}
\put(153,153){\line(-1,-1){66}}
\put(167,153){\line(1,-1){66}}
\put(87,73){\line(1,-1){66}}
\end{picture}
\end{center}


\begin{enumerate}
\item  \label{q.nodepack}
Consider a node packing problem on the above graph, with each vertex having weight~1.
The LP relaxation includes the clique constraints $\sum_{v \in C} x_v \leq 1$
for each maximal clique~$C$ in the graph.
The point $x_A=0.4$, $x_B=0.6$, $x_C=0.4$, $x_D=0.5$, $x_E=0.4$,
$x_F=0.5$, $x_G=0.1$, $x_H=0.4$, and $x_J=0.1$
is feasible in the LP relaxation.
\begin{enumerate}
\item Show that the given point is not in the convex hull of feasible solutions,
by giving a valid constraint that is violated by this point.
\item Find an optimal solution to the node packing problem for this graph.
Prove your solution is optimal.
\end{enumerate}

\item \label{q.matching}
Consider a matching problem on the graph above.
The LP relaxation consists of nonnegativity together with the constraints
$\sum_{e \in \delta(v)}x_e \leq 1$ for each vertex~$v$.
A feasible solution to the LP relaxation is to take
$x_{AB}=0.5$, $x_{AF}=0$, $x_{AG}=0.5$, $x_{BC}=0.5$, $x_{CD}=0$, $x_{CG}=0$,
$x_{CH}=0.4$, $x_{CJ}=0.1$, $x_{DE}=0.2$, $x_{DJ}=0.8$,
$x_{EF}=0.8$, $x_{EJ}=0$, $x_{FH}=0.2$, $x_{GH}=0.4$, and $x_{HJ}=0$.
\begin{enumerate}
\item Find a valid constraint for the matching problem that is violated by this solution.
\item Assume the six diagonal edges each have weight two and the remaining 10 edges each
have weight one.
What is the maximum weight matching?
Prove your solution is optimal.
\end{enumerate}

\item Consider the integer programming problem
\begin{displaymath}
\begin{array}{lrcrcl}
\max & x_1 & + & 4x_2  \\
\mbox{subject to} & 2x_1 & + & 2x_2 & \geq & 3 \\
& -x_1 & + & 2x_2 & \leq & 5  \\
& 3x_1 & - & x_2 & \leq & 3  \\
& \multicolumn{3}{r}{x_1, x_2} & \geq & 0, \mbox{ integer} 
\end{array}
\end{displaymath}
Let $S$ be the set of feasible solutions to this problem.
\begin{enumerate}
\item Give an inequality description of the convex hull of~$S$.
\item What is the Chvatal rank of each of the facet-defining inequalities
of the convex hull of~$S$?
\item Solve the LP relaxation of this problem.
What is the Gomory cutting plane given by the objective function?
What is the strong Gomory cutting plane given by the objective function,
using the Gomory mixed integer cut?
Express the constraints in terms of the original variables~$x_1$ and $x_2$.
(You may want to use AMPL to solve the LP relaxation.
You can then get the reduced costs directly.)
\end{enumerate}

%  \item
%         An instance $d$ of a feasibility problem $X \in NP$
%         depends upon two positive integer parameters $m$ and $n$.
%         Assume $d$ requires storage $m2^n$ in binary, and that we
%         know an algorithm $A$ which solves $d$ in time $2^{m+n}$.
%         \begin{enumerate}
%           \item Can we conclude $X$ is in $P$?
%           \item Assume we know in addition that $m\leq n$ for every
%             instance $d$ of $X$.  What can we conclude now?
%         \end{enumerate}
%  \item
%  Consider the constraints
%  \begin{displaymath}
%  \begin{array}{rcl}
%  t_1 & \geq & | x_1 - x_2 |  \\
%  t_2 & \geq & | x_1+x_2 - 1|  \\
%  t_1,t_2 && \mbox{integer}  \\
%  x_1,x_2&& \mbox{binary}
%  \end{array}
%  \end{displaymath}
%  \begin{enumerate}
%  \item By considering the different possibilities for $x$, show that $t_1+t_2 \geq 1$.
%  \item The constraints can be modeled equivalently as
%  \begin{displaymath}
%  \begin{array}{rcrcrcc}
%  t_1 & \geq &  x_1 & - & x_2   \\
%  t_1 & \geq &  -x_1 & + & x_2   \\
%  t_2 & \geq & x_1 & + & x_2 & - &  1  \\
%  t_2 & \geq & -x_1 & - & x_2 & + & 1  \\
%  t_1,t_2 && \multicolumn{5}{l}{\mbox{integer}}  \\
%  x_1,x_2&& \multicolumn{5}{l}{\mbox{binary}}
%  \end{array}
%  \end{displaymath}
%Show that the valid constraint $t_1+t_2 \geq 1$ has Chvatal rank equal to 2.
%  \end{enumerate}
 
%   \item
%           Let $P=\{x\in {\re}_+^2: x_1-x_2\leq 1, 4x_1 + x_2 \leq 28,
%                      x_1 + 4x_2 \leq 27 \}$ and $S=P \cap Z_+^2$.
%           \begin{enumerate}
%              \item  Find an inequality description of conv($S$).
%              \item  Find the extreme points of conv($S$).
%              \item  Derive each of the facets of conv($S$) as a
%                C-G inequality.
%              \item  What is the Chvatal rank of the facets of conv($S$)?
%                (Hint: To get a lower bound for the difficult one,
%                ask yourself {\em ``What is the best inequality of the form
%                $x_1+2x_2\leq \pi_0$ which is valid for $P$?''}
%                To get an upper bound, consider
%                $x_1+\alpha x_2 \leq \alpha_0$ for various values
%                of~$\alpha$.)
%           \end{enumerate}
%   \item
%The MAXCUT problem is defined on a graph $G=(V,E)$
%with edge weights~$c_e$.
%The vertices are partitioned into two sets with the objective of
%maximizing the total weight of the edges with one endpoint in each set.
%This can be modeled as an integer programming problem
%by introducing binary variables $x_e$ to indicate whether an edge is
%in the cut.
%Assume $G$ is a complete graph.
%Let $i$ and $j$ be two vertices.
%Show that if $x_{ij}=1$ then we must have $x_{ik}+x_{jk}=1$ for
%any other vertex $k\in V$.
%  \item \label{q.ineq}
%  Let $C$ denote the edges of a particular cycle in a graph G=(V,E).
%  It is a standard result in graph theory that the intersection of
%   a cycle and a cut consists of an even number of edges.
%   \begin{enumerate}
%   \item \label{q.ineq.1edge}
%   Let $e$ be an edge of the cycle~$C$.
%   Construct a linear constraint on the variable $x$ used in the MAXCUT
%   problem to prevent the infeasible solution corresponding to  using only
%   the edge $e$ from~$C$.
%   (Hint: look at the difference between $x_e$ and $x$ for the other edges in~$C$.)
%   \item Let $e_1$, $e_2$, and $e_3$ be three edges from the cycle~$C$.
%   Construct a linear constraint on the variable $x$ used in the MAXCUT
%   problem to prevent the infeasible solution corresponding to  using only
%   edges $e_1$, $e_2$, and $e_3$ from~$C$.
%   \end{enumerate}
%   \item \label{q.4cycle}
%   Let $G=(V,E)$ be a complete graph.
%   Let $C$ be a cycle of length 4 and let $e$ be an edge of~$C$.
%   Show that your inequality for $C$ from question~\ref{q.ineq.1edge}
%   is implied by two inequalities corresponding to cycles of length~3.
%   \item
%           Let $P=\{x\in {\re}_+^2: x_1-x_2\leq 1, 4x_1 + x_2 \leq 28,
%                      x_1 + 4x_2 \leq 27 \}$ and $S=P \cap Z_+^2$.
%          Consider the integer program
%          $\max \{ 2x_1 + 3x_2: x \in S \}$.  %, where $S$ is given in
%          %question~\ref{question.S}.
%          Solve this problem by using the following two-step procedure:
%          \begin{itemize}
%            \item Solve the LP relaxation
%            \item If necessary, add one of the inequalities from the polyhedral
%              description of the convex hull of the feasible region
%          \end{itemize}
%          and repeating.
%          (You may want to use AMPL or a similar package to solve
%          the LP relaxations.
%          See the course webpage for details on how to use AMPL.
%          If you need more help, see me.
%          The initial AMPL file is
%          in http://www.rpi.edu/\til mitchj/matp6620/hw3/hw3q4.mod0.)

%{\flushright{/ over}}

%\pagebreak

%\item \label{q.comp}
%Let $G=(V,E)$ be a complete graph with five vertices.
%The edge lengths $c_e$ for this graph are in Table~\ref{table.lengths}.
%\begin{table}
%\begin{center}
%\begin{tabular}{l|ccccc}
%& $v_1$ & $v_2$ & $v_3$ & $v_4$ & $v_5$ \\ \hline
%$v_1$ & --- &  9  &  1 &  2 &  8\\
%$v_2$ & --- & --- & 8  & 3  & 2 \\
%$v_3$ & --- & --- & --- & 7 & 1  \\
%$v_4$ & --- & --- & --- & --- &  9 \\
%$v_5$ & --- & --- & --- & --- & ---
%\end{tabular}
%\end{center}
%\caption{Edge lengths for Question~\ref{q.comp}
%\label{table.lengths}}
%\end{table}
%Take the unit box as the initial LP relaxation.
%          Solve this problem by using the following two-step procedure:
%          \begin{itemize}
%            \item Solve the LP relaxation
%            \item If necessary, add one or more of the inequalities from question~\ref{q.ineq}
%            corresponding to cycles of length~3 that are violated by the solution to the LP
%            relaxation.
%          \end{itemize}
%          and repeating.

%          (You may want to use AMPL or a similar package to solve
%          the LP relaxations.
%          See the course webpage for details on how to use AMPL.
%          If you need more help, see me.
%          The initial AMPL model and data file are
%          in
%          \begin{quote}
%          http://www.rpi.edu/\til mitchj/matp6620/hw3/hw3q4.mod  \\
%          and  \\
%          http://www.rpi.edu/\til mitchj/matp6620/hw3/hw3q4.dat
%          \end{quote}
%          respectively.)

%   \item  Consider the maximum weight matching problem on a graph $G=(V,E)$,
%          where all edge weights are nonnegative.
%          One possible separation routine is to search for connected
%          components in the graph $G=(V,E')$, where $E'$ consists of all
%          edges in $E$ with $x_e>0$ in the
%          optimal solution to the LP-relaxation,
%          and see if those components violate the odd set constraints.
%          Use this in a cutting plane algorithm to solve the
%          maximum weight matching problem on the graph below.
%          Edge $e_i$ has weight~$i$.

%%\begin{picture}(540,90)(270,0)
%\begin{picture}(540,90)(40,0)
%  \put(20,20){\line(1,0){100}}
%  \put(20,20){\line(0,1){60}}
%  \put(120,20){\line(0,1){60}}
%  \put(20,80){\line(1,0){100}}
%  \put(120,20){\line(3,1){90}}
%  \put(120,80){\line(3,-1){90}}
%  \put(120,20){\line(1,0){280}}
%  \put(210,50){\line(1,0){100}}
%  \put(310,50){\line(3,1){90}}
%  \put(310,50){\line(3,-1){90}}
%  \put(400,20){\line(1,0){100}}
%  \put(400,20){\line(0,1){60}}
%  \put(500,20){\line(0,1){60}}
%  \put(400,80){\line(1,0){100}}

%

%  \put(20,20){\circle*{5}}
%  \put(20,80){\circle*{5}}
%  \put(120,20){\circle*{5}}
%  \put(120,80){\circle*{5}}
%  \put(210,50){\circle*{5}}
%  \put(310,50){\circle*{5}}
%  \put(400,20){\circle*{5}}
%  \put(400,80){\circle*{5}}
%  \put(500,20){\circle*{5}}
%  \put(500,80){\circle*{5}}

%
%  \put(260,55){1}
%  \put(260,25){2}
%  \put(490,47){3}
%  \put(65,25){4}
%  \put(25,47){5}
%  \put(450,25){6}
%  \put(450,70){7}
%  \put(65,70){8}
%  \put(110,47){9}
%  \put(350,37){10}
%  \put(405,47){11}
%  \put(155,37){12}
%  \put(345,70){13}
%  \put(165,70){14}

%  \put(10,75){1}
%  \put(10,15){2}
%  \put(110,25){3}
%  \put(110,70){4}
%  \put(210,55){5}
%  \put(300,55){6}
%  \put(405,70){7}
%  \put(505,75){8}
%  \put(505,15){9}
%  \put(405,25){10}
%\end{picture}
%          (Again, you may want to use AMPL to solve the LP relaxations.
%          The initial AMPL model file is
%          in http://www.rpi.edu/\til mitchj/matp6620/hw3/hw3q5.mod0
%          and the data file is in  \\
%          http://www.rpi.edu/\til mitchj/matp6620/hw3/hw3q5.dat0)

\end{enumerate}

\vfill

\begin{tabular}{@{\hspace{.5in}}l}
   John Mitchell  \\
   Amos Eaton 325  \\
   x6915.  \\
   mitchj@rpi.edu  \\
   Office hours:
   Tuesday 2pm -- 3pm, Wednesday 11am -- noon.
\end{tabular}


\end{document}
