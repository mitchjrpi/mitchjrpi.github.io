\documentclass[12pt]{article}

\pagestyle{empty}

%\oddsidemargin -.1in
%\textwidth 6.5in
%\textheight 8.9in
%\topmargin -30pt
%\headsep 0in
%\headheight 0in

\usepackage{fullpage}

\pagestyle{empty}

\newcommand{\til}{\char '176}
\newcommand{\re}{I\!\!R}


\begin{document}

\begin{center}
  \begin{large}
     MATP6620 / DSES6760 \\
Combinatorial Optimization and Integer Programming \\
       Homework 5.
  \end{large}
\end{center}

\begin{flushright}
   Due:  Friday, March 30, 2007.
\end{flushright}

%\vspace{\baselineskip}


\begin{enumerate}
\item Nemhauser and Wolsey, page 381, question 3.  \label{q.381.3}
\item Find a valid cover inequality that is violated by the optimal solution to the initial LP relaxation
for the problem in Question~\ref{q.381.3}.
What do you get when you lift this inequality?
What is the optimal solution when you add your lifted cover inequality to the LP relaxation?
\item Nemhauser and Wolsey, page 381, question 8.

%\item Nemhauser and Wolsey, page 346, question 14, parts (i), (ii), (iii), and (v).
%\item Nemhauser and Wolsey, page 346, question 15.
\item Nemhauser and Wolsey, page 346, question 13.
\item Nemhauser and Wolsey, page 347, question 16.
(Consider the criteria listed in question 15 on page 346.)

\item Give a progress report on your project. Your report should be at least a couple of
paragraphs long. It doesn't need to repeat anything from your initial report from February~23.

%   \item  Consider the 0-1 equality knapsack problem
%          \begin{displaymath}
%             \max \{-x_{n+1} : 2x_1 + 2x_2 + \ldots + 2x_n + x_{n+1} = n,
%                         x \in B^{n+1} \},
%          \end{displaymath}
%          where $n$ is an odd integer.
%          We wish to solve this problem using a branch and bound algorithm
%          with linear programming relaxations.
%          Assume
%          we always branch using variable dichotomies
%          (ie, add the constraints $x_i = 0$ or $x_i = 1$ for some $i$).
%          Show that an exponential number of nodes of the
%          branch and bound tree must be considered to solve the problem.
%%          What is the minimum number of nodes we need if we can separate
%%          using any linear inequality?

%   \item  Use branch-and-bound with LP relaxations to solve the knapsack problem
%          \begin{displaymath}
%           \begin{array}{lrrrrl}
%            \max          &  20x_1 & + 30x_2 & + 40x_3 & + 42x_4  \\
%            \mbox{s.t.}   &    x_1 & +  3x_2 & +  5x_3 & +  6x_4 & \leq 7 \\
%                          &&&&     x_i & \mbox{binary,}
%           \end{array}
%          \end{displaymath}
%          branching using variable dichotomies.
%          Give a valid cover inequality violated by your solution to the root node.
%          What do you get if you lift this inequality?

%\item
%   Consider the integer programming problem
%     \begin{displaymath}
%      \begin{array}{lrcrcrcrcrcl}
%       \min & -x_1 & + & 2x_2 & + & 3x_3 &
%                   - & 10x_4 & + & 7x_5 \\
%       \mbox{subject to} & 2x_1 & + & x_2 &&& + & 5x_4 & - & 3x_5 & = & 1 \\
%                         & -2x_1 &&& + & x_3 &- & 6x_4 & + & 4x_5 & = & 1 \\
%                     &&&&&&&   x  & \geq & 0, & \mbox{and} & \mbox{integer}
%       \end{array}
%     \end{displaymath}
%One feasible integral solution is $x=(0,1,1,0,0)$, which has value~5.
%The linear programming relaxation of this problem is obtained by ignoring
%the condition that the variables $x$ should be integral.
%The optimal tableau for the LP relaxation is
%\begin{displaymath}
%\begin{tabular}{|rrrrr|r|}
%\hline
%2 & 1 & 0.5 & 0 & 0 & -3.5 \\  \hline
%1 & 2 & 1.5 & 1 & 0 & 3.5 \\
%1 & 3 & 2.5 & 0 & 1 & 5.5 \\ \hline
%\end{tabular}
%\end{displaymath}
%\begin{enumerate}
%% \item What is the dual problem to the LP relaxation?
% \item Using the optimal tableau,
% show that any feasible solution to the LP relaxation with
%$x_1 \geq 1$ has value at least 5.5.
%%(Hint: There are many ways to show this.
%%For example, you could add the constraint $x_1 \geq 1$ to the
%%given tableau and reoptimize using the dual simplex method.)
% \item What can you say about the value of $x_1$ in any optimal solution?
%\end{enumerate}

%\item  \label{sdpclique}
%Given a graph $G=(V,E)$, let $n=|V|$ be the number of nodes.
%We can set up a semidefinite programming problem
%with the $n \times n$ symmetric matrix $X$ of variables:
%\begin{displaymath}
%\begin{array}{lrcllr}
%\max & \sum_{i=1}^n \sum_{j=1}^n X_{ij} &&\qquad \\
%\mbox{subject to} & X_{ij} & = & 0 & \mbox{if } (i,j) \not\in E  & \qquad (P) \\
%& \sum_{i=1}^n X_{ii} & = & 1  \\
%& X & \succeq & 0
%\end{array}
%\end{displaymath}
%Show that the optimal value of this SDP gives an upper bound on the
%maximum cardinality of a clique in~$G$.

%\item
%Problem $(P)$ in question~\ref{sdpclique} has a dual:
%\begin{displaymath}
%\begin{array}{lrcrcrcllr}
%\min &&& z \\
%\mbox{subject to} & -S_{ii} & + & z &&& = & 1 & i=1,\ldots,n & \qquad (D)  \\
%& -S_{ij} &&& + & y_{ij} & = & 1 &   \mbox{if } (i,j) \not\in E  \\
%& -S_{ij} &&&&& = & 1 & \mbox{if } (i,j) \in E \\
%& S &&&&& \succeq & 0
%\end{array}
%\end{displaymath}
%Here, $S$ is constrained to be a symmetric positive semidefinite $n \times n$
%matrix,
%$z$ is a scalar, and each $y_{ij}$ for $(i,j) \not \in E$ is a scalar with
%$y_{ij}=y_{ji}$.
%Show that the optimal value of the dual problem gives a lower bound
%on the minimum number of colours needed to colour the nodes of the
%graph so that no adjacent nodes receive the same colour.
%(Hint: An $n \times n$ symmetric matrix~$M$ is diagonally dominant if
%$M_{ii} \geq \sum_{j \neq i, j=1}^n |M_{ij}|$ for each row~$i$.
%A diagonally dominant matrix is positive semidefinite.)



%   \item
%In the examples below, $x$ is required to be a nonnegative vector
%in~$\re^3$ and $y$ is constrained to be a binary vector with three
%components.
%A set of constraints that must be satisfied by $x$ and $y$ is given,
%and a point that satisfies these constraints is also given.
%Find a valid flow cover inequality that cuts off this point.
%\begin{enumerate}
%\item
%Constraints:
%\begin{displaymath}
%x_1+x_2+x_3 \leq 7, \; x_1 \leq 3y_1, \; x_2 \leq 5y_2, \; x_3 \leq 6y_3.
%\end{displaymath}
%Point:
%\begin{displaymath}x=(2,5,0), \; y=(\frac{2}{3},1,0).
%\end{displaymath}
%\item
%Constraints:
%\begin{displaymath}
%7 \leq x_1+x_2+x_3, \; x_1 \leq 3y_1, \; x_2 \leq 5y_2, \; x_3 \leq 6y_3.
%\end{displaymath}
%Point:
%\begin{displaymath}x=(2,5,0), \; y=(\frac{2}{3},1,0).
%\end{displaymath}
%\end{enumerate}

%   \item
%         Interpret the numbers on the edges in the graph
%         below as both edge numbers and edge weights.
%         We are looking for a minimum weight spanning tree,
%         subject to the constraints
%         \begin{displaymath}
%           x_1 \leq x_6, \quad x_2 \leq x_7, \quad x_3 \leq x_8,
%         \end{displaymath}
%         where $x_i=1$ if edge $e_i$ is in the tree, and 0 otherwise.
%         Use tabu search to find the minimum weight spanning tree subject
%         to these constraints. The move should be from tree to tree,
%         with one edge entering and one leaving.
%         Use a penalty of 50 for each violated constraint.
%         The tabu restrictions are that added edges can not leave for
%         one iteration, and dropped edges can not enter for one iteration.
%         The initial tree should be the minimum weight spanning tree,
%         without considering the constraints.
%         (You should not need to use aspiration criteria.
%         You may assume that once you have visited two local minima,
%         one of them is the optimal solution.)

%\begin{picture}(420,140)(-80,0)
%  \put(20,20){\line(1,0){100}}
%  \put(20,20){\line(1,1){100}}
%  \put(120,20){\line(0,1){100}}
%  \put(120,120){\line(1,0){100}}
%  \put(120,20){\line(1,0){100}}
%  \put(220,20){\line(0,1){100}}
%  \put(220,20){\line(1,1){100}}
%  \put(220,120){\line(1,0){100}}

%
%  \put(20,20){\circle*{3}}
%  \put(120,20){\circle*{3}}
%  \put(120,120){\circle*{3}}
%  \put(220,20){\circle*{3}}
%  \put(220,120){\circle*{3}}
%  \put(320,120){\circle*{3}}

%
%  \put(55,70){1}
%  \put(70,10){8}
%  \put(125,70){4}
%  \put(170,10){5}
%  \put(170,125){3}
%  \put(210,70){6}
%  \put(270,125){7}
%  \put(270,60){2}
%\end{picture}

\end{enumerate}

\vfill

\begin{tabular}{@{\hspace{.5in}}l}
   John Mitchell  \\
   Amos Eaton 325  \\
   x6915.  \\
   mitchj@rpi.edu  \\
   Office hours:
   Tuesday 2pm -- 3pm, Wednesday 11am -- noon.
\end{tabular}


\end{document}
