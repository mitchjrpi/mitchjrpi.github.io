\documentclass[11pt]{article}

\pagestyle{empty}

%\oddsidemargin -.1in
%\textwidth 6.5in
%\textheight 8.9in
%\topmargin -30pt
%\headsep 0in
%\headheight 0in

\usepackage{fullpage}

\pagestyle{empty}

\newcommand{\til}{\char '176}
\newcommand{\re}{I\!\!R}


\begin{document}

\begin{center}
  \begin{large}
     MATP6620 / DSES6760 \\
Combinatorial Optimization and Integer Programming \\
       Homework 6.
  \end{large}
\end{center}

\begin{flushright}
   Due:  Friday, April 20, 2007.
\end{flushright}

%\vspace{\baselineskip}


\begin{enumerate}

\item
Given a complete graph $G=(V,E)$ with edge weights $w_e$ and $n=kp$ nodes,
we wish to partition the vertices into
$k$ sets each containing exactly $p$ vertices so as to minimize the sum of the edge weights of the edges with endpoints in the same set.
Let $e$ denote the $n$-vector of ones and let $w_{ii}=0$ for $i=1,\ldots,n$.
Show that the following semidefinite program gives a lower bound
on the optimal value of this problem, where $X$ is an $n \times n$ matrix:
\begin{displaymath}
\begin{array}{lrcll}
\min & 0.5  \sum_{i=1}^n \sum_{j=1}^n w_{ij} X_{ij}  \\
\mbox{subject to} & X_{ii} & = & 1 & i = 1,\ldots,n \\
& Xe & = & pe \\
& X && \multicolumn{2}{l}{\mbox{symmetric and positive semidefinite}}
\end{array}
\end{displaymath}
%where $w_{ii}=0$ for $i=1,\ldots,n$ and $e$ is the vector of ones.

\item  \label{q.greedy}
Define a greedy algorithm for the weighted node packing problem.
How does your algorithm do on the instance given in Figure 9.1 on page 343 of the text?

\item
How would you modify your algorithm of question~\ref{q.greedy}
to construct a GRASP routine for weighted node packing?
Give five sample packings found by your algorithm for the
instance given in Figure 9.1 on page 343 of the text.

\item
Describe a tabu search algorithm for weighted node packing.
Be sure to define your valid solutions, your neighbourhoods,
your tabu criteria,
and what you would do with violated constraints.
%and how you would handle violated constraints,
%or if you would allow violated constraints.
Run your algorithm on the instance given in Figure 9.1 on page 343 of the text,
starting from the solution you found in Question~\ref{q.greedy}
and running until the algorithm finds three more local minima.


%\item  \label{sdpclique}
%Given a graph $G=(V,E)$, let $n=|V|$ be the number of nodes.
%We can set up a semidefinite programming problem
%with the $n \times n$ symmetric matrix $X$ of variables:
%\begin{displaymath}
%\begin{array}{lrcllr}
%\max & \sum_{i=1}^n \sum_{j=1}^n X_{ij} &&\qquad \\
%\mbox{subject to} & X_{ij} & = & 0 & \mbox{if } (i,j) \not\in E  & \qquad (P) \\
%& \sum_{i=1}^n X_{ii} & = & 1  \\
%& X & \succeq & 0
%\end{array}
%\end{displaymath}
%Show that the optimal value of this SDP gives an upper bound on the
%maximum cardinality of a clique in~$G$.

%%\pagebreak

%\item
%Problem $(P)$ in question~\ref{sdpclique} has a dual:
%\begin{displaymath}
%\begin{array}{lrcrcrcllr}
%\min &&& z \\
%\mbox{subject to} & -S_{ii} & + & z &&& = & 1 & i=1,\ldots,n & \qquad (D)  \\
%& -S_{ij} &&& + & y_{ij} & = & 1 &   \mbox{if } (i,j) \not\in E  \\
%& -S_{ij} &&&&& = & 1 & \mbox{if } (i,j) \in E \\
%& S &&&&& \succeq & 0
%\end{array}
%\end{displaymath}
%Here, $S$ is constrained to be a symmetric positive semidefinite $n \times n$
%matrix,
%$z$ is a scalar, and each $y_{ij}$ for $(i,j) \not \in E$ is a scalar with
%$y_{ij}=y_{ji}$.
%Show that the optimal value of the dual problem gives a lower bound
%on the minimum number of colours needed to colour the nodes of the
%graph so that no adjacent nodes receive the same colour.
%(Hint: The sum of positive semidefinite matrices is positive semidefinite.)

   \item
   We have an integer program of the form
   \begin{displaymath}
   \begin{array}{lrclr}
   \min & c^Tx \\
   \mbox{s.t.} & Ax & \geq & b & \qquad (IP)  \\
   & x && \mbox{binary}
   \end{array}
   \end{displaymath}
   where $b>0$ and $c>0$, and each nonzero element $a_{ij}$ of $A$ satisfies
   $a_{ij} \geq b_i$.
         \begin{enumerate}
           \item
           Show that the optimal solution to the LP relaxation of $(IP)$
           can be used to find a feasible integer solution.
            \item
           Let $p$ be the maximum number of nonzeroes in any row of~$A$.
           Give a polynomial time algorithm to find a feasible integer
           solution that has value within a factor of $p$ of the optimal value.
         \end{enumerate}


%   \item
%In the examples below, $x$ is required to be a nonnegative vector
%in~$\re^3$ and $y$ is constrained to be a binary vector with three
%components.
%A set of constraints that must be satisfied by $x$ and $y$ is given,
%and a point that satisfies these constraints is also given.
%Find a valid flow cover inequality that cuts off this point.
%\begin{enumerate}
%\item
%Constraints:
%\begin{displaymath}
%x_1+x_2+x_3 \leq 7, \; x_1 \leq 3y_1, \; x_2 \leq 5y_2, \; x_3 \leq 6y_3.
%\end{displaymath}
%Point:
%\begin{displaymath}x=(2,5,0), \; y=(\frac{2}{3},1,0).
%\end{displaymath}
%\item
%Constraints:
%\begin{displaymath}
%7 \leq x_1+x_2+x_3, \; x_1 \leq 3y_1, \; x_2 \leq 5y_2, \; x_3 \leq 6y_3.
%\end{displaymath}
%Point:
%\begin{displaymath}x=(2,5,0), \; y=(\frac{2}{3},1,0).
%\end{displaymath}
%\end{enumerate}

%   \item
%         Interpret the numbers on the edges in the graph
%         below as both edge numbers and edge weights.
%         We are looking for a minimum weight spanning tree,
%         subject to the constraints
%         \begin{displaymath}
%           x_1 \leq x_6, \quad x_2 \leq x_7, \quad x_3 \leq x_8,
%         \end{displaymath}
%         where $x_i=1$ if edge $e_i$ is in the tree, and 0 otherwise.
%         Use tabu search to find the minimum weight spanning tree subject
%         to these constraints. The move should be from tree to tree,
%         with one edge entering and one leaving.
%         Use a penalty of 50 for each violated constraint.
%         The tabu restrictions are that added edges can not leave for
%         one iteration, and dropped edges can not enter for one iteration.
%         The initial tree should be the minimum weight spanning tree,
%         without considering the constraints.
%         (You should not need to use aspiration criteria.
%         You may assume that once you have visited two local minima,
%         one of them is the optimal solution.)

%\begin{picture}(420,140)(-80,0)
%  \put(20,20){\line(1,0){100}}
%  \put(20,20){\line(1,1){100}}
%  \put(120,20){\line(0,1){100}}
%  \put(120,120){\line(1,0){100}}
%  \put(120,20){\line(1,0){100}}
%  \put(220,20){\line(0,1){100}}
%  \put(220,20){\line(1,1){100}}
%  \put(220,120){\line(1,0){100}}

%
%  \put(20,20){\circle*{3}}
%  \put(120,20){\circle*{3}}
%  \put(120,120){\circle*{3}}
%  \put(220,20){\circle*{3}}
%  \put(220,120){\circle*{3}}
%  \put(320,120){\circle*{3}}

%
%  \put(55,70){1}
%  \put(70,10){8}
%  \put(125,70){4}
%  \put(170,10){5}
%  \put(170,125){3}
%  \put(210,70){6}
%  \put(270,125){7}
%  \put(270,60){2}
%\end{picture}

\end{enumerate}

\vfill

\begin{tabular}{@{\hspace{1in}}l}
   John Mitchell  \\
   Amos Eaton 325  \\
   x6915.  \\
   mitchj@rpi.edu  \\
   Office hours:
   Tuesday 2pm -- 3pm, Wednesday 11am -- noon.
\end{tabular}


\end{document}
