%\documentstyle[11pt]{article}
\documentclass[12pt]{article}

\usepackage{fullpage}
\usepackage{hyperref}


\pagestyle{empty}

%\oddsidemargin -.5in
%\textwidth 7.5in
%\textheight 9.8in
%\topmargin -30pt
%\headsep 0in
%\headheight 0in

%\renewcommand{\baselinestretch}{1.5}

\newcommand{\til}{\char '176}

\begin{document}

\begin{center}
  {\bf\large\bf  MATP6620 / DSES6760  \\
     Combinatorial Optimization and Integer Programming} \\
{\bf\large\bf Class Project}
\end{center}
\begin{flushright} \today \end{flushright}




\begin{flushleft}  \underline{\bf Projects and Presentations:}  \end{flushleft}

\noindent
The project will involve modeling and computational testing.
You will write up your solution and give a presentation in class.
%
%\noindent
Your project can be one of the following:
\begin{itemize}
\item
a topic arising in your research
that fits well with the topics covered in the course.
You would work on your own on such a project.
\item
another project you suggest or I suggest.
You can work in groups of up to three people on such
a project.
All group members should contribute equally to the project. 
Each individual should turn in a one-page description of their 
contribution to the project along with the group report.
\end{itemize}

%\bigskip

\noindent
Possible topics include:
\begin{itemize}
\item
A cutting plane approach to an integer programming problem.
The cutting plane methods will require the use of AMPL or the CPLEX
callable library
(or C or Fortran).
\item
A semidefinite
programming relaxation approach to an integer programming problem.
This will require the use of an SDP package
(written in MATLAB).
\item
Investigation of a
heuristic method or of a relaxation approach for an integer programming problem.
\end{itemize}

%If you choose the ``presentation'' option,
%you will make a
% presentation of recent research for approximately 50 minutes,
%on some fairly broad topic such as tabu search or approximation algorithms.
%It will be lecture-style and will require the synthesis of a number of
%papers.

\noindent
You can suggest the project to me, or I can suggest one.
I would like you to give me an idea for your project by {\bf Friday,
February 23}. This should be at least a paragraph, perhaps a page,
longer if you have more you want to tell me.
Sources for optimization problems include the following:
\begin{itemize}
\item
\noindent
\url{http://www.nada.kth.se/~viggo/problemlist/compendium.html}
\end{itemize}


%\bigskip

\begin{flushleft}  \underline{\bf Other matters:}  \end{flushleft}

\noindent
There will be no class on Tuesday February 20.
I will give out Homework 4 on Friday February 16, and it will be due on Friday March~2.


\vfill


\begin{tabular}{@{\hspace{.3in}}l@{\hspace{.5in}}l}
   John Mitchell  &  276--6915  \\
   Amos Eaton 325 &  mitchj@rpi.edu \\
\multicolumn{2}{@{\hspace{.3in}}l}{\url{http://www.rpi.edu/~mitchj}}  \\
%   276--6915.  \\
%   mitchj@rpi.edu  %\\
 \multicolumn{2}{@{\hspace{.3in}}l}{Office hours:  Tuesday, Wednesday, 2.0--4.0.}  \\
\end{tabular}






\end{document}
