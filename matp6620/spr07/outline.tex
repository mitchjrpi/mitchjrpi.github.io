%\documentstyle[11pt]{article}
\documentclass{article}

\pagestyle{empty}

%\oddsidemargin -.5in
%\textwidth 7.5in
%\textheight 9.8in
%\topmargin -30pt
%\headsep 0in
%\headheight 0in

\usepackage{fullpage}

%\renewcommand{\baselinestretch}{1.5}

\newcommand{\til}{\char '176}

\begin{document}

\begin{center}
  {\bf\large\bf  MATP6620 / DSES6760  \\
     Combinatorial Optimization and Integer Programming}
\end{center}
\begin{flushright} \today \end{flushright}

\begin{flushright}  TF	 12:00--1:50 Jonsson-Rowland 1W01
       \hspace*{\fill}   Spring 2007  \end{flushright}

\begin{flushleft}  \underline{\bf Course Outline}  \end{flushleft}

I intend to follow this outline fairly closely,
but, if appropriate, I will alter what is included in the course.
Problems such as node packing, the traveling salesman problem,
the knapsack problem,
the quadratic assignment problem, and the MaxCut problem, will be
discusses throughout the course.

\begin{enumerate}
  \item  {\bf Fundamentals:}  (1--2 weeks)
           Linear programming. Graph theory and network flows.
  \item  {\bf Algorithm complexity and NP-completeness:}  (2--3 weeks)
           Analysis of algorithms.  Polynomial time algorithms.  Separation
           and optimization problems.  The ellipsoid method and its
           consequences for combinatorial optimization.
  \item  {\bf Branch-and-bound.}  (1 week)
  \item  {\bf Polyhedral theory and cutting plane methods:}  (3 weeks)
           Facets of polyhedra.
           Gomory's cutting plane procedure.  Chvatal cuts.  Knapsack problem.
           Branch-and-cut.
  \item  {\bf Heuristic algorithms:} (2 weeks)
           Simulated annealing. Tabu search. Genetic algorithms.
  \item  {\bf Duality in integer programming.}  (1 week)
           Lagrangian relaxation methods.
  \item {\bf Higher order relaxations:} (1--2 weeks)
           Semidefinite programming, lift-and-project, reformulation-linearization.
\end{enumerate}

\begin{flushleft}  \underline{\bf Homework:}  Approximately every two weeks.  \end{flushleft}
Homework and exam solutions will be placed on reserve in the
library.  You may discuss the homeworks with other students,
but you must write up your solutions on your own.


\begin{flushleft}  \underline{\bf Exam:}  One in-class final.  \end{flushleft}
As you would expect, the exam must be all your own work.

\begin{flushleft}  \underline{\bf Project:}  \end{flushleft}
The project will involve modeling and computational testing. You will write up your solution and give a presentation in class. Your project can be one of the following:
\begin{itemize}
\item
a topic arising in your research that fits well with the topics covered in the course. You would work on your own on such a project.
\item
another project you suggest or I suggest. You can work in groups of up to three people on such a project. All group members should contribute equally to the project. Each individual should turn in a one-page description of their contribution to the project along with the group report.
\end{itemize}
Possible topics include:
\begin{itemize}
\item
A cutting plane approach to an integer programming problem. The cutting plane methods will require the use of AMPL or the CPLEX callable library or a package from COIN-OR (or C or Fortran).
\item
A semidefinite programming relaxation approach to an integer programming problem. This will require the use of an SDP package (written in MATLAB).
\item
Investigation of a heuristic method or of a relaxation approach for an integer programming problem.
\end{itemize}

%\begin{flushleft}  \underline{\bf Projects and Presentations:}  \end{flushleft}
%You can either work on an appropriate project or make a presentation of recent research.
%You can suggest the project to me, or I can suggest one.
%The project will involve modeling and computational testing, perhaps of a heuristic method or
%of a relaxation approach.

\begin{flushleft}  \underline{\bf Grading policy:}
     $\frac{1}{3}$ homeworks, $\frac{1}{3}$ final, $\frac{1}{3}$ project/presentation.  \end{flushleft}

\begin{flushleft} \underline{\bf Office hours:} Tuesday 2--3pm, Wednesday, 11am--noon   \end{flushleft}

%\begin{flushright}   /over     \end{flushright}

%\pagebreak

\begin{flushleft}   \underline{\bf Textbooks:}   \end{flushleft}

%\begin{tabular}{l@{\hspace{.2in}}p{5in}}
\begin{tabular}{l@{\hspace{.2in}}p{5.8in}}
   \multicolumn{2}{l}{\bf Required:}  \\
    \quad     &  Nemhauser and Wolsey, {\em Integer and Combinatorial
                    Optimization.}  Wiley, 1988.
                    Very detailed, and a standard reference.
                    Recently published in paperback.  \\
   \multicolumn{2}{l}{\bf Recommended:}  \\
    \quad     &   Papadimitriou and Steiglitz,  {\em Combinatorial
                    Optimization:
                    Algorithms and Complexity.}   Prentice Hall 1982.
                    A good alternate text.  More graph theory, less
                    polyhedral theory than N\&W.
                    Recently reprinted by Dover.   \\
   \multicolumn{2}{l}{\bf Also on reserve:}  \\
    \quad     &  Wolsey,  {\em Integer Programming.}  Wiley 1998.
                    Better coverage of heuristics and a little more
                    up-to-date than Nemhauser and Wolsey.
                    %This is the book I used in 1999.
                    \\
   \quad     & Lee, {\em A First Course in Combinatorial Optimization}.
                    Cambridge University Press, 2004.
                    Available online from the library.  \\
   \quad     & Schrijver, {\em Combinatorial Optimization: Polyhedra and Efficiency}.
                    Springer Verlag, 2003. Three volumes, encyclopedic,
                    but still only costing about \$100.  \\
   \quad      &   Cook, Cunningham, Pulleyblank, and Schrijver,
                    {\em Combinatorial Optimization.}  Wiley 1997.
                    \\
   \quad      &   Parker and Rardin, {\em Discrete
                    Optimization.}  Academic Press,
                    1988.  A good reference for branch-and-bound.  \\
   \quad      &   Gr\"{o}tschel et al., {\em Geometric Algorithms and
                    Combinatorial Optimization.}
                    Springer-Verlag, 1993.
                    Discusses the theoretical importance of the
                    ellipsoid algorithm for integer programming.  \\
   \quad      &   Aarts and Lenstra, {\em Local Search in Combinatorial
                    Optimization.}
                    Wiley, 1997.
                    A good text on local search heuristics. \\
   \quad      &  Garey and Johnson,  {\em Computers and Intractability.}
                    Freeman, 1979. The bible of NP-completeness.  \\
%   \quad      &   Nemhauser et al., {\em Optimization.}  North-Holland, 1989.
%                    Several different authors contribute chapters on various
%                    aspects of optimization.
%                    See particularly chapters 2, 4, 5,~6.  \\
              & \quad \\
   \quad      &   I will also put {\em selected papers} on reserve.
\end{tabular}


\begin{flushleft}   {\bf \underline{Academic integrity:}}  \end{flushleft}
   Student-teacher relationships are based on mutual trust.
Acts which violate this trust undermine the educational process.
The {\em Rensselaer Handbook} defines various forms of academic
dishonesty and procedures for responding to them.
The penalties for cheating can include failure in the course,
as well as harsher punishments.


\vspace{.5in}


\begin{tabular}{@{\hspace{3in}}l@{\hspace{.5in}}l}
   John Mitchell  &  276--6915  \\
   Amos Eaton 325 &  mitchj@rpi.edu \\
\multicolumn{2}{@{\hspace{3in}}l}{http://www.rpi.edu/\til mitchj}
%   276--6915.  \\
%   mitchj@rpi.edu  %\\
%   Office hours:  Monday, Wednesday, 2.0--4.0.  \\
\end{tabular}






\end{document}
