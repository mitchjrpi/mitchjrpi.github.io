\documentclass[12pt]{article}

\pagestyle{empty}

\oddsidemargin -.5in
\textwidth 7.5in
\textheight 10in
\topmargin -30pt
\headsep 0in
\headheight 0in

%\usepackage{fullpage}

\newcommand{\real}{I\!\! R}
\newcommand{\binary}{I\!\! B}
\newcommand{\integer}{Z\!\!\! Z}

\begin{document}

\begin{center}
  \begin{large}
     MATP6620 / DSES6760 Combinatorial Optimization and Integer Programming \\
       Homework 1.
  \end{large}
\end{center}

\begin{flushright}
   Due:  Friday, January 26, 2007.
\end{flushright}

%\vspace{\baselineskip}

%Throughout, let $\binary^n:=
%\{x \in \real^n: x_i = 0 \mbox{ or } 1 \;\; \forall i\}$.

\begin{enumerate}
%   \item  Model the following scheduling problem as a mixed integer
%programming problem:
%\begin{quote}
%A set of $n$ jobs must be carried out on a single machine that can only
%do one job at a time.
%Each job $j$ takes $p_j$ hours to complete.
%Given job weights $w_j$ for $j=1,\ldots,n$,
%in what order should the jobs be carried out in order to minimize
%the weighted sum of their completion times?
%\end{quote}
   \item  Model the following scheduling problem as a mixed integer
programming problem:
\begin{quote}
A set of $n$ jobs must be carried out on a single machine that can only
do one job at a time.
Each job $j$ takes $p_j$ hours to complete.
Each job has a target time $t_j$ for $j=1,\ldots,n$,
and a priority weight $w_j$ for $j=1,\ldots,n$.
in what order should the jobs be carried out in order to minimize
the total weighted tardiness?
(The {\em tardiness} of a job is how late it is,
that is, the difference between the completion time and
the target time if the completion time is greater than the target time.
The tardiness is weighted by the priority weight~$w_j$.)
\end{quote}
  \item
Show that
\begin{eqnarray*}
S & := & \{x \in \integer_+^3: 7x_1 + 9x_2 + 12x_3  \geq 17\} \\
&=& \{x \in \integer_+^3: 4x_1 + 5x_2 + 6x_3  \geq 10 \}  \\
&=& \{x \in \integer_+^3:
x_1+x_2+x_3 \geq 2, \; 2x_1+3x_3+4x_4 \geq 6 \}.
%S & := & \{x \in \binary^4: 15x_1 + 12x_2 + 37x_3 + 9x_4 \leq 55\} \\
%&=& \{x \in \binary^4: x_1 + x_2 + 2x_3 + x_4 \leq 3 \}  \\
%&=& \{x \in \binary^4:
%x_1+x_2+x_3 \leq 2, \; x_1+x_3+x_4 \leq 2, \; x_2+x_3 + x_4 \leq 2 \}.
%%%
%S & := & \{x \in \binary^4: 97x_1 + 32x_2 + 25x_3 + 20x_4 \leq 139\} \\
%&=& \{x \in \binary^4: 2x_1 + x_2 + x_3 + x_4 \leq 3 \}  \\
%&=& \{x \in \binary^4:
%x_1+x_2+x_3 \leq 2, \; x_1+x_3+x_4 \leq 2, \; x_1+x_2 + x_4 \leq 2 \}.
%\begin{array}{ccccccccc}
%x_1 &+& x_2 &+& x_3 &&& \leq & 2 \\
\end{eqnarray*}
Which formulation do you think is most effective for solving
$\max\{c^Tx: x\in S\}$? Why?
You may want to experiment with AMPL to confirm your answer.
%   \item
%Let $x \in \binary^n$, where $\binary^n:=
%\{x \in \real^n: x_i = 0 \mbox{ or } 1 \;\; \forall i\}$.
%What, if anything, is implied by the following?
%\begin{enumerate}
%\item $x_i + x_j \geq 1$ and $ x_i \leq x_j$.
%\item $x_i \leq x_j$ and $x_i+x_j + x_k \leq 1$.
%\end{enumerate}
  \item  Consider the following LP, which we call problem $(P)$:
         \begin{displaymath}
          \begin{array}{lrcrcc}
           \max         &  2y_1 & + & 3y_2 &      &     \\
           \mbox{s.t.}  &  y_1 & + &  2y_2 & \leq & 7   \\
                        & -y_1 & + &  y_2 & \leq & 4   \\
                        &  y_1    &   &  & \leq & 1.
          \end{array}
         \end{displaymath}
         \begin{enumerate}
           \item  Solve $(P)$ geometrically.
           \item  What is the dual $(D)$ of $(P)$?
           \item  Write down the complementary slackness conditions for
                  this problem and use them to solve the dual $(D)$.
                  Check your answer by evaluating the optimal costs for
                  $(P)$ and $(D)$.
           \item  For problem $(D)$, what is the optimal basis matrix $B$?
                  What are the reduced costs at the optimal solution?
                  What are $B^{-1}$, $B^{-1}N$, $c_B^TB^{-1}N$,
                  $c_B^TB^{-1}b$, and $c_N^T-c_B^TB^{-1}N$?
         \end{enumerate}
%  \item  %The following game for two players can be enjoyed at the Ontario
%         %Science Center in Toronto.
%         A game is played on a set of six points, no three collinear.
%         The players, red and black, alternate turns. When it is a player's
%         turn, she selects two of the six points not yet connected by a line
%         segment, and draws a line segment of her colour between them. The
%         first player to complete a triangle of her colour loses. Prove that
%         this game cannot end in a draw.
%%         (Hint: Assume the game ends in a draw and obtain a contradiction.
%%         Consider one vertex at the end of the game.
%%         If it has two black edges incident to it, what can you say about
%%         the colour of the edge between the other two endpoints of those
%%         black edges? What happens if the vertex has three black edges
%%         incident to it?)


\item Model the following problem as an integer program:
\begin{quote}
Given a graph $G=(V,E)$, vertex weights $w_v$,
and subsets $U_i, i=1,\ldots,k$ of the vertices.
Find a minimum weight subset $W$ of the vertices such that
$W$ contains at least one vertex in each $U_i$
and so that the subgraph of $G$ induced by the vertices in $W$ is  connected.
\end{quote}

%\item Model the following feasibility problem as an integer programming feasibility problem:
%\begin{quote}
%Given a graph $G=(V,E)$, does there exist a partition of
%the edges $E$ into two sets $E_1$ and $E_2$
%such that neither $E_1$ nor $E_2$ contains a triangle?
%\end{quote}
%Is the LP relaxation to your formulation feasible?

%  \item  Given a connected graph $G=(V,E)$ with edge weights $w$,
%a greedy algorithm
%for constructing a minimum spanning tree is as follows.
%(Let $m$ be the number of edges in the graph.
%Assume $w_1 \leq w_2 \leq \ldots \leq w_m$.)
%\begin{itemize}
%\item Initialize with $T = \phi$, the empty set.
%\item For $i=1,\ldots,m$:
%Add edge $i$ to the tree, provided this does not create a cycle.
%\end{itemize}
%Prove that this does create a spanning tree of minimum total weight.
%(Hint: Assume there is another spanning tree that is better than the
%one created by the algorithm. Let $e$ be the edge with minimum weight that
%is in this alternative tree and not in the tree created by the algorithm.
%Derive a contradiction.)
\end{enumerate}

\vfill

\begin{tabular}{@{\hspace{.5in}}l}
   John Mitchell  \\
   Amos Eaton 325  \\
   x6915.  \\
   mitchj@rpi.edu  \\
   Office hours:
   Tuesday 2--3pm, Wednesday 11am--12noon.
\end{tabular}


\end{document}
