%\documentstyle[11pt]{article}
\documentclass[12pt]{article}


\usepackage{fullpage}

\pagestyle{empty}

%\oddsidemargin -.5in
%\textwidth 7.5in
%\textheight 9.8in
%\topmargin -30pt
%\headsep 0in
%\headheight 0in

%\renewcommand{\baselinestretch}{1.5}

\newcommand{\til}{\char '176}

\begin{document}

\bibliographystyle{plain}

\begin{center}
  {\bf\large\bf  MATP6620 / DSES6760  \\
     Combinatorial Optimization and Integer Programming} \\
{\bf\large\bf Class Project}
\end{center}
\begin{flushright} \today \end{flushright}




\begin{flushleft}  \underline{\bf Structure of the presentations:}
\end{flushleft}

The presentations should be about 20 minutes long.
You should prepare 14 copies of your transparencies to hand out in advance.

There will be eight presentations,
with four each on April 27 and May~1.
The time of each presentation on each day will be determined on the day.

Your report should be about 5 pages long and is due
in class on April~27.
It must be typewritten.
It should describe the problem you worked on,
what you did to solve the problem,
and the significance of what you did.
You should also cite relevant references
and state what was novel about your approach.
For group projects, also hand in one sheet from each group member
discussing his or her individual contribution.

\bigskip

\begin{flushleft}  \underline{\bf Schedule:}  \end{flushleft}

\begin{itemize}
\item {\bf Friday, April 27:}
\begin{itemize}
\item {\bf Matthew Brom, Miguel Jaller, Sarah Nurre}
\item {\bf Deng Ge, Wilfredo Yushimoto}
\item {\bf Greg Moore}
\item {\bf Di Wang, Zhi Zhou}
\end{itemize}
\item {\bf Tuesday, May 1:}
\begin{itemize}
\item {\bf Haibing Gao, He Zhang}
\item {\bf Cindy Hui}
\item {\bf Sergey Yakovlev}
\item {\bf Bin Yu}
\end{itemize}
\end{itemize}


\vfill


\begin{tabular}{@{\hspace{.3in}}l@{\hspace{.5in}}l}
   John Mitchell  &  276--6915  \\
   Amos Eaton 325 &  mitchj@rpi.edu \\
http://www.rpi.edu/\til mitchj  \\
%   276--6915.  \\
%   mitchj@rpi.edu  %\\
   Office hours:  Tuesday 2--3, Wednesday 11--noon.  \\
\end{tabular}






\end{document}
