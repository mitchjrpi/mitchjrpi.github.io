%\documentclass{article}
\documentclass[12pt]{article}

\pagestyle{empty}

%\oddsidemargin -.5in
%\textwidth 7.5in
%\textheight 10in
%\topmargin -30pt
%\headsep 0in
%\headheight 0in

\newcommand{\til}{\char '176}

\newcommand{\real}{I\!\! R}

\begin{document}

\begin{center}
  \begin{large}
     MATP6960 Stochastic Programming
  \end{large}
\end{center}

\begin{center}
http://www.rpi.edu/\til mitchj/stochprog
\end{center}

\begin{center}
  \begin{large}
 Student Presentations
  \end{large}
\end{center}


\vspace{\baselineskip}


\begin{itemize}
\item
Student presentations will take place on Wednesday, December 11,
from 6.30pm to 9.30pm in DCC 239.
\item
There will be eleven presentations, so you will have approximately 15
minutes each.
\item
The order of presentation will be random.
\item
You need to pick the most important points from your paper.
You should limit your talk to at most 8 overheads.
\item
I need a 3--5 page summary of the contributions of the paper
on Tuesday, December~10.
This will be photocopied and handed out.
\item
In order to encourage questions, your grade will not be lowered if you
are unable to answer questions from other students, but it may be raised.
Moreover, I may give some
{\em bonus points} for asking a particularly good question.
\end{itemize}

\vfill

\begin{tabular}{@{\hspace{.5in}}ll}
   John Mitchell  &
   Amos Eaton 325  \\
   x6915.  &
   mitchj@rpi.edu  \\
   Office hours:  &
   Tuesday: 2pm -- 4pm.
\end{tabular}

\end{document}
