%\documentclass{article}
\documentclass[12pt]{article}

\pagestyle{empty}

\oddsidemargin -.5in
\textwidth 7.5in
\textheight 10in
\topmargin -30pt
\headsep 0in
\headheight 0in

\newcommand{\real}{I\!\! R}

\begin{document}

\begin{center}
  \begin{large}
     MATP6960 Stochastic Programming, Homework 5
  \end{large}
\\
Revised: November 4, 2002.
\end{center}

\begin{flushright}
   Due:  Thursday, November 7, 2002.
\end{flushright}

\vspace{\baselineskip}


\begin{enumerate}
\item Birge and Louveaux, page 179, question 1.
\item Birge and Louveaux, page 191, question 2.
\item Birge and Louveaux, page 197, question 3.
For this question, you do not need to use the specialized algorithm
for general network flows. It is sufficient to {\bf formulate
the linear program} in the manner described in the section
and to then {\bf solve that LP using a package}.
\item Birge and Louveaux, page 343, question 7.
For this question, take $\xi^0=\bar{\xi}=3$.
Take just four further iterations, with $\xi^1=1.5$,
$\xi^2=4.2$, $\xi^3=3.4$, and $\xi^4=3.7$.
\end{enumerate}

\vfill

\begin{tabular}{@{\hspace{.5in}}ll}
   John Mitchell  &
   Amos Eaton 325  \\
   x6915.  &
   mitchj@rpi.edu  \\
   Office hours:  &
   Tuesday: 2pm -- 4pm.
\end{tabular}

\end{document}
