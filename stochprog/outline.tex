\documentstyle[11pt]{article}

\pagestyle{empty}
\newcommand{\til}{\char '176}

\oddsidemargin -.1in
\textwidth 5.8in
\textheight 8.5in
\topmargin -30pt
%\headsep 0in
%\headheight 0in

%\oddsidemargin -.5in
%\textwidth 7.5in
%\textheight 9.5in
%\topmargin -30pt
%\headsep 0in
%\headheight 0in

\renewcommand{\baselinestretch}{1.2}

\begin{document}

\begin{center}
  {\large\bf  MATP6960 Stochastic Programming}
\end{center}
%\begin{flushright} \today \end{flushright}

\begin{flushright}  MR 12:00--1:50, Carnegie 106
       \hspace*{\fill}   Fall 2002  \end{flushright} 

{\raggedright  {\bf \underline{Course Outline:}} }

\noindent
Many optimization problems involve uncertainty,
where the eventual outcome depends on a future random event.
Stochastic programming is concerned with decision making
in the presence of uncertainty.
Topics include
modeling uncertainty in optimization problems,
algorithms for stochastic programming, and
approximation and sampling methods.
Applications discussed will include portfolio optimization.  \\

\noindent
The course will follow the text by Birge and Louveaux,
with some supplementary material.
I expect to cover most of the introductory material in Part~I (Models)
and at least some of the material in each of Parts II (Basic Properties),
III (Solution Methods), and IV (Approximation and Sampling Methods).  \\

%\begin{enumerate}
%  \item {\bf Polyhedral theory:} (2 weeks)
%  \end{enumerate}

{\raggedright {\bf \underline{Homework:}} }

\noindent
I will give out a homework approximately every two weeks.
The homeworks will probably contain some computational work.
You should learn a fair amount from the homeworks.
  Therefore, try working out the solutions on your own.
  If you have difficulties,
  you may talk to me or to other students about the homeworks,
  but you must write up your solutions on your own.   \\

{\raggedright {\bf \underline{Presentation:}} }

\noindent
At the end of the semester, each student will present a recent paper from
the stochastic programming literature.
I will provide a list of papers.
You can pick a paper from this list to present, or you can suggest
a different paper, or I will suggest a paper.
Each talk will last approximately
fifteen to twenty minutes.
You should describe the major contribution(s) of the paper you present.
You should type up a short (3-5 pages) analysis of the paper, which I will
photocopy and give out {\em before} your presentation.  \\

{\raggedright  {\bf \underline{Grading policy:}} }

\noindent
Homeworks will count for 75\% of the course grade,
with the presentations counting for the remaining 25\%.  \\

{\raggedright  {\bf \underline{Office hours:}} }

\noindent
Tuesday, 2--4pm, or by appointment.

\vspace{\fill}

\begin{flushright}   /over     \end{flushright}

\pagebreak

{\raggedright   {\bf \underline{Textbooks:}}   }  \\

\begin{tabular}{l@{\hspace{.2in}}p{4.5in}}
              &   {\bf Birge and Louveaux,} {\em Introduction to Stochastic Programming.}
                     Springer, 1997. 
                     Required. I will follow this text for much of the course.  \\
              &   {\bf Kall and Wallace,} {\em Stochastic Programming.}
                    Wiley, 1994.
                    Recommended. This book was out-of-print for a while.
                    It has recently been republished, but at an excessive price ($>$\$200).
\end{tabular}

\medskip
\noindent
Because there is a great deal of continuing research on stochastic programming,
I will draw some material from current papers.
I will put several
papers on reserve as the semester progresses.  \\

{\raggedright   {\bf \underline{Prerequisites:}}  }

\noindent
MATP4700 / DSES4770, or permission of instructor.
Familiarity with linear programming and linear algebra will be assumed.
Some elements of MATP6600 / DSES6780 and MATP4820 / DSES4780,
such as the Karush-Kuhn-Tucker
optimality conditions and Newton's
method, will be used, although it is not necessary to have seen this material
before.  \\

{\raggedright  {\bf \underline{The World Wide Web:}}  }

\noindent
This outline, the homeworks, and other information
will be available via my homepage,
\begin{quote}
  http://www.rpi.edu/\til mitchj/stochprog
\end{quote}

{\raggedright   {\bf \underline{Academic Integrity:}}  }

\noindent
Student-teacher relationships are based on mutual trust.
Acts which violate this trust undermine the educational process.
The {\em Rensselaer Handbook} defines various forms of academic
dishonesty and procedures for responding to them.
The penalties for cheating can include failure in the course,
as well as harsher punishments. \\


{\raggedright   {\bf \underline{Appealing grades:}} }

\noindent
As with any other administrative question regarding this course,
see me in the first instance.  If we are unable to reach agreement,
you may appeal my decision to the Chair of the Math Sciences department. \\


\vspace{\fill}


\begin{tabular}{@{\hspace{2in}}ll}
   John Mitchell    &  276--6915 \\
   Amos Eaton 325   &  mitchj@rpi.edu  \\
   \multicolumn{2}{l}{{\hspace{1.92in}}Office hours:
Tuesday, 2--4pm, or by appointment.}
\end{tabular}






\end{document}
