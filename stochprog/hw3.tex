%\documentclass{article}
\documentclass[12pt]{article}

\pagestyle{empty}

\oddsidemargin -.5in
\textwidth 7.5in
\textheight 10in
\topmargin -30pt
\headsep 0in
\headheight 0in

\newcommand{\real}{I\!\! R}

\begin{document}

\begin{center}
  \begin{large}
     MATP6960 Stochastic Programming, Homework 3
  \end{large}
\end{center}

\begin{flushright}
   Due:  Thursday, October 10, 2002.
\end{flushright}

%\vspace{\baselineskip}


\begin{enumerate}
\item Birge and Louveaux, page 109, question 1.
\item Birge and Louveaux, page 109, question 2.
\item Birge and Louveaux, page 121, question 1.
Assume $x \geq 0$ and $\xi \geq 0$ for this question.
The {\em Gomory function} for $y_1^*$ is
$y_1^*=\min\{2,\lfloor \xi x \rfloor \}$.
See page 110 for other examples of Gomory functions.
\item Birge and Louveaux, page 121, question 2.
\item %PTSP question.
The complete graph below consists of a number of teeth and a base.
Each tooth has length one.
The base has width and length one.
It contains $k^4(2k-1)+2k(2k+1)+4 \approx 2k^5$ vertices arranged as follows:
\begin{itemize}
\item
There are $k$ teeth in the graph, each of width $\frac{1}{2k-1}$.
\item
There are $k^4$ points evenly spaced in each
short horizontal segment, both at the top and the bottom of the teeth.
\item
There are $2k+1$ teeth evenly spaced along each side of each tooth.
\item
There are four additional vertices at the corners of the handle.
\end{itemize}
The graph is complete, with each vertex connected to each other vertex.
The distance between two vertices is their standard Euclidean distance.
The optimal TSP tour for the graph is to travel up and down each tooth
successively and then go round the handle.
Let each vertex be present with probability $p=\frac{1}{k^3}$,
with the presence of each vertex independent of the others.
\begin{enumerate}
\item
Show that the expected length of the {\em a priori} tour given
by the optimal TSP tour is approximately $E(L_{TSP})=2k+2$, for large~$k$.
\item
Find another {\em a priori} tour that has an expected length of
approximately $E(L_{PTSP})=4$ in the probabilistic TSP, for large~$k$.
\item
As constructed, the ratio
$\frac{E(L_{TSP})}{E(L_{PTSP})}$
takes a value of approximately the fifth root of $n$ as $n\rightarrow\infty$,
where $n$ is the number of vertices in the graph.
By redesigning this graph, how large can you make the ratio
$\frac{E(L_{TSP})}{E(L_{PTSP})}$?
\end{enumerate}

\begin{picture}(300,160)(0,20)
\put(20,20){\line(0,1){160}}
\put(20,20){\line(1,0){160}}
\put(180,20){\line(0,1){160}}
\put(20,180){\line(1,0){20}}
\put(60,180){\line(1,0){20}}
\put(120,180){\line(1,0){20}}
\put(160,180){\line(1,0){20}}
\put(40,100){\line(1,0){20}}
\put(140,100){\line(1,0){20}}
\put(40,100){\line(0,1){80}}
\put(60,100){\line(0,1){80}}
\put(80,100){\line(0,1){80}}
\put(160,100){\line(0,1){80}}
\put(140,100){\line(0,1){80}}
\put(120,100){\line(0,1){80}}
\multiput(85,140)(5,0){7}{\circle*{3}}
\put(200,140){$k$ teeth, each of width $\frac{1}{2k-1}$ and height one}
\put(200,60){Handle of width one and height one}
\end{picture}
\end{enumerate}

\vfill

\begin{tabular}{@{\hspace{.5in}}ll}
   John Mitchell  &
   Amos Eaton 325  \\
   x6915.  &
   mitchj@rpi.edu  \\
   Office hours:  &
   Tuesday: 2pm -- 4pm.
\end{tabular}

\end{document}
