\batchmode
\documentclass[10pt]{article}
\makeatletter


\usepackage[myheadings]{fullpage}


\markboth{Math Models of OR, Exam 1A, Fall 2004}
   {Math Models of OR, Exam 1A, Fall 2004}


\usepackage[dvips]{color}
\pagecolor[gray]{.7}



\makeatletter

\makeatletter
\count@=\the\catcode`\_ \catcode`\_=8 
\newenvironment{tex2html_wrap}{}{} \catcode`\_=\count@
\makeatother
\let\mathon=$
\let\mathoff=$
\ifx\AtBeginDocument\undefined \newcommand{\AtBeginDocument}[1]{}\fi
\newbox\sizebox
\setlength{\hoffset}{0pt}\setlength{\voffset}{0pt}
\addtolength{\textheight}{\footskip}\setlength{\footskip}{0pt}
\addtolength{\textheight}{\topmargin}\setlength{\topmargin}{0pt}
\addtolength{\textheight}{\headheight}\setlength{\headheight}{0pt}
\addtolength{\textheight}{\headsep}\setlength{\headsep}{0pt}
\setlength{\textwidth}{349pt}
\newwrite\lthtmlwrite
\makeatletter
\let\realnormalsize=\normalsize
\global\topskip=2sp
\def\preveqno{}\let\real@float=\@float \let\realend@float=\end@float
\def\@float{\let\@savefreelist\@freelist\real@float}
\def\end@float{\realend@float\global\let\@freelist\@savefreelist}
\let\real@dbflt=\@dbflt \let\end@dblfloat=\end@float
\let\@largefloatcheck=\relax
\def\@dbflt{\let\@savefreelist\@freelist\real@dbflt}
\def\adjustnormalsize{\def\normalsize{\mathsurround=0pt \realnormalsize
 \parindent=0pt\abovedisplayskip=0pt\belowdisplayskip=0pt}\normalsize}%
\def\lthtmltypeout#1{{\let\protect\string\immediate\write\lthtmlwrite{#1}}}%
\newcommand\lthtmlhboxmathA{\adjustnormalsize\setbox\sizebox=\hbox\bgroup}%
\newcommand\lthtmlvboxmathA{\adjustnormalsize\setbox\sizebox=\vbox\bgroup%
 \let\ifinner=\iffalse }%
\newcommand\lthtmlboxmathZ{\@next\next\@currlist{}{\def\next{\voidb@x}}%
 \expandafter\box\next\egroup}%
\newcommand\lthtmlmathtype[1]{\def\lthtmlmathenv{#1}}%
\newcommand\lthtmllogmath{\lthtmltypeout{l2hSize %
:\lthtmlmathenv:\the\ht\sizebox::\the\dp\sizebox::\the\wd\sizebox.\preveqno}}%
\newcommand\lthtmlfigureA[1]{\let\@savefreelist\@freelist
       \lthtmlmathtype{#1}\lthtmlvboxmathA}%
\newcommand\lthtmlfigureZ{\lthtmlboxmathZ\lthtmllogmath\copy\sizebox
       \global\let\@freelist\@savefreelist}%
\newcommand\lthtmldisplayA[1]{\lthtmlmathtype{#1}\lthtmlvboxmathA}%
\newcommand\lthtmldisplayB[1]{\edef\preveqno{(\theequation)}%
  \lthtmldisplayA{#1}\let\@eqnnum\relax}%
\newcommand\lthtmldisplayZ{\lthtmlboxmathZ\lthtmllogmath\lthtmlsetmath}%
\newcommand\lthtmlinlinemathA[1]{\lthtmlmathtype{#1}\lthtmlhboxmathA  \vrule height1.5ex width0pt }%
\newcommand\lthtmlinlineA[1]{\lthtmlmathtype{#1}\lthtmlhboxmathA}%
\newcommand\lthtmlinlineZ{\egroup\expandafter\ifdim\dp\sizebox>0pt %
  \expandafter\centerinlinemath\fi\lthtmllogmath\lthtmlsetinline}
\newcommand\lthtmlinlinemathZ{\egroup\expandafter\ifdim\dp\sizebox>0pt %
  \expandafter\centerinlinemath\fi\lthtmllogmath\lthtmlsetmath}
\def\lthtmlsetinline{\hbox{\vrule width.1em\vtop{\vbox{%
  \kern.1em\copy\sizebox}\ifdim\dp\sizebox>0pt\kern.1em\else\kern.3pt\fi
  \ifdim\hsize>\wd\sizebox \hrule depth1pt\fi}}}
\def\lthtmlsetmath{\hbox{\vrule width.1em\vtop{\vbox{%
  \kern.1em\kern0.8 pt\hbox{\hglue.17em\copy\sizebox\hglue0.8 pt}}\kern.3pt%
  \ifdim\dp\sizebox>0pt\kern.1em\fi \kern0.8 pt%
  \ifdim\hsize>\wd\sizebox \hrule depth1pt\fi}}}
\def\centerinlinemath{%\dimen1=\ht\sizebox
  \dimen1=\ifdim\ht\sizebox<\dp\sizebox \dp\sizebox\else\ht\sizebox\fi
  \advance\dimen1by.5pt \vrule width0pt height\dimen1 depth\dimen1 
 \dp\sizebox=\dimen1\ht\sizebox=\dimen1\relax}

\def\lthtmlcheckvsize{\ifdim\ht\sizebox<\vsize\expandafter\vfill
  \else\expandafter\vss\fi}%
\makeatletter \tracingstats = 1 


\begin{document}
\pagestyle{empty}\thispagestyle{empty}%
\lthtmltypeout{latex2htmlLength hsize=\the\hsize}%
\lthtmltypeout{latex2htmlLength vsize=\the\vsize}%
\lthtmltypeout{latex2htmlLength hoffset=\the\hoffset}%
\lthtmltypeout{latex2htmlLength voffset=\the\voffset}%
\lthtmltypeout{latex2htmlLength topmargin=\the\topmargin}%
\lthtmltypeout{latex2htmlLength topskip=\the\topskip}%
\lthtmltypeout{latex2htmlLength headheight=\the\headheight}%
\lthtmltypeout{latex2htmlLength headsep=\the\headsep}%
\lthtmltypeout{latex2htmlLength parskip=\the\parskip}%
\lthtmltypeout{latex2htmlLength oddsidemargin=\the\oddsidemargin}%
\makeatletter
\if@twoside\lthtmltypeout{latex2htmlLength evensidemargin=\the\evensidemargin}%
\else\lthtmltypeout{latex2htmlLength evensidemargin=\the\oddsidemargin}\fi%
\makeatother

% !!! IMAGES START HERE !!!

{\newpage\clearpage
\lthtmldisplayA{displaymath13}%
\begin{displaymath}
  \begin{array}{|r|rrrr|}
  \multicolumn{1}{c}{} & x_1 & x_2 & x_3 & \multicolumn{1}{r}{x_4}  \\\hline
  3 & 0 & a & 0 & 1 \\\hline
  b & 1 & c & 0 & 3  \\
  4 & 0 & d & 1 & 1  \\\hline
  \end{array}
  \end{displaymath}%
\lthtmldisplayZ
\hfill\lthtmlcheckvsize\clearpage}

{\newpage\clearpage
\lthtmldisplayA{displaymath27}%
\begin{displaymath}
  M = \begin{array}{|r|rrrr|}
  \multicolumn{1}{c}{} & x_1 & x_2 & x_3 & \multicolumn{1}{r}{x_4}  \\\hline
  3 & 0 & -2 & 0 & 1 \\\hline
  2 & 1 & 1 & 0 & 3  \\
  4 & 0 & 4 & 1 & 1  \\\hline
  \end{array}
  \end{displaymath}%
\lthtmldisplayZ
\hfill\lthtmlcheckvsize\clearpage}

{\newpage\clearpage
\lthtmldisplayA{displaymath40}%
\begin{displaymath}
  M = \begin{array}{|r|rrrrrrr|}
  \multicolumn{1}{c}{} & x_1 & x_2 & x_3 & x_4 & x_5 & x_6 & \multicolumn{1}{r}{x_7}  \\\hline
  3 & 0 & -3 & 0 & 1 & -5 & 2 & 1 \\\hline
  4 & 1 & 1 & 0 & 3 & 2 & -1 & 1  \\
  9 & 0 & 2 & 1 & 1 & 3 & -1 & 4  \\\hline
  \end{array}
  \end{displaymath}%
\lthtmldisplayZ
\hfill\lthtmlcheckvsize\clearpage}

{\newpage\clearpage
\lthtmldisplayA{displaymath51}%
\begin{displaymath}
  P=
  \left[ \begin{array}{rrr}1&3&0\\0&1&0\\0&-2&1\end{array} \right] .
  \end{displaymath}%
\lthtmldisplayZ
\hfill\lthtmlcheckvsize\clearpage}

{\newpage\clearpage
\lthtmldisplayA{displaymath60}%
\begin{displaymath}
   \begin{array}{lrcrcrcll}
   \min & 3x_1 & + & x_2 & - & 4x_3 &  \\
   \mbox{subject to} & x_1 & + & x_2 & + & x_3 & = & 3  \\
   & 2x_1 & - & x_2 & + & x_3 & = & 6  \\
   \multicolumn{6}{r}{x_i} & \geq & 0 & \mbox{for } i=1,\ldots,3.
   \end{array}
   \end{displaymath}%
\lthtmldisplayZ
\hfill\lthtmlcheckvsize\clearpage}

{\newpage\clearpage
\lthtmldisplayA{displaymath70}%
\begin{displaymath}
   \begin{array}{lrcrcrcrcrcllr}
   \min & 8x_1 & + & x_2 & + & 2x_3 & + & 3x_4 & + & 4x_5 \\
   \mbox{subject to} & 5x_1 & + & x_2 & + & 2x_3 & + & 4x_4 &+ & 2x_5 & = & 9 && \qquad (P)  \\
   & 4x_1 & + & x_2 & + & x_3 & + & 3x_4 & + & 3x_5 & = & 7 \\
   \multicolumn{10}{r}{x_i} & \geq & 0 & \mbox{for } i=1,\ldots,4.
   \end{array}
   \end{displaymath}%
\lthtmldisplayZ
\hfill\lthtmlcheckvsize\clearpage}

{\newpage\clearpage
\lthtmldisplayA{displaymath80}%
\begin{displaymath}
   M= \begin{array}{|r|rrrrrrr|}
   \multicolumn{2}{r}{x_1} & x_2 & x_3 & x_4 & x_5 & y_1 & \multicolumn{1}{r}{y_1} \\\hline
   0 & 0 & 0 & 0 & 0 & 0 & 1 & 1 \\\hline
   2 & 1 & 0 & 1 & 1 & -1 & 1 & -1 \\
   5 & 3 & 1 & 0 & 2 & 4 & -1 & 2  \\\hline
   \end{array}
   \end{displaymath}%
\lthtmldisplayZ
\hfill\lthtmlcheckvsize\clearpage}


\end{document}
