\batchmode
\documentclass[12pt]{article}
\makeatletter


\usepackage{fullpage}


\providecommand{\real}{I\!\!R}  


\usepackage[dvips]{color}
\pagecolor[gray]{.7}



\makeatletter

\makeatletter
\count@=\the\catcode`\_ \catcode`\_=8 
\newenvironment{tex2html_wrap}{}{} \catcode`\_=\count@
\makeatother
\let\mathon=$
\let\mathoff=$
\ifx\AtBeginDocument\undefined \newcommand{\AtBeginDocument}[1]{}\fi
\newbox\sizebox
\setlength{\hoffset}{0pt}\setlength{\voffset}{0pt}
\addtolength{\textheight}{\footskip}\setlength{\footskip}{0pt}
\addtolength{\textheight}{\topmargin}\setlength{\topmargin}{0pt}
\addtolength{\textheight}{\headheight}\setlength{\headheight}{0pt}
\addtolength{\textheight}{\headsep}\setlength{\headsep}{0pt}
\setlength{\textwidth}{349pt}
\newwrite\lthtmlwrite
\makeatletter
\let\realnormalsize=\normalsize
\global\topskip=2sp
\def\preveqno{}\let\real@float=\@float \let\realend@float=\end@float
\def\@float{\let\@savefreelist\@freelist\real@float}
\def\end@float{\realend@float\global\let\@freelist\@savefreelist}
\let\real@dbflt=\@dbflt \let\end@dblfloat=\end@float
\let\@largefloatcheck=\relax
\def\@dbflt{\let\@savefreelist\@freelist\real@dbflt}
\def\adjustnormalsize{\def\normalsize{\mathsurround=0pt \realnormalsize
 \parindent=0pt\abovedisplayskip=0pt\belowdisplayskip=0pt}\normalsize}%
\def\lthtmltypeout#1{{\let\protect\string\immediate\write\lthtmlwrite{#1}}}%
\newcommand\lthtmlhboxmathA{\adjustnormalsize\setbox\sizebox=\hbox\bgroup}%
\newcommand\lthtmlvboxmathA{\adjustnormalsize\setbox\sizebox=\vbox\bgroup%
 \let\ifinner=\iffalse }%
\newcommand\lthtmlboxmathZ{\@next\next\@currlist{}{\def\next{\voidb@x}}%
 \expandafter\box\next\egroup}%
\newcommand\lthtmlmathtype[1]{\def\lthtmlmathenv{#1}}%
\newcommand\lthtmllogmath{\lthtmltypeout{l2hSize %
:\lthtmlmathenv:\the\ht\sizebox::\the\dp\sizebox::\the\wd\sizebox.\preveqno}}%
\newcommand\lthtmlfigureA[1]{\let\@savefreelist\@freelist
       \lthtmlmathtype{#1}\lthtmlvboxmathA}%
\newcommand\lthtmlfigureZ{\lthtmlboxmathZ\lthtmllogmath\copy\sizebox
       \global\let\@freelist\@savefreelist}%
\newcommand\lthtmldisplayA[1]{\lthtmlmathtype{#1}\lthtmlvboxmathA}%
\newcommand\lthtmldisplayB[1]{\edef\preveqno{(\theequation)}%
  \lthtmldisplayA{#1}\let\@eqnnum\relax}%
\newcommand\lthtmldisplayZ{\lthtmlboxmathZ\lthtmllogmath\lthtmlsetmath}%
\newcommand\lthtmlinlinemathA[1]{\lthtmlmathtype{#1}\lthtmlhboxmathA  \vrule height1.5ex width0pt }%
\newcommand\lthtmlinlineA[1]{\lthtmlmathtype{#1}\lthtmlhboxmathA}%
\newcommand\lthtmlinlineZ{\egroup\expandafter\ifdim\dp\sizebox>0pt %
  \expandafter\centerinlinemath\fi\lthtmllogmath\lthtmlsetinline}
\newcommand\lthtmlinlinemathZ{\egroup\expandafter\ifdim\dp\sizebox>0pt %
  \expandafter\centerinlinemath\fi\lthtmllogmath\lthtmlsetmath}
\def\lthtmlsetinline{\hbox{\vrule width.1em\vtop{\vbox{%
  \kern.1em\copy\sizebox}\ifdim\dp\sizebox>0pt\kern.1em\else\kern.3pt\fi
  \ifdim\hsize>\wd\sizebox \hrule depth1pt\fi}}}
\def\lthtmlsetmath{\hbox{\vrule width.1em\vtop{\vbox{%
  \kern.1em\kern0.8 pt\hbox{\hglue.17em\copy\sizebox\hglue0.8 pt}}\kern.3pt%
  \ifdim\dp\sizebox>0pt\kern.1em\fi \kern0.8 pt%
  \ifdim\hsize>\wd\sizebox \hrule depth1pt\fi}}}
\def\centerinlinemath{%\dimen1=\ht\sizebox
  \dimen1=\ifdim\ht\sizebox<\dp\sizebox \dp\sizebox\else\ht\sizebox\fi
  \advance\dimen1by.5pt \vrule width0pt height\dimen1 depth\dimen1 
 \dp\sizebox=\dimen1\ht\sizebox=\dimen1\relax}

\def\lthtmlcheckvsize{\ifdim\ht\sizebox<\vsize\expandafter\vfill
  \else\expandafter\vss\fi}%
\makeatletter \tracingstats = 1 


\begin{document}
\pagestyle{empty}\thispagestyle{empty}%
\lthtmltypeout{latex2htmlLength hsize=\the\hsize}%
\lthtmltypeout{latex2htmlLength vsize=\the\vsize}%
\lthtmltypeout{latex2htmlLength hoffset=\the\hoffset}%
\lthtmltypeout{latex2htmlLength voffset=\the\voffset}%
\lthtmltypeout{latex2htmlLength topmargin=\the\topmargin}%
\lthtmltypeout{latex2htmlLength topskip=\the\topskip}%
\lthtmltypeout{latex2htmlLength headheight=\the\headheight}%
\lthtmltypeout{latex2htmlLength headsep=\the\headsep}%
\lthtmltypeout{latex2htmlLength parskip=\the\parskip}%
\lthtmltypeout{latex2htmlLength oddsidemargin=\the\oddsidemargin}%
\makeatletter
\if@twoside\lthtmltypeout{latex2htmlLength evensidemargin=\the\evensidemargin}%
\else\lthtmltypeout{latex2htmlLength evensidemargin=\the\oddsidemargin}\fi%
\makeatother

% !!! IMAGES START HERE !!!

{\newpage\clearpage
\lthtmldisplayA{displaymath17}%
\begin{displaymath}
   \begin{array}{lccccl}
   \min & -x_1^2 & - & (x_2-9)^2 \\
   \mbox{subject to} & 4x_1^2 & + & x_2^2 & \leq & 400.
   \end{array}
   \end{displaymath}%
\lthtmldisplayZ
\hfill\lthtmlcheckvsize\clearpage}

{\newpage\clearpage
\lthtmldisplayA{displaymath25}%
\begin{displaymath}
         \begin{array}{lc}
            \min        &   \frac{1}{2} \parallel c-x {\parallel}^2_2  \\
            \mbox{s.t.} &    Ax = 0,
         \end{array}
      \end{displaymath}%
\lthtmldisplayZ
\hfill\lthtmlcheckvsize\clearpage}

{\newpage\clearpage
\lthtmlinlinemathA{tex2html_wrap_inline138}%
$c,x\in{I\!\!R}^n$%
\lthtmlinlinemathZ
\hfill\lthtmlcheckvsize\clearpage}

{\newpage\clearpage
\lthtmlinlinemathA{tex2html_wrap_inline140}%
$A\in{I\!\!R}^{m{\times}n}$%
\lthtmlinlinemathZ
\hfill\lthtmlcheckvsize\clearpage}

{\newpage\clearpage
\lthtmlinlinemathA{tex2html_wrap_inline156}%
$\bar{x}:=c-A^T(AA^T)^{-1}Ac$%
\lthtmlinlinemathZ
\hfill\lthtmlcheckvsize\clearpage}

{\newpage\clearpage
\lthtmlinlinemathA{tex2html_wrap_inline158}%
$\bar{x}$%
\lthtmlinlinemathZ
\hfill\lthtmlcheckvsize\clearpage}

{\newpage\clearpage
\lthtmlinlinemathA{tex2html_wrap_inline162}%
$K \subseteq I\!\!R^n$%
\lthtmlinlinemathZ
\hfill\lthtmlcheckvsize\clearpage}

{\newpage\clearpage
\lthtmldisplayA{displaymath43}%
\begin{displaymath}
   K^+ := \{v \in I\!\!R^n : v^Tx \geq 0 \;\;  \forall x \in K \}.
   \end{displaymath}%
\lthtmldisplayZ
\hfill\lthtmlcheckvsize\clearpage}

{\newpage\clearpage
\lthtmldisplayA{displaymath45}%
\begin{displaymath}
   \begin{array}{lrcl}
   \min & c^Tx \\
   \mbox{subject to} & Ax & = & b  \\
   & x & \in & K
   \end{array}
   \end{displaymath}%
\lthtmldisplayZ
\hfill\lthtmlcheckvsize\clearpage}

{\newpage\clearpage
\lthtmldisplayA{displaymath51}%
\begin{displaymath}
   \theta (u) := \inf\{c^Tx + u^T(b-Ax) : x \in K\}
   \end{displaymath}%
\lthtmldisplayZ
\hfill\lthtmlcheckvsize\clearpage}

{\newpage\clearpage
\lthtmlinlinemathA{tex2html_wrap_inline170}%
$b, u \in I\!\!R^m$%
\lthtmlinlinemathZ
\hfill\lthtmlcheckvsize\clearpage}

{\newpage\clearpage
\lthtmlinlinemathA{tex2html_wrap_inline176}%
$\theta(u)$%
\lthtmlinlinemathZ
\hfill\lthtmlcheckvsize\clearpage}

{\newpage\clearpage
\lthtmlinlinemathA{tex2html_wrap_inline178}%
$u \in I\!\!R^m$%
\lthtmlinlinemathZ
\hfill\lthtmlcheckvsize\clearpage}

{\newpage\clearpage
\lthtmldisplayA{displaymath54}%
\begin{displaymath}
   \max \; \theta(u) \quad \equiv \quad
   \begin{array}{lrcl}
   \max & b^Tu \\\mbox{subject to} & c - A^Tu & \in & K^+.
   \end{array}
   \end{displaymath}%
\lthtmldisplayZ
\hfill\lthtmlcheckvsize\clearpage}

{\newpage\clearpage
\lthtmldisplayA{displaymath60}%
\begin{displaymath}
   \begin{array}{lrcrcrcrcl}
   \min &&&&&&& x_{22} \\
   \mbox{subject to} & x_{11} & - & x_{12} & - & x_{21} & + & x_{22} & = & 0 \\
   & x_{11} &&&&&&& = & 1 \\
   & \multicolumn{7}{r}{x= \left[ \begin{array}{rr} x_{11} & x_{12} \\x_{21} & x_{22}
   \end{array}  \right]} & \in & K
   \end{array}
   \end{displaymath}%
\lthtmldisplayZ
\hfill\lthtmlcheckvsize\clearpage}

{\newpage\clearpage
\lthtmlinlinemathA{tex2html_wrap_inline188}%
$x(t) \in I\!\!R^n$%
\lthtmlinlinemathZ
\hfill\lthtmlcheckvsize\clearpage}

{\newpage\clearpage
\lthtmldisplayA{displaymath82}%
\begin{displaymath}
   x(t)=Ax(t-1)+bu(t), \quad t=1,\ldots,T
   \end{displaymath}%
\lthtmldisplayZ
\hfill\lthtmlcheckvsize\clearpage}

{\newpage\clearpage
\lthtmlinlinemathA{tex2html_wrap_inline196}%
$b \in I\!\!R^n$%
\lthtmlinlinemathZ
\hfill\lthtmlcheckvsize\clearpage}

{\newpage\clearpage
\lthtmlinlinemathA{tex2html_wrap_inline202}%
$x(T)=\bar{x}$%
\lthtmlinlinemathZ
\hfill\lthtmlcheckvsize\clearpage}

{\newpage\clearpage
\lthtmldisplayA{displaymath85}%
\begin{displaymath}
   f(u(t)) = \left\{ \begin{array}{ll}
   | u(t) | & \mbox{if } | u(t) |  \leq 1  \\
   2| u(t) | - 1 & \mbox{if } | u(t) | > 1
   \end{array}   \right.
   \end{displaymath}%
\lthtmldisplayZ
\hfill\lthtmlcheckvsize\clearpage}

{\newpage\clearpage
\lthtmldisplayA{displaymath92}%
\begin{displaymath}
   \begin{array}{lrcrr} 
    \min &   x_1^2 + 4 x_2^2 + 6x_3^2 \\
    \mbox{subject to} & 3(x_1-4)^2 + 4(x_2+2)^2 + (x_3-5)^2 & \leq & 40 & \quad (NLP)  \\
    & -x_1  & \leq & -2
   \end{array}
   \end{displaymath}%
\lthtmldisplayZ
\hfill\lthtmlcheckvsize\clearpage}

{\newpage\clearpage
\lthtmlinlinemathA{tex2html_wrap_inline210}%
$\theta(u)=\min\{L(x,u) : x \in I\!\!R^3\}$%
\lthtmlinlinemathZ
\hfill\lthtmlcheckvsize\clearpage}

{\newpage\clearpage
\lthtmldisplayA{displaymath100}%
\begin{displaymath}
   \max_{u \geq 0} \; \theta(u)
   \end{displaymath}%
\lthtmldisplayZ
\hfill\lthtmlcheckvsize\clearpage}

{\newpage\clearpage
\lthtmlinlinemathA{tex2html_wrap_inline212}%
$x(\bar{u})$%
\lthtmlinlinemathZ
\hfill\lthtmlcheckvsize\clearpage}

{\newpage\clearpage
\lthtmlinlinemathA{tex2html_wrap_inline214}%
$\theta(\bar{u})=L(x(\bar{u}),\bar{u})$%
\lthtmlinlinemathZ
\hfill\lthtmlcheckvsize\clearpage}

{\newpage\clearpage
\lthtmlinlinemathA{tex2html_wrap_inline216}%
$g(x(\bar{u}))$%
\lthtmlinlinemathZ
\hfill\lthtmlcheckvsize\clearpage}

{\newpage\clearpage
\lthtmlinlinemathA{tex2html_wrap_inline220}%
$\bar{u}$%
\lthtmlinlinemathZ
\hfill\lthtmlcheckvsize\clearpage}

{\newpage\clearpage
\lthtmldisplayA{displaymath109}%
\begin{displaymath}
   \theta(u) \leq \theta(\bar{u}) + g(x(\bar{u}))^T(u-\bar{u})
   \end{displaymath}%
\lthtmldisplayZ
\hfill\lthtmlcheckvsize\clearpage}

{\newpage\clearpage
\lthtmlinlinemathA{tex2html_wrap_inline222}%
$u^1, \ldots, u^p$%
\lthtmlinlinemathZ
\hfill\lthtmlcheckvsize\clearpage}

{\newpage\clearpage
\lthtmldisplayA{displaymath114}%
\begin{displaymath}
   \begin{array}{lrcllr}
   \max_{z,u} & z \\
   \mbox{subject to} & z & \leq &  \theta(u^i) + g(x(u^i))^T(u-u^i) & i=1,\ldots,p & \quad (DAp) \\
   & u & \geq & 0
   \end{array}
   \end{displaymath}%
\lthtmldisplayZ
\hfill\lthtmlcheckvsize\clearpage}


\end{document}
